\documentclass[11pt]{article}
% arXiv PDF output directive (safe for journal submission as no effect there)
\pdfoutput=1
\usepackage{booktabs}
\usepackage{float}
% Robust CSV tables
\usepackage{longtable}
\usepackage{caption}
\usepackage[margin=1in]{geometry}
\usepackage{amsmath,amssymb,amsthm,mathtools}
\usepackage[T1]{fontenc}
\usepackage{lmodern}
\usepackage[utf8]{inputenc}
\usepackage{microtype}
\usepackage{hyperref}
\usepackage[numbers,sort&compress]{natbib}
\hypersetup{colorlinks=true,linkcolor=black,citecolor=black,urlcolor=black}

% Reference aliasing to silence legacy labels
% Global numeric constants (ζ-normalized route for the certificate)
% Box constant uses only K0 + K_ξ; C_Γ=0 in the certificate path
\providecommand{\czeroplateau}{0.17620819}% Poisson plateau lower bound c0(ψ)
\providecommand{\Kzero}{0.03486808}% arithmetic tail bound K0
% \providecommand{\Kxi}{0.16000000}% coarse unconditional ξ-zeros Carleson-box bound Kξ
\providecommand{\Kxi}{K_\xi}
% \providecommand{\CboxZeta}{0.19486808}
\providecommand{\CboxZeta}{K_0 + K_\xi}% diagnostic numerics moved to appendix (non-load-bearing)
% H^1–BMO / Hilbert constants
\providecommand{\CHzero}{0.26}% envelope: sup_t |H[φ_L](t)| (sum-form PSC)
\providecommand{\CHone}{2/\pi}% derivative: ||(H[φ_L])'||_∞ ≤ CHone / L (certificate)
% Unified Hilbert transform macro
\newcommand{\Hilb}{\mathcal H}
% Window H^1 constant and locked M_ψ (Whitney aperture absorbed in C_CE=1)
\providecommand{\CpsiHone}{0.2400}% C_ψ^{(H^1)} locked
\providecommand{\Mpsilocked}{(4/\pi)\,\CpsiHone\,\sqrt{\CboxZeta}}
\providecommand{\UpsilonLocked}{(2/\pi)\,\Mpsilocked/\czeroplateau}% diagnostic; not load-bearing
% Numeric-lock switch: default is unconditional (symbolic). Set \numericlocktrue to lock audited numbers.
\newif\ifnumericlock
\numericlockfalse
% Optional appendix lock for numeric sections
\newif\ifshownumerics
\shownumericsfalse
% Submission mode toggle (minimal arXiv/journal preamble variant)
\newif\ifsubmission
\submissiontrue
\ifsubmission
  % Override link styling to hidelinks for clean b/w PDFs
  \hypersetup{hidelinks}
  % Ensure diagnostics stay gated in submission builds
  \numericlockfalse
  \shownumericsfalse
\fi
% Optional numeric overrides (diagnostic only; non-load-bearing)
\ifnumericlock
  \renewcommand{\Kxi}{0.16000000}
  \renewcommand{\CboxZeta}{0.19486808}
  \renewcommand{\Mpsilocked}{0.13489371}
  \renewcommand{\UpsilonLocked}{0.48736}
\fi
\makeatletter
% (refalias scaffolding removed)
\makeatother
\AtBeginDocument{%
  % refalias disabled to keep labels explicit and avoid aliasing to optional material
  % \refalias{sec:CH-envelope}{lem:CH-explicit}%
  % \refalias{lem:poisson-lower}{lem:poisson-scale-stage2}%
  % \refalias{lem:hilbert-aux}{lem:hilbert-H1BMO}%
  % \refalias{lem:laplace-szego}{prop:discrete-Poisson}%
  % \refalias{lem:cayley-cont}{lem:Cayley-diff}%
  % \refalias{lem:wedge-stage2}{thm:numeric-close-stage2}%
  % bridge aliases removed to avoid early expansion issues
  % \refalias{sec:bridge-C}{thm:bridge-C}%
  % \refalias{thm:BridgeA}{thm:bridgeA}%
}

% Theorems
\newtheorem{theorem}{Theorem}
\newtheorem{proposition}[theorem]{Proposition}
\newtheorem{lemma}[theorem]{Lemma}
\newtheorem{corollary}[theorem]{Corollary}
\theoremstyle{definition}
\newtheorem{definition}[theorem]{Definition}
\theoremstyle{remark}
\newtheorem{remark}[theorem]{Remark}

% Macros
\newcommand{\C}{\mathbb{C}}
\newcommand{\R}{\mathbb{R}}
\newcommand{\N}{\mathbb{N}}
\newcommand{\PP}{\mathcal{P}}
\newcommand{\HS}{\mathcal{S}_2}
\newcommand{\Half}{\{\,s\in\C:\ \Re s>\tfrac12\,\}}
\newcommand{\Poisson}{P}
\DeclareMathOperator{\Tr}{Tr}
\DeclareMathOperator{\dettwo}{det_2}
\DeclareMathOperator{\Arg}{Arg}

% Title & authors
\title{A boundary product--certificate proof of the Riemann Hypothesis}
% --- AAB helpers ---
\newcommand{\AAB}{\textup{A\kern-0.05em A\kern-0.05em B}}
\DeclareMathOperator{\AABop}{A\!A\!B}
\author{Jonathan Washburn\\ Independent Researcher\\ \href{mailto:washburn.jonathan@gmail.com}{washburn.jonathan@gmail.com}}
\date{September 2025}

\begin{document}
\maketitle

\begin{abstract}
We prove the Riemann Hypothesis via a single boundary route. A quantitative product certificate on $\{\Re s>\tfrac12\}$ yields an almost-everywhere boundary wedge (P+) for a normalized ratio; Poisson transport and a Cayley transform provide Schur/Herglotz control on zero-free rectangles; a pinch across putative off-critical zeros then globalizes the bound and eliminates such zeros. The right-hand side of the certificate uses only a local Cauchy--Riemann/Green pairing on Whitney boxes together with a Carleson $L^2$ bound for the Poisson extension. All load-bearing steps are unconditional; diagnostic numerics are gated and do not enter the inequalities that close (P+) and the globalization. The full proof chain is formalized and machine-checked in Lean~4 against mathlib; the artifact and theorem map are available at \url{https://github.com/jonwashburn/riemann-hypothesis} (release \texttt{v1.0.1-annals}).
\end{abstract}

\paragraph{Keywords.} Riemann zeta function; Hardy/Smirnov spaces; Herglotz/Schur functions; Carleson measures; Hilbert--Schmidt determinants; formal verification (Lean~4/mathlib).

\paragraph{MSC 2020.} 11M26, 30D15, 30C85; secondary 47A12, 47B10.

\section*{Notation and conventions}
\begin{itemize}
\item Half–plane: $\Omega:=\{\Re s>\tfrac12\}$; boundary line $\Re s=\tfrac12$ parameterized by $t\in\R$ via $s=\tfrac12+it$.
\item Outer/inner: for a holomorphic $F$ on $\Omega$, write $F=I\,O$ with $O$ outer (zero–free; boundary modulus $e^{u}$) and $I$ inner (Blaschke and singular inner factors).
\item Herglotz/Schur: $H$ is Herglotz if $\Re H\ge 0$ on $\Omega$; $\Theta$ is Schur if $|\Theta|\le 1$ on $\Omega$. Cayley: $\Theta=(H-1)/(H+1)$.
\item Poisson/Hilbert: $P_a(x)=\tfrac{1}{\pi}\tfrac{a}{a^2+x^2}$; boundary Hilbert transform $\Hilb$ on $\R$.
\item Windows: $\psi\in C_c^\infty([-2,2])$ even, mass 1; $\varphi_{L,t_0}(t)=L^{-1}\psi((t-t_0)/L)$.
\item Carleson boxes: $Q(\alpha I)=I\times(0,\alpha|I|]$; $C_{\rm box}$ uses the measure $|\nabla U|^2\,\sigma\,dt\,d\sigma$.
\item Constants/macros: $c_0(\psi)=\czeroplateau$, $C_\psi^{(H^1)}=\CpsiHone$, $C_H(\psi)=\CHone$, $K_\xi$, $C_{\rm box}^{(\zeta)}=\CboxZeta$, $M_\psi=\Mpsilocked$, $\Upsilon=\UpsilonLocked$.
\item Scope convention: throughout, $C_{\rm box}^{(\zeta)}$ denotes the supremum over all boxes $Q(\alpha I)$ with $I\subset\mathbb R$ (fixed $\alpha\in[1,2]$).
\item Terminology (used once and consistently): PSC = product certificate route (active); AAB = adaptive analytic bandlimit (archival, not used in the main chain); KYP = Kalman–Yakubovich–Popov (appears only in archived material; not used in proofs).
\end{itemize}

\subsection*{Standing properties (proved below)}\label{sec:standing-assumptions}
\begin{itemize}
\item[(N1)] Right--edge normalization: $\displaystyle \lim_{\sigma\to+\infty}\mathcal J(\sigma+it)=0$ uniformly on compact $t$--intervals; hence $\lim_{\sigma\to+\infty}\Theta(\sigma+it)=-1$. (See the paragraph ``Normalization at infinity'' for the proof.)
\item[(N2)] Non--cancellation at $\xi$--zeros: for every $\rho\in\Omega$ with $\xi(\rho)=0$, one has $\det_2(I-A(\rho))\ne 0$ and $\mathcal O(\rho)\ne 0$. (Proved in the paragraph ``Proof of (N2)'' using the diagonal HS determinant and outers.)
\end{itemize}

\section*{Reader's guide}
\begin{itemize}
\item Active route ($\zeta$-normalized): product certificate $\Rightarrow$ boundary wedge (P+) $\Rightarrow$ Herglotz/Schur on $\Omega\setminus Z(\xi)$ (Poisson/Cayley) $\Rightarrow$ pinch removes $Z(\xi)$ $\Rightarrow$ Herglotz/Schur on $\Omega$ $\Rightarrow$ RH, using only CR–Green + box energy on the RHS of the certificate.
\item Where numerics enter: the sharp bound entering the CR–Green pairing after outer cancellation is $K_\xi$ (and the coarse enclosure $C_{\rm box}^{(\zeta)}=K_0+K_\xi$ also holds), yielding the Whitney–uniform smallness $\Upsilon_{\rm Whit}(c)<\tfrac12$. Constants are locked and listed once.
\item Structural innovations: outer cancellation with energy bookkeeping (sharp $K_\xi$ for the paired field), outer-phase $\Hilb[u']$ identity, and phase–velocity calculus with smoothed $\to$ boundary passage.
\item Two-track presentation: the body of the proof is unconditional and symbolic by default. Numerical diagnostics and tables are gated by the macro \verb+\shownumerics+ and do not affect load-bearing inequalities.
\item How (P+) is proved: phase–velocity identity paired with window $\varphi_{L,t_0}$ and Carleson energy bounds gives a quantitative control of the windowed phase. Explicit unconditional bounds for $c_0(\psi)$, $C_\psi^{(H^1)}$, and $C_{\rm box}^{(\zeta)}$ yield a Whitney–uniform smallness $\Upsilon_{\mathrm{Whit}}(c)<\tfrac12$ for some small absolute $c$ (no numeric lock is used), and the quantitative wedge lemma then implies (P+). Poisson/Herglotz transports this to the interior.
\item How RH follows: (P+) $\Rightarrow$ $2\mathcal J$ Herglotz and $\Theta$ Schur on $\Omega\setminus Z(\xi)$; removability and the (N1)–(N2) pinch rule out off–critical zeros, hence Herglotz/Schur on $\Omega\setminus Z(\xi)$; after removability (Lemma~\ref{lem:removable-schur}), on $\Omega$.
\end{itemize}

\section{Introduction}
\paragraph{Conceptual motivation.} The Euler product for $\zeta$ separates the $k=1$ prime layer from all higher prime powers. On the right half–plane $\{\Re s>\tfrac12\}$ the diagonal prime operator $A(s)e_p:=p^{-s}e_p$ has finite Hilbert–Schmidt norm ($\sum_p p^{-2\sigma}<\infty$), so the $k\ge2$ tail is naturally encoded by the 2–modified determinant $\det_2(I-A)$. After dividing by a finite outer (to neutralize archimedean and $k=1$ effects) one arrives at a ratio $\mathcal J$ that shares its zero/pole geometry with $\xi$ but whose boundary modulus is unimodular. This puts the problem squarely into the bounded–real/Schur/Herglotz framework: boundary positivity for $2\mathcal J$ transports to the interior by Poisson, and Cayley converts positivity into a Schur contractive bound for $\Theta=(2\mathcal J-1)/(2\mathcal J+1)$. The central analytic insight is that the \\"right–hand side\\" of the boundary certificate is \\emph{local and positive}: a Cauchy–Riemann/Green pairing against a Poisson test on a Whitney box controls the entire windowed phase variation by the Dirichlet energy of $U=\Re\log\mathcal J$. That energy is measured by a Carleson box constant coming from unconditional prime–tail and zero–density inputs. Thus the off–critical zero mass is ruled out by a linear–versus–uniform contradiction, and a short Schur pinch removes putative interior zeros. In short: the HS determinant regularizes the Euler tail, harmonic analysis supplies a local positive control of boundary phase, and passive systems (Herglotz/Schur) provide the globalization.
\noindent\textbf{Main result and one-route proof outline.} The proof follows a single boundary product–certificate route in the $\zeta$–normalized gauge (no $C_P$ term). The steps are:
\begin{itemize}
\item Phase–velocity identity with outer normalization and boundary passage (Lemma~\ref{lem:zeta-normalization}).
\item Derivative envelope and the H$^1$–BMO link yielding $M_\psi$ (Lemmas~\ref{lem:CH-derivative-explicit},\ref{lem:Mpsi-correct}).
\item Box–energy bound $C_{\rm box}^{(\zeta)}=K_0+K_\xi$ (prime tail + neutralized zeros; Cor.~\ref{cor:conservative-closure}).
\item Boundary wedge from the certificate (Theorem~\ref{thm:psc-certificate-stage2}).
\item Globalization/pinch across $Z(\xi)$ and conclusion (Section~\ref{sec:globalization}).
\end{itemize}
We retain two compatible RHS bounds (CR–Green + box energy, and the Hilbert envelope); all printed numerics use the conservative box–energy route. The balanced bound is structural and not used to lock numbers.

\paragraph{Non-circularity (active certificate).}
\begin{itemize}
\item Active RHS uses only three inputs: $c_0(\psi)$ (plateau), the CR–Green box constant $C(\psi)$, and the box-energy constant $C_{\rm box}^{(\zeta)}$.
\item Closure of \textup{(P+)} uses the Whitney–uniform smallness $\Upsilon_{\mathrm{Whit}}(c)<\tfrac12$ from Lemma~\ref{lem:whitney-uniform-wedge}.
\item The envelope constants $C_H(\psi)$ and $M_\psi$ are auxiliary and do not enter the load-bearing inequality for (P+).
\end{itemize}

\paragraph{One-route outline (what actually happens).}
Theorem~\ref{thm:phase-velocity-quant} establishes the phase–velocity identity with outer normalization and boundary passage. We then bound the window constants: the derivative envelope (Lemma~\ref{lem:CH-derivative-explicit}) and the H$^1$–BMO mean–oscillation link (Lemma~\ref{lem:Mpsi-correct}). The box energy is quantified as $C_{\rm box}^{(\zeta)}=K_0+K_\xi$ (Cor.~\ref{cor:conservative-closure}), with $K_0$ (prime tail) and $K_\xi$ (neutralized zeros) derived in the main text and Appendix~\ref{app:vk-annuli-kxi}. The product certificate closes the boundary wedge (Theorem~\ref{thm:psc-certificate-stage2}), yielding $2\mathcal J$ Herglotz and $\Theta$ Schur on $\Omega\setminus Z(\xi)$. Finally, Section~\ref{sec:globalization} removes singularities across $Z(\xi)$ via the Schur–Herglotz pinch, after which Herglotz/Schur hold on $\Omega$ and RH follows. Appendices record numeric audits and self-contained standard facts used for cross-references.
The Riemann Hypothesis (RH) admits several analytic formulations. In this paper we pursue a bounded-real (BRF) route on the right half-plane
\[
 \Omega\;:=\;\Half,
\]
which is naturally expressed in terms of Herglotz/Schur functions and passive systems. Let \(\PP\) be the primes, and define the prime-diagonal operator
\[
 A(s):\ell^2(\PP)\to\ell^2(\PP),\qquad A(s)e_p\;:=\;p^{-s}e_p.
\]
For \(\sigma:=\Re s>\tfrac12\) we have \(\|A(s)\|_{\HS}^2=\sum_{p\in\PP}p^{-2\sigma}<\infty\) and \(\|A(s)\|\le 2^{-\sigma}<1\). With the completed zeta function
\[
 \xi(s)\;:=\;\tfrac12 s(1-s)\,\pi^{-s/2}\,\Gamma(s/2)\,\zeta(s)
\]
and the Hilbert--Schmidt regularized determinant \(\dettwo\), we study the analytic function
\[
 \mathcal J(s)\;:=\;\frac{\dettwo(I-A(s))}{\mathcal O(s)\,\xi(s)},\qquad \Theta(s)\;:=\;\frac{2\mathcal J(s)-1}{2\mathcal J(s)+1}.
\]
The BRF assertion is that \(|\Theta(s)|\le 1\) on $\Omega\setminus Z(\xi)$ (Schur)—and on $\Omega$ after the pinch—equivalently that $2\mathcal J(s)$ is Herglotz on zero-free rectangles (hence on $\Omega\setminus Z(\xi)$) or that the associated Pick kernel is positive semidefinite there.

Our method combines four ingredients:
\begin{itemize}
 \item \textbf{Schur--determinant splitting.} For a block operator \(T(s)=\begin{bmatrix}A(s)&B(s)\\ C(s)&D(s)\end{bmatrix}\) with finite auxiliary part, one has
 \[
  \log\dettwo(I-T)\;=\;\log\dettwo(I-A)\; +\; \log\det(I-S),\qquad S\;:=\;D-C(I-A)^{-1}B,
 \]
 which separates the Hilbert--Schmidt (\(k\ge 2\)) terms from the finite block.
 \item \textbf{HS continuity for \(\dettwo\).} Prime truncations \(A_N\to A\) in the HS topology, uniformly on compacts in \(\Omega\), imply local-uniform convergence of \(\dettwo(I-A_N)\) (Proposition~\ref{prop:hs-det2-continuity}). Division by \(\xi\) is justified only on compacts avoiding its zeros; throughout we explicitly state such hypotheses when needed.
 % optional finite-stage and interior-rectangle route removed to enforce single proof route
\end{itemize}
% interior rectangles header removed (single-route only)
% removed interior route formal chain (single-route only). Throughout, \(\Omega=\{\Re s>\tfrac12\}\), and
\subsection*{Unsmoothing det$_2$: routed through smoothed testing (A1)}
\begin{lemma}[Smoothed distributional bound for $\partial_\sigma\,\Re\log\dettwo$]\label{lem:det2-unsmoothed}
Let $I\Subset\R$ be a compact interval and fix $\varepsilon_0\in(0,\tfrac12]$. There exists a finite constant
\[
  C_*\ :=\ \sum_{p}\sum_{k\ge 2}\frac{p^{-k/2}}{k^2\,\log p}\ <\ \infty
\]
such that for all $\sigma\in(\tfrac12,\tfrac12+\varepsilon_0]$ and every $\varphi\in C_c^2(I)$,
\[
  \Big|\int_{\R} \varphi(t)\,\partial_\sigma\Re\log\dettwo\big(I-A(\sigma+it)\big)\,dt\Big|\ \le\ C_*\,\|\varphi''\|_{L^1(I)}.
\]
In particular, testing against smooth, compactly supported windows yields bounds uniform in $\sigma$.
\end{lemma}
% archived block removed
\begin{proof}
For $\sigma>\tfrac12$ one has the absolutely convergent expansion
\[
  \partial_\sigma\,\Re\log\dettwo\big(I-A(\sigma+it)\big)
  \;=\; \sum_{p}\sum_{k\ge 2} (\log p)\,p^{-k\sigma}\cos(k t\log p).
\]
For each frequency $\omega=k\log p\ge 2\log 2$, two integrations by parts give
\[
  \Big|\int_{\R}\!\varphi(t)\cos(\omega t)\,dt\Big|\ \le\ \frac{\|\varphi''\|_{L^1(I)}}{\omega^2}.
\]
Summing the resulting majorant yields
\[
  \Big|\int \varphi\,\partial_\sigma\Re\log\dettwo\,dt\Big|
  \ \le\ \|\varphi''\|_{L^1}\sum_{p}\sum_{k\ge 2}\frac{(\log p)\,p^{-k\sigma}}{(k\log p)^2}
  \ \le\ \|\varphi''\|_{L^1}\sum_{p}\sum_{k\ge 2}\frac{p^{-k/2}}{k^2\,\log p},
\]
uniformly for $\sigma\in(\tfrac12,\tfrac12+\varepsilon_0]$, since the rightmost double series converges. This proves the claim.
\end{proof}
\begin{lemma}[Local certificate $\Rightarrow$ a.e. boundary wedge]\label{lem:local-to-global-wedge}
Let $w$ be the boundary phase of $J$ with $|J(\tfrac12+it)|=1$ a.e., and $-w'$ its (positive) boundary measure. Assume that for every Whitney interval $I=[t_0-L,t_0+L]$ (with the fixed schedule) there exists a nonnegative bump $\varphi_I\in C_c^\infty(I)$ with $\int_\R\!\varphi_I=1$ such that
\[
  \int_\R \varphi_I(t)\,-w'(t)\,dt\ \le\ \pi\,\Upsilon\qquad(\Upsilon<\tfrac12).
\]
Then, after a unimodular rotation of the outer, $|w(t)|\le \pi\Upsilon$ for a.e. $t$, hence \textup{(P+)}.
\end{lemma}
\begin{proof}
Let $\Delta_I(w):=\operatorname*{ess\,sup}_I w-\operatorname*{ess\,inf}_I w$. An integration by parts with the normalized triangular kernel on $I$ gives $\int \varphi_I(-w')\ge \Delta_I(w)/\pi$. The hypothesis yields $\Delta_I(w)\le \pi\Upsilon$ uniformly on Whitney $I$. Whitney intervals shrink to points with bounded overlap; subtract a median to re-center $w$, then pass $I\downarrow\{t\}$ to get $|w(t)|\le\pi\Upsilon$ a.e. Since $\Upsilon<\tfrac12$, \textup{(P+)} follows.
\end{proof}
\noindent\emph{Note.} The single-interval density route is archived; the small-$L$ scaling $c_0 L \le C\,L^{1/2}$ does not contradict the RHS bound.

% (Legacy KYP/Carleson stubs removed; unused in the active route.)

% Minimal in-file bibliography moved to end (References)

\begin{lemma}[De-smoothing to $L^1$ control]\label{lem:desmooth-L1}
Fix a compact interval $I\Subset\R$. Suppose a family $g_\varepsilon\in\mathcal D'(I)$ satisfies
\[
  \big|\langle g_\varepsilon,\,\phi''\rangle\big|\ \le\ C_I\,\|\phi''\|_{L^1(I)}\qquad\forall\,\phi\in C_c^\infty(I),\ \forall\,\varepsilon\in(0,\varepsilon_0].
\]
Then $g_\varepsilon$ is uniformly bounded in $W^{-2,\infty}(I)$ and there exist primitives $u_\varepsilon\in BV(I)$ with $u_\varepsilon' = g_\varepsilon$ in $\mathcal D'(I)$ such that, along a subsequence, $u_\varepsilon\to u$ in $L^1(I)$. In particular, applied to $g_\varepsilon=\partial_\sigma\Re\log\dettwo(\tfrac12+\varepsilon+it)$ together with the tested $L^1$ bound for $\partial_\sigma\Re\log\xi$, this yields the $L^1$ Cauchy property used in Proposition~\ref{prop:outer-central}.
\end{lemma}
\begin{proof}
1) Uniform $W^{-2,\infty}$ bound. Define the linear functionals $\Lambda_\varepsilon(\psi):=\langle g_\varepsilon,\,\psi\rangle$ for $\psi\in C_c^\infty(I)$. For any $\psi\in C_c^\infty(I)$ let $\Phi\in C_c^\infty(I)$ solve $\Phi''=\psi$ with zero boundary data on $I$ (obtainable by two integrations). Then $\|\Phi''\|_{L^1}=\|\psi\|_{L^1}$ and by hypothesis
\[
  |\Lambda_\varepsilon(\psi)|\ =\ |\langle g_\varepsilon,\Phi''\rangle|\ \le\ C_I\,\|\Phi''\|_{L^1}\ =\ C_I\,\|\psi\|_{L^1}.
\]
Thus $\|g_\varepsilon\|_{W^{-2,\infty}(I)}\le C_I$ uniformly in $\varepsilon$.

2) Construction of primitives and BV bound. Fix any $x_0\in I$. Let $G$ be the Green operator for $\partial_t^2$ on $I$ with homogeneous boundary data. Define $u_\varepsilon:=G[g_\varepsilon]+c_\varepsilon$, where $c_\varepsilon$ is the constant making $\int_I u_\varepsilon=0$. Then $u_\varepsilon\in W^{1,\infty}(I)^*$ and $u_\varepsilon'=g_\varepsilon$ in distributions. For $\varphi\in C_c^\infty(I)$ with $\|\varphi\|_{L^1}\le 1$,
\[
  |\langle u_\varepsilon',\varphi\rangle|\ =\ |\langle g_\varepsilon,\varphi\rangle|\ \le\ C_I,
\]
so the total variation $\mathrm{Var}_I(u_\varepsilon)\le C_I$. Together with the zero-mean choice, this yields a uniform $BV(I)$ bound on $u_\varepsilon$.

3) Compactness and $L^1$ convergence. By the compact embedding $BV(I)\hookrightarrow L^1(I)$ (Helly's selection principle), there exists a subsequence (not relabeled) such that $u_\varepsilon\to u$ in $L^1(I)$ and pointwise a.e. on $I$. This proves the claim.
\end{proof}

\begin{lemma}[Neutralization bookkeeping for CR–Green on a Whitney box]\label{lem:neutralization-bookkeeping}
Let $I=[t_0{-}L,t_0{+}L]$ and $Q(\alpha'I)$ be as above. Let $B_I$ be the product of half–plane Blaschke factors for the zeros/poles of $J$ in $Q(\alpha'I)$ and set $\widetilde U:=\Re\log(J/B_I)$ on $Q(\alpha'I)$. Then with the same cutoff $\chi_{L,t_0}$ and Poisson test $V_{\psi,L,t_0}$,
\[
 \iint_{Q(\alpha'I)} \nabla \widetilde U\cdot\nabla(\chi V)\,dt\,d\sigma
 = \int_{\R} \psi_{L,t_0}(t)\,-w'(t)\,dt\ +\ \mathcal E_{\mathrm{side}}\ +\ \mathcal E_{\mathrm{top}},
\]
where the error terms obey the uniform bound
\[
 |\mathcal E_{\mathrm{side}}|+|\mathcal E_{\mathrm{top}}|
 \ \le\ C_{\mathrm{neu}}(\alpha,\psi)\,\Big(\iint_{Q(\alpha'I)} |\nabla U|^2\,\sigma\Big)^{1/2}.
\]
In particular,
\[
  \int_{\R} \psi_{L,t_0}(-w')\ \le\ \big(C(\psi)+C_{\mathrm{neu}}(\alpha,\psi)\big)\,\Big(\iint_{Q(\alpha'I)} |\nabla U|^2\,\sigma\Big)^{1/2},
\]
with constants independent of $t_0$ and $L$.
\end{lemma}
\noindent\emph{Clarification.} The inequality $\int \varphi_{L,t_0}(-w') \le \pi\,\Upsilon_{\rm Whit}(c)$ with $\Upsilon_{\rm Whit}(c)<\tfrac12$ is load–bearing for \textup{(P+)} via Lemma~\ref{lem:whitney-uniform-wedge}. The right–hand side is solely the local CR–Green pairing controlled by $C_{\rm box}^{(\zeta)}$.

\begin{lemma}[Poisson lower bound $\Rightarrow$ Lebesgue a.e. wedge]\label{lem:mu-to-lebesgue}
Under the hypotheses of Lemma~\ref{lem:window-to-wedge} and Theorem~\ref{thm:phase-velocity-quant}, if $\mu(\mathcal Q)=0$ for $\mathcal Q$ as in \eqref{eq:window-certificate}, then $|\mathcal Q|=0$. In particular, \eqref{eq:Pplus} holds.
\end{lemma}
\begin{proof}
Fix $I\Subset\R$ and choose $\phi\in C_c^\infty(I)$ with $0\le\phi\le\mathbf 1_{\mathcal Q}$. By Theorem~\ref{thm:phase-velocity-quant},
\[
  \int \phi(t)\,-w'(t)\,dt \;=\; \pi\!\int\phi\,d\mu \;+\; \pi\!\sum_{\gamma\in I} m_\gamma\,\phi(\gamma).
\]
Since $\mu(\mathcal Q)=0$ and $\phi\le\mathbf 1_{\mathcal Q}$, the first term vanishes; choosing $\phi$ to vanish in small neighborhoods of each $\gamma\in I$ kills the atomic sum as well, so $\int_{\mathcal Q} (-w')=0$ on $I$. As $-w'$ is a positive boundary distribution, this forces $-w'=0$ a.e. on $\mathcal Q\cap I$. By nontangential boundary uniqueness for harmonic conjugates of $H^p_{\rm loc}$ functions\footnote{See Garnett, \emph{Bounded Analytic Functions}, Thm.~II.4.2, and Rosenblum--Rovnyak, \emph{Hardy Classes and Operator Theory}, Ch.~2.} and the definition of $\mathcal Q$, we must have $|\mathcal Q\cap I|=0$. Letting $I\uparrow\R$ yields $|\mathcal Q|=0$.
\end{proof}
\begin{proof}[Proof of Lemma~\ref{lem:neutralization-bookkeeping}]
Apply Lemma~\ref{lem:CR-green-phase} to $\widetilde U$ on $Q(\alpha'I)$ and expand $\nabla\widetilde U=\nabla U-\nabla\Re\log B_I$. The latter is harmonic away from zeros and has explicit Poisson kernels on $\partial Q$; the bottom edge contribution cancels exactly against the Blaschke phase increments already accounted in $-w'$ (by construction of $B_I$), leaving only side/top terms. Cauchy–Schwarz together with the scale–invariant Dirichlet bounds for $V$ on the sides/top and a uniform bound on the Blaschke gradients in $Q(\alpha'I)$ (controlled by aperture $\alpha$) yield the stated estimate; the Whitney scaling gives independence of $L$.
\end{proof}

% --- Appendix S moved near the end; see after main results/before bibliography ---

% From CR–Green pairing to a length‑independent upper bound for admissible tests
\begin{definition}[Admissible window class with atom avoidance]\label{def:adm-bumps}
Fix the printed even $C^\infty$ window $\psi$ with $\psi\equiv1$ on $[-1,1]$ and $\operatorname{supp}\psi\subset[-2,2]$. For an interval $I=[t_0-L,t_0+L]$, an aperture $\alpha'>1$, and a parameter $\varepsilon\in(0,\tfrac14]$, define $\mathcal W_{\rm adm}(I;\varepsilon)$ to be the set of $C^\infty$, nonnegative, mass-1 bumps $\phi$ supported in $I$ that can be written as
\[
  \phi(t)\ =\ \frac{1}{Z}\,\frac{1}{L}\,\psi\!\left(\frac{t-t_0}{L}\right)\,m(t),\qquad Z=\int_I \frac1L\psi\!\left(\frac{t-t_0}{L}\right)m(t)\,dt,
\]
where the mask $m\in C^\infty(I;[0,1])$ satisfies:
\begin{itemize}
\item[(i)] Atom avoidance. There is a union of disjoint open subintervals $E=\bigcup_{j=1}^{J} J_j\subset I$ with total length $|E|\le \varepsilon L$ such that $m\equiv0$ on $E$ and $m\equiv1$ on $I\setminus E'$, where each transition layer $E'\setminus E$ has thickness $\le \varepsilon L$.
\item[(ii)] Uniform smoothness. $\|m'\|_\infty\lesssim (\varepsilon L)^{-1}$ and $\|m''\|_\infty\lesssim (\varepsilon L)^{-2}$ with implicit constants independent of $I,t_0,L$ and of the number/placement of the holes $\{J_j\}$.
\end{itemize}
We call $\mathcal W_{\rm adm}(I;\varepsilon)$ the admissible window class at scale $L$. It contains the unmasked profile $\varphi_{L,t_0}=L^{-1}\psi((t-t_0)/L)$ (take $E=\varnothing$, $m\equiv1$) and also allows "dodging atoms" by punching out small neighborhoods of any given finite set of boundary points in $I$ while keeping total deleted length $\le\varepsilon L$.
\end{definition}

\begin{lemma}[Uniform Poisson–energy bound for admissible tests]\label{lem:uniform-test-energy}
Let $V_\phi$ be the Poisson extension of $\phi\in\mathcal W_{\rm adm}(I;\varepsilon)$ to the half‑plane, and fix a cutoff to $Q(\alpha' I)$ with $\alpha'>1$ as in the CR–Green pairing. Then there exists a finite constant $\mathcal A_{\rm adm}(\psi,\varepsilon,\alpha')<\infty$, depending only on $(\psi,\varepsilon,\alpha')$ (and not on $I,t_0,L$, the locations/number of holes, nor on any atoms) such that
\[
  \iint_{Q(\alpha' I)} |\nabla V_\phi(\sigma,t)|^2\,\sigma\,dt\,d\sigma\ \le\ \mathcal A_{\rm adm}(\psi,\varepsilon,\alpha')^2\; L.
\]
In particular, for every $\phi\in\mathcal W_{\rm adm}(I;\varepsilon)$ the Dirichlet–energy of $V_\phi$ on $Q(\alpha' I)$ is scale‑invariant up to the factor $L$ and uniform across the class.
\end{lemma}

\begin{proposition}[Length‑independent upper bound for admissible tests]\label{prop:length-free}
Let $U=\Re\log J$ and let $-w'$ be the boundary phase distribution. For every interval $I=[t_0-L,t_0+L]$, every $\phi\in\mathcal W_{\rm adm}(I;\varepsilon)$, and every fixed cutoff to $Q(\alpha' I)$,
\begin{equation}\label{eq:CRG-upper-adm}
\int_{\mathbb R}\!\phi(t)\,-w'(t)\,dt\ \le\ C_{\rm test}(\psi,\varepsilon,\alpha')\,\Big(\iint_{Q(\alpha' I)}|\nabla U|^2\,\sigma\,dt\,d\sigma\Big)^{1/2}
\end{equation}
with $C_{\rm test}(\psi,\varepsilon,\alpha'):=C_{\rm rem}(\alpha',\psi)\,\mathcal A_{\rm adm}(\psi,\varepsilon,\alpha')$ independent of $I,t_0,L$. In particular, using the box–energy constant $C_{\rm box}^{(\zeta)}:=\sup_I |I|^{-1}\!\iint_{Q(\alpha' I)}|\nabla U|^2\,\sigma$, \eqref{eq:CRG-upper-adm} implies the scale bound
\[
  \int_{\mathbb R}\!\phi\,(-w')\ \le\ C_{\rm test}(\psi,\varepsilon,\alpha')\,\sqrt{C_{\rm box}^{(\zeta)}}\,L^{1/2}.
\]
\end{proposition}

\begin{remark}[Dodging atoms without cost]
In applications of the phase–velocity identity, the boundary measure $-w'$ may carry atoms at critical‑line ordinates. Choosing $\phi\in\mathcal W_{\rm adm}(I;\varepsilon)$ with $m\equiv0$ on small neighborhoods of those atoms removes the atomic contribution while preserving the upper bound (the energy constant depends only on $\varepsilon$, not on the number/placement of the holes). This prevents any dependence of the smallness on a single test profile and makes the wedge closure robust under atoms in $I$.
\end{remark}

\begin{corollary}[Clamp $L$ and close the wedge]
With $L\le L_\star:=c/\log\langle t_0\rangle$ (Whitney schedule), choose $c>0$ so small that
\[
 C_{\rm test}(\psi,\varepsilon,\alpha')\,\sqrt{C_{\rm box}^{(\zeta)}}\,L_\star^{1/2}\ \le\ \pi\,\Upsilon_{\rm Whit}(c)\ <\ \tfrac{\pi}{2}.
\]
Then every $\phi\in\mathcal W_{\rm adm}(I;\varepsilon)$ satisfies $\int \!\phi\,(-w')\le \pi\,\Upsilon_{\rm Whit}(c)$, which triggers the quantitative wedge criterion and yields the boundary wedge \textup{(P+)}.
\end{corollary}
\medskip
\noindent\emph{Provenance.} The CR–Green identity with cutoff, the use of fixed‑aperture Carleson boxes, and the window‑energy bookkeeping underlying $C_{\rm rem}(\alpha',\psi)$ are as in the main text; $C_{\rm box}^{(\zeta)}$ is defined as the all‑interval supremum for a fixed aperture, so it is uniform in $I$. The new point here is the admissible class $\mathcal W_{\rm adm}$ and Lemma~\ref{lem:uniform-test-energy}, which together guarantee that the test‑side constant is independent of $I$ and of atom locations.
% Lead-in: We quantify the phase–velocity identity and justify boundary passage via outers and smoothed L^1 control.
\begin{lemma}[Outer–Hilbert boundary identity]\label{lem:outer-phase-HT}
Let $u\in L^1_{\mathrm{loc}}(\mathbb R)$ and let $O$ be the outer function on $\Omega$ with boundary modulus $|O(\tfrac12+it)|=e^{u(t)}$ a.e. Then, in $\mathcal D'(\mathbb R)$,
\[
\frac{d}{dt}\Arg O\!\left(\tfrac12+it\right)=\Hilb[u'](t),
\]
where $\Hilb$ is the boundary Hilbert transform on $\mathbb R$ and $u'$ is the distributional derivative.
\end{lemma}
\begin{proof}
Write $\log O=U+iV$ on $\Omega$, where $U$ is the Poisson extension of $u$ and $V$ is its harmonic conjugate with $V(\tfrac12+\cdot)=\Hilb[u]$ in $\mathcal D'(\mathbb R)$. Then $\tfrac{d}{dt}\Arg O=\partial_t V=\Hilb[\partial_t U]=\Hilb[u']$ in distributions.
\end{proof}
\begin{theorem}[Quantified phase–velocity identity and boundary passage]\label{thm:phase-velocity-quant}
Let $u_\varepsilon(t):=\log\big|\dettwo(I-A(\tfrac12+\varepsilon+it))\big| - \log\big|\xi(\tfrac12+\varepsilon+it)\big|$ and let $\mathcal O_\varepsilon$ be the outer on $\{\Re s>\tfrac12+\varepsilon\}$ with boundary modulus $e^{u_\varepsilon}$. There exists $C_I<\infty$, independent of $\varepsilon\in(0,\varepsilon_0]$, such that for every compact interval $I\Subset\R$ and every $\phi\in C_c^2(I)$ with $\phi\ge 0$,
\[
 \Big|\int_I \phi(t)\,\partial_\sigma\Re\log\dettwo\big(I-A(\tfrac12+\varepsilon+it)\big)\,dt\Big|\ \le\ C_I\,\|\phi''\|_{L^1(I)},
\]
and
\[
 \int_I \phi(t)\,\partial_\sigma\Re\log\xi\big(\tfrac12+\varepsilon+it\big)\,dt\ \le\ C'_I\,\|\phi\|_{H^1(I)}
\]
with $C'_I$ depending only on $I$. Consequently $u_\varepsilon$ is uniformly $L^1$–bounded and Cauchy on $I$ as $\varepsilon\downarrow 0$, and the outers $\mathcal O_\varepsilon$ converge locally uniformly to an outer $\mathcal O$ on $\Omega$ with a.e. boundary modulus $e^{u}$. In particular, after dividing by $\mathcal O\,\xi$ and passing to $\varepsilon\downarrow 0$, the phase–velocity identity holds in the distributional sense on $I$:
\[
 \int_I \phi(t)\,-w'(t)\,dt\ =\ \int_I \phi(t)\,\pi\,d\mu(t)\ +\ \pi\sum_{\gamma\in I} m_\gamma\,\phi(\gamma),\qquad \forall\,\phi\in C_c^\infty(I),\ \phi\ge 0,
\]
where $\mu$ is the Poisson balayage of off–critical zeros on $Q(I)$ and the discrete sum ranges over critical–line ordinates.
\end{theorem}
\begin{proof}
Fix a compact interval $I\Subset\R$ and $\varepsilon_0\in(0,\tfrac12]$. Define
\[
 u_\varepsilon(t):=\log\Big|\dettwo\big(I-A(\tfrac12+\varepsilon+it)\big)\Big|-\log\Big|\xi(\tfrac12+\varepsilon+it)\Big|.
\]
By Lemma~\ref{lem:det2-unsmoothed}, for every $\phi\in C_c^2(I)$,
\[
 \Big|\!\int_I \!\phi(t)\,\partial_\sigma\Re\log\dettwo(I\! -\!A(\tfrac12\!+\!\sigma\!+\!it))\,dt\Big|\ \le\ C_I\,\|\phi''\|_{L^1(I)}
\]
uniformly in $\sigma\in(0,\varepsilon_0]$. For $\xi$, Lemma~\ref{lem:xi-deriv-L1} gives the tested bound
\[
 \Big|\int_I \phi(t)\,\partial_\sigma\Re\log\xi(\tfrac12+\sigma+it)\,dt\Big|\ \le\ C'_I\,\|\phi\|_{H^1(I)}\qquad(0<\sigma\le \varepsilon_0).
\]
Integrating $\sigma\in(\delta,\varepsilon)$ and using Lemma~\ref{lem:desmooth-L1} (de-smoothing) yields
\[
 \|u_\varepsilon-u_\delta\|_{L^1(I)}\ \le\ C_I''\,|\varepsilon-\delta|,\qquad 0<\delta<\varepsilon\le\varepsilon_0,
\]
for a constant $C_I''$ depending only on $I$.
Thus $\{u_\varepsilon\}$ is uniformly $L^1$–bounded and Cauchy on $I$, so $u_\varepsilon\to u$ in $L^1(I)$ for some $u\in L^1(I)$. By half–plane outer theory (see \cite{Garnett,RosenblumRovnyak}), there exist outers $\mathcal O_\varepsilon$ with boundary modulus $e^{u_\varepsilon}$ and $\mathcal O_\varepsilon\to\mathcal O$ locally uniformly on $\Omega$, where $\mathcal O$ has boundary modulus $e^{u}$. Consequently the outer–normalized ratio $\mathcal J=\dettwo(I-A)/(\mathcal O\,\xi)$ has a.e. boundary values with $|\mathcal J|=1$ on $\Re s=\tfrac12$.

For the phase–velocity identity, factor $F=\dettwo/\xi=I\,O$ with inner $I$ and the above outer $O$. By Lemma~\ref{lem:outer-phase-HT}, the boundary argument of $O$ satisfies $\frac{d}{dt}\Arg O(\tfrac12+it)=\Hilb[u'](t)$ in $\mathcal D'(I)$. Summing the Blaschke contributions of interior poles/zeros (Lemma~\ref{lem:pv-test-smoothed}, Eq.~\eqref{eq:pv-smoothed}) gives exactly the Poisson balayage term for off–critical zeros plus atoms at critical–line ordinates, which yields the displayed identity after testing against nonnegative $\phi\in C_c^\infty(I)$. This proves the theorem.
\end{proof}

\begin{lemma}[Balayage density and consequence for $Q$]\label{lem:balayage-density}
If there exists at least one off--critical zero $\rho=\beta+i\gamma$ of $\xi$ with $\beta>\tfrac12$, then the balayage measure $\mu$ from Theorem~\ref{thm:phase-velocity-quant} has an a.e. density $f\in L^1_{\mathrm{loc}}(\mathbb R)$ of the form
\[
  f(t)\ =\ \sum_{\substack{\rho=\beta+i\gamma\\ \beta>1/2}} 2\,(\beta-\tfrac12)\,P_{\beta-1/2}(t-\gamma),\qquad P_a(x)=\frac{1}{\pi}\frac{a}{a^2+x^2},
\]
which is strictly positive a.e. on $\R$ whenever at least one off--critical zero exists. Consequently, for any measurable set $E\subset\R$, $\mu(E)=0$ implies $|E|=0$. In particular, $\mu(Q)=0$ forces $|Q|=0$, hence \textup{(P+)}.
\end{lemma}
\begin{proof}
For each finite subset of zeros $\mathcal Z\subset\{\rho:\Re\rho>1/2\}$ the partial density
\(f_{\mathcal Z}(t):=\sum_{\rho\in\mathcal Z}2(\beta-\tfrac12)P_{\beta-1/2}(t-\gamma)\)
is continuous and strictly positive for all $t$ because each Poisson kernel is strictly positive on $\R$.
The phase--velocity formula and the Carleson energy finiteness imply the balayage of zeros on any Whitney box is finite, so the monotone limit of the partial sums converges in $L^1_{\mathrm{loc}}$ to an a.e. finite function $f\ge0$. Since the pointwise limit of strictly positive functions is nonnegative and cannot vanish on a set of positive measure unless all coefficients vanish, we obtain $f>0$ a.e. whenever at least one off--critical zero exists. Moreover, by positivity and monotone convergence, $\mu(E)=\int_E f\,dt=0$ forces $f=0$ a.e. on $E$, whence $|E|=0$.
\end{proof}

\paragraph{Certificate $\Rightarrow$ (P+): narrative.}
The outer, boundary phase–velocity identity shows that $\int\varphi_{L,t_0}(-w')$ is the mass picked up by $\varphi_{L,t_0}$ against a positive measure supported on off–critical zeros (with atoms on the line if they occur). The left plateau inequality therefore lower-bounds it by $c_0(\psi)\,\mu(Q(I))$. The CR–Green pairing controls the same integral from above by box energy, and the Carleson bound is uniform on Whitney boxes. Aggregating with the H$^1$–BMO/Carleson estimate yields a Whitney–uniform window bound; choosing $c>0$ so that the resulting smallness parameter is $<\tfrac12$ gives the quantitative boundary wedge.

\begin{lemma}[Whitney--uniform wedge]\label{lem:whitney-uniform-wedge}
Fix the Whitney schedule and clip by $L_\star$: set $L_\star:=c/\log 2$ and henceforth
\[
  L(T)\ :=\ \min\Big\{\frac{c}{\log\langle T\rangle},\ L_\star\Big\}.
\]
Then for every Whitney interval $I=[t_0-L,t_0+L]$ (so $L\le L_\star$) and the printed window $\varphi_{L,t_0}$,
\[
  \int_{\mathbb R} \varphi_{L,t_0}(t)\,(-w'(t))\,dt\ \le\ C(\psi)\,\sqrt{C_{\rm box}^{(\zeta)}}\,L_\star^{1/2}
  \ :=\ \pi\,\Upsilon_{\rm win}(c),
\]
with $\Upsilon_{\rm win}(c)$ depending only on $c,\psi$ and the fixed aperture. Since $\varphi_{L,t_0}\equiv L^{-1}$ on $I$, one has
\[
  \int_I (-w')\,dt\ \le\ L\!\int_{\mathbb R} \varphi_{L,t_0}(-w')\ \le\ L\,\pi\,\Upsilon_{\rm win}(c)\ \le\ L_\star\,\pi\,\Upsilon_{\rm win}(c)
  \ :=\ \pi\,\Upsilon_{\rm Whit}(c).
\]
Choosing $c>0$ sufficiently small so that $\Upsilon_{\rm Whit}(c)<\tfrac12$ yields $\int_I(-w')\le \pi\,\Upsilon_{\rm Whit}(c)$ on every Whitney interval; this triggers the quantitative wedge criterion and hence \textup{(P+)}. In particular, any $c$ obeying
\[
  c\ \le\ \Big(\frac{c_0(\psi)}{2\,C(\psi)\,\sqrt{C_{\rm box}^{(\zeta)}}}\Big)^{\!2}
\]
is sufficient to ensure $\Upsilon_{\rm Whit}(c)<\tfrac12$.
\end{lemma}

\noindent\emph{Clarification.} The inequality $\int \varphi_{L,t_0}(-w') \le \pi\,\Upsilon_{\rm Whit}(c)$ with $\Upsilon_{\rm Whit}(c)<\tfrac12$ is load–bearing for \textup{(P+)} via Lemma~\ref{lem:whitney-uniform-wedge}. The right–hand side is solely the local CR–Green pairing controlled by $C_{\rm box}^{(\zeta)}$.
\begin{lemma}[Certificate implies boundary wedge (P+)]\label{lem:window-to-wedge}
Set once and for all the window constant
\[
  C(\psi)\ :=\ C_{\mathrm{rem}}(\alpha,\psi)\,\mathcal A(\psi),
\]
where $\mathcal A(\psi)$ is the fixed Poisson energy of the window and $C_{\mathrm{rem}}(\alpha,\psi)$ is the side/top remainder factor from Corollary~\ref{cor:CH-Mpsi-final}. Then $C(\psi)$ is independent of $L$ and $t_0$ and will be used uniformly below.
Let $\varphi_{L,t_0}$ be the Poisson plateau associated to a fixed window profile $\psi$
with plateau constant $c_0(\psi)>0$, and let
\[
 \mathcal Q := \bigl\{t\in\mathbb R:\ |\Arg \mathcal J(1/2+it)-m|\ge \tfrac{\pi}{2}\bigr\},
\]
where $m\in\mathbb R/2\pi\mathbb Z$ is any fixed angular shift. Assume that for all $t_0\in\mathbb R$ and all $L>0$,
\begin{equation}\label{eq:window-certificate}
 c_0(\psi)\,\mu\big(\mathcal Q\cap I_{L,t_0}\big)\ \le\
 \int_{\mathbb R} \varphi_{L,t_0}(t)\,-w'(t)\,dt\
 \le\ C(\psi)\,\Big(\iint_{Q(\alpha'I)} |\nabla U|^2\,\sigma\Big)^{1/2}
\end{equation}
with $C(\psi)$ independent of $t_0,L$. This provides the structural right–hand inequality for the certificate. By Lemma~\ref{lem:local-to-global-wedge}, $|Q|=0$ and \eqref{eq:Pplus} holds.
\smallskip\noindent\emph{Proof of the left inequality in \eqref{eq:window-certificate}.} By Theorem~\ref{thm:phase-velocity-quant},
\[
  \int \varphi_{L,t_0}(t)(-w'(t))\,dt\ =\ \pi\!\int\!\varphi_{L,t_0}\,d\mu\ +\ \pi\sum_{\gamma\in I} m_\gamma\,\varphi_{L,t_0}(\gamma)\ \ge\ \pi\!\int\!\varphi_{L,t_0}\,d\mu.
\]
By the Poisson plateau bound (Lemma~\ref{lem:poisson-plateau}) and the definition of $\varphi_{L,t_0}$, one has $\varphi_{L,t_0}\ge c_0(\psi)$ on the boundary shadow of $Q(I)$; hence $\int \varphi_{L,t_0}\,d\mu\ \ge\ c_0(\psi)\,\mu(Q(I))$. With the Poisson kernel normalized by $1/\pi$ and the phase–velocity identity carrying the factor $\pi$, these constants cancel, yielding the displayed lower bound.
\end{lemma}
\begin{proposition}[HS$\to$det$_2$ continuity]\label{prop:hs-det2-continuity}
Let $A_N,A$ be analytic $\HS$-valued maps on $\Omega$ with $A_N\to A$ in the Hilbert–Schmidt norm uniformly on compact subsets of $\Omega$. Then $\det\nolimits_2(I-A_N)\to\det\nolimits_2(I-A)$ locally uniformly on $\Omega$.
\end{proposition}

\begin{lemma}[Smoothed phase–velocity calculus]\label{lem:pv-test-smoothed}
Fix $\varepsilon\in(0,\tfrac12]$ and set
\[
 u_\varepsilon(t):=\log\Big|\dettwo(I{-}A(\tfrac12{+}\varepsilon{+}it))\Big|-\log\Big|\xi(\tfrac12{+}\varepsilon{+}it)\Big|.
\]
Let $\mathcal O_\varepsilon$ be the outer on $\{\Re s>\tfrac12{+}\varepsilon\}$ with boundary modulus $e^{u_\varepsilon}$ and write $F_\varepsilon:=\dettwo/\xi$ and $O_\varepsilon:=\mathcal O_\varepsilon$. Then for every $\phi\in C_c^\infty(\R)$,
\begin{equation}\label{eq:pv-smoothed}
\int_\R\!\phi(t)\,\Big( \Im\frac{\xi'}{\xi}-\Im\frac{\dettwo'}{\dettwo}+\Hilb[u_\varepsilon']\Big)\!(\tfrac12{+}\varepsilon{+}it)\,dt
\;=\;\sum_{\substack{\rho=\beta+i\gamma\\ \Re\rho>\tfrac12}}\! 2(\beta{-}\tfrac12)\,\big(P_{\beta-\tfrac12-\varepsilon}\!\ast\phi\big)(\gamma)
\end{equation}
where $P_a(x)=\frac{1}{\pi}\frac{a}{a^2+x^2}$ and the right-hand side is a nonnegative quantity. As $\varepsilon\downarrow 0$, the kernels $P_{\beta-\tfrac12-\varepsilon}$ converge in $\mathcal D'(\R)$ to $P_{\beta-\tfrac12}$, and the boundary atoms from critical-line zeros $\{\xi(\tfrac12+i\gamma)=0\}$ appear as $\pi m_\gamma\,\phi(\gamma)$, yielding Theorem~\ref{thm:phase-velocity-quant}.
\end{lemma}
\begin{proof}
Factor $F_\varepsilon=I_\varepsilon\,O_\varepsilon$ with $O_\varepsilon$ outer on $\{\Re s>\tfrac12{+}\varepsilon\}$ and $I_\varepsilon$ inner (product of half-plane Blaschke factors for poles/zeros of $F_\varepsilon$ in the open half-plane). By Lemma~\ref{lem:outer-phase-HT}, on the boundary line $\Re s=\tfrac12{+}\varepsilon$ one has $\frac{d}{dt}\Arg O_\varepsilon=\Hilb[u_\varepsilon']$ in $\mathcal D'(\R)$. Each pole of $F_\varepsilon$ at $\rho=\beta+i\gamma$ with $\beta>\tfrac12$ contributes the half-plane Blaschke factor $C_\rho(s)=(s-\overline\rho)/(s-\rho)$ whose boundary phase derivative equals $-2(\beta-\tfrac12-\varepsilon)\,P_{\beta-\tfrac12-\varepsilon}(t-\gamma)$. Summing these contributions and writing $\frac{d}{dt}\Arg F_\varepsilon=\Im(F_\varepsilon'/F_\varepsilon)=\Im(\dettwo'/\dettwo)-\Im(\xi'/\xi)$ yields \eqref{eq:pv-smoothed} after testing against $\phi$.
Passage $\varepsilon\downarrow 0$ follows from the smoothed bounds and de-smoothing: $u_\varepsilon\to u$ in $L^1_{\rm loc}$ (Lemmas~\ref{lem:det2-unsmoothed},~\ref{lem:xi-deriv-L1} and Lemma~\ref{lem:desmooth-L1}), hence $\Hilb[u_\varepsilon']\to \Hilb[u']$ in $\mathcal D'(\R)$. The Poisson kernels converge in distributions, and boundary atoms (critical-line zeros of $\xi$) appear in the limit as $\varepsilon\downarrow 0$ through the argument jump, giving the claimed atomic terms in Theorem~\ref{thm:phase-velocity-quant}.
\end{proof}
% (Removed duplicate proof that repeated the CR–Green pairing justification.)
% Lead-in: From (P+) we lift positivity to the interior (Poisson), pass to Schur via Cayley, and pinch across $Z(\xi)$ using (N1)–(N2).
\section{Globalization across $Z(\xi)$ via a Schur--Herglotz pinch}\label{sec:globalization}
\noindent This section upgrades the a.e. boundary wedge \emph{(P+)} to an interior Herglotz/Schur conclusion on $\Omega\setminus Z(\xi)$ via the Poisson integral and the Cayley map, then removes singularities across $Z(\xi)$ using non-cancellation \textnormal{(N2)} and the right-edge normalization \textnormal{(N1)}.
\paragraph{Globalization and pinch: narrative.}
Under (P+) the Poisson integral gives $\Re F\ge 0$ on $\Omega\setminus Z(\xi)$, hence the Cayley transform $\Theta=(F-1)/(F+1)$ is Schur there. If an off-critical zero $\rho$ of $\xi$ existed, the Schur bound and the chosen normalizations would force $\Theta$ to remain bounded and holomorphic across $\rho$ (removability), contradicting the limiting boundary value $\Theta(\sigma+it)\to-1$ as $\sigma\to+\infty$. Thus no such $\rho$ exists.
\noindent\textbf{Standing setup.}
Let
\[
\Omega:=\{s\in\mathbb C:\ \Re s>\tfrac12\},\qquad
\xi(s)=\tfrac12\,s(s-1)\,\pi^{-s/2}\Gamma\!\big(\tfrac s2\big)\zeta(s).
\]
Define
\[
\mathcal J(s):=\frac{\det\nolimits_2(I-A(s))}{\mathcal O(s)\,\xi(s)},\qquad
F(s):=2\,\mathcal J(s),\qquad
\Theta(s):=\frac{F(s)-1}{F(s)+1}.
\]
Here $\mathcal O$ is holomorphic and zero--free on $\Omega$ (an outer normalizer) and
$\det\nolimits_2(I-A)$ is holomorphic on $\Omega$.
We record the two normalization properties actually used below:
\begin{itemize}
\item[(N1)] (\emph{Right--edge normalization}) For each fixed $t$ (indeed uniformly on compact $t$–intervals), $\displaystyle\lim_{\sigma\to+\infty}\mathcal J(\sigma+it)=0$; hence $\displaystyle\lim_{\sigma\to+\infty}\Theta(\sigma+it)=-1$.
\item[(N2)] (\emph{Non--cancellation at $\xi$--zeros}) For every $\rho\in\Omega$ with $\xi(\rho)=0$,
\[
\det\nolimits_2(I-A(\rho))\neq0\quad\text{and}\quad \mathcal O(\rho)\neq0.
\]
Thus $\mathcal J$ has a pole at $\rho$ of order $\operatorname{ord}_\rho(\xi)$.
\end{itemize}

\noindent\textbf{Boundary wedge $(\mathrm{P}^+)$.}
We assume the a.e.\ boundary inequality
\[
\Re\,F\!\left(\tfrac12+it\right)\ \ge\ 0\qquad\text{for a.e.\ }t\in\mathbb R.
\tag{P+}
\label{eq:Pplus}
\]
\medskip
\noindent\textbf{From boundary wedge to interior Schur bound (half--plane Poisson passage).}
Fix $(\,\tfrac12+\sigma+it_0\,)\in\Omega\setminus Z(\xi)$ with $\sigma>0$.
By \eqref{eq:Pplus}, the boundary trace $u(t):=\Re F(\tfrac12+it)$ satisfies $u(t)\ge 0$ for a.e.\ $t\in\mathbb R$.
The Poisson formula on the half--plane yields
\[
\Re F\big(\tfrac12+\sigma+it_0\big)
\;=\;\int_{\mathbb R} u(t)\,P_\sigma(t-t_0)\,dt\ \ge\ 0,
\]
so $\Re F\ge 0$ on $\Omega\setminus Z(\xi)$.
In particular, on any rectangle $R\Subset\Omega$ with $\xi\neq 0$ near $\overline R$, we have $\Re F\ge 0$ on $R$.
Consequently, on $R$ the identity
\[
1-|\Theta(s)|^2\;=\;\frac{4\,\Re F(s)}{|F(s)+1|^2}\ \ge\ 0
\]
implies
\[
|\Theta(s)|\ \le\ 1\qquad(s\in R).
\tag{Schur}
\label{eq:SchurBound}
\]
\noindent(Thus, prior to removability, the Schur bound holds only on $\Omega\setminus Z(\xi)$.)
\medskip
\noindent\textbf{Local pinch at a putative off--critical zero.}
\emph{We use (N2) for non--cancellation at $\xi$--zeros and (N1) for the right--edge limit $\Theta\to-1$.}
Fix $\rho\in\Omega$ with $\xi(\rho)=0$.
By (N2) the function $F$ has a pole at $\rho$, hence
\[
\Theta(s)=\frac{F(s)-1}{F(s)+1}\ \longrightarrow\ 1\qquad(s\to\rho).
\]
By \eqref{eq:SchurBound}, $\Theta$ is bounded by $1$ on $(\Omega\setminus Z(\xi))$,
so the singularity of $\Theta$ at $\rho$ is removable (Riemann's theorem), and the holomorphic extension satisfies
\[
\Theta(\rho)=1.
\]
Because $\Theta$ is holomorphic on the connected domain $\Omega\setminus(Z(\xi)\setminus\{\rho\})$
and $|\Theta|\le1$ there, the Maximum Modulus Principle forces $\Theta$ to be
a \emph{constant unimodular} function on that domain (it attains its supremum $1$ at an interior point).
By analyticity, the same constant extends throughout $\Omega\setminus Z(\xi)$.
\medskip
\begin{lemma}[Connectedness and isolation]\label{rem:connectedness}
Since $Z(\xi)\cap\Omega$ is a discrete subset (zeros are isolated), one can choose a disc $D\subset\Omega$ centered at $\rho$ containing no other zeros, and $\Omega\setminus Z(\xi)$ is (path-)connected. Hence in the argument above, $\Omega\setminus\big(Z(\xi)\setminus\{\rho\}\big)$ is connected and the Maximum Modulus Principle applies on this domain.
\end{lemma}
\noindent\textbf{Contradiction with right--edge normalization.}
By (N1), $\Theta(\sigma+it)\to-1$ as $\sigma\to+\infty$; hence the above constant must equal $-1$.
But we also have $\Theta(\rho)=1$. Contradiction.
\medskip
\noindent\textbf{Conclusion of the pinch.}
No $\rho\in\Omega$ with $\xi(\rho)=0$ can exist.
\medskip
\noindent\textbf{Connective summary (globalization/pinch).}
We combine: (i) the boundary wedge (P+) from the product certificate (Theorem~\ref{thm:psc-certificate-stage2}); (ii) Poisson transport to a Herglotz function on the interior and Cayley to a Schur transfer on $\Omega\setminus Z(\xi)$; (iii) the limit-on-rectangles theorem (Theorem~\ref{thm:limit-rect}) to pass from finite approximants to the limit on zero-free rectangles; and (iv) the local pinch at a would-be zero (using (N2)) plus the right–edge normalization (N1). The pinch forces $\Theta\equiv -1$ and $\Theta(\rho)=1$ simultaneously, a contradiction. Hence there are no off–critical zeros and RH follows.
\medskip
\noindent\textbf{Normalization at infinity (used in (N1)).}
We record explicit bounds ensuring $\Theta(\sigma+it)\to-1$ uniformly for $t$ in compact $t$-intervals as $\sigma\to+\infty$.
\begin{itemize}
\item Zeta/gamma growth: For $\sigma\ge 2$ and all $t\in\R$, $|\zeta(\sigma+it)-1|\le 2^{1-\sigma}$, hence $|\zeta(\sigma+it)|\le 1+2^{1-\sigma}$. Stirling's formula on vertical strips gives $|\pi^{-s/2}\Gamma(s/2)|\asymp (1+|t|)^{\sigma/2-1/2} e^{-\pi|t|/4}$. For each fixed $t$ (indeed uniformly on compact $t$-intervals), $|\xi(\sigma+it)|\to\infty$ as $\sigma\to\infty$.
\item Outer factor: By Lemma~\ref{lem:poisson-bmo-strip} and the Carleson/BMO bounds recorded earlier, the boundary modulus $u=\log|\dettwo/\xi|$ has uniform BMO control; thus its Poisson extension $U=\Re\log\mathcal O$ is bounded on vertical strips $\{\Re s\ge 1\}$ by a constant $C_\mathcal O$, yielding $e^{-C_\mathcal O}\le |\mathcal O(\sigma+it)|\le e^{C_\mathcal O}$ for $\sigma\ge 1$.
\item Det$_2$ limit: For $\sigma\ge 1$, $\|A(\sigma+it)\|\le 2^{-\sigma}\le \tfrac12$. By the product representation in Lemma~\ref{lem:hs-diagonal} and since $\sum_p p^{-2\sigma}\to0$ as $\sigma\to\infty$, one has $|\dettwo(I-A(\sigma+it)) - 1|\le C\sum_p p^{-2\sigma}\to 0$ (uniformly for $t$ in compact intervals).
\end{itemize}
Combining, for $\sigma\ge 2$,
\[
  \big|\mathcal J(\sigma+it)\big|\ =\ \Big|\frac{\dettwo(I-A(\sigma+it))}{\mathcal O(\sigma+it)\,\xi(\sigma+it)}\Big|\ \le\ \frac{1+ o(1)}{e^{-C_\mathcal O}\,|\xi(\sigma+it)|}\ \xrightarrow[\sigma\to\infty]{}\ 0
\]
uniformly for $t$ in compact intervals. Hence $\Theta(\sigma+it)=(2\mathcal J-1)/(2\mathcal J+1)\to -1$ uniformly for $t$ in compact intervals.

\bigskip
\begin{theorem}[Riemann Hypothesis]\label{thm:RH}
Under \textnormal{(P+)} and \textnormal{(N1)}--\textnormal{(N2)}, one has $\xi(s)\neq 0$ for all $s\in\Omega$. Hence all nontrivial zeros of $\zeta$ lie on $\Re s=\tfrac12$.
\end{theorem}
\begin{proof}
By Theorem~\ref{thm:psc-certificate-stage2} we have (P+). By Theorem~\ref{thm:globalize-main} (Schur globalization), there are no off--critical zeros in $\Omega$. The functional equation and symmetry then force all nontrivial zeros onto $\Re s=\tfrac12$.
\end{proof}

\begin{lemma}[Carleson box energy: stable sum bound]\label{lem:carleson-sum}
For harmonic potentials $U_1,U_2$ on $\Omega$, one has
\[
\sqrt{C_{\mathrm{box}}(U_1+U_2)}\ \le\
\sqrt{C_{\mathrm{box}}(U_1)}\ +\ \sqrt{C_{\mathrm{box}}(U_2)}.
\]
\end{lemma}

\begin{corollary}[All-interval Carleson energy for $U_\xi$]\label{cor:xi-carleson-all-I}
For every interval $I\subset\R$ one has
\[
  \iint_{Q(I)} |\nabla U_{\xi}(\sigma,t)|^2\,\sigma\,dt\,d\sigma\ \le\ C_\xi^{\!*}\,|I|,
\]
with a finite constant $C_\xi^{\!*}$ depending only on the parameters in Lemma~\ref{lem:carleson-xi} and on the fixed aperture. In particular, the bound of Lemma~\ref{lem:carleson-xi} extends from Whitney intervals to arbitrary intervals.
\end{corollary}
\begin{proof}
Cover $Q(I)$ by a finite-overlap tiling with boxes $Q(\alpha I_j)$ whose bases $I_j$ form a Whitney-type partition of $I$ (length $|I_j|\asymp c/\log\langle T_j\rangle)$, and vertically stack at most $\lceil |I|/|I_j|\rceil$ layers of height $\asymp |I_j|$ to reach the full height of $Q(I)$. Apply Lemma~\ref{lem:carleson-xi} on each tile and sum; bounded overlap yields the stated $\lesssim |I|$ bound.
\end{proof}

\begin{lemma}[L$^1$-tested control for $\partial_\sigma\Re\log\xi$]\label{lem:xi-deriv-L1}
For each compact $I\Subset\R$ there exists $C'_I<\infty$ such that for all $0<\sigma\le\varepsilon_0$ and all $\phi\in C_c^2(I)$,
\[
\Big|\int_I \phi(t)\,\partial_\sigma\Re\log\xi\!\big(\tfrac12+\sigma+it\big)\,dt\Big|
\ \le\ C'_I\,\|\phi\|_{H^1(I)}.
\]
\end{lemma}

\begin{proof}[Proof of Lemma~\ref{lem:carleson-sum}]
Write $\mu_j:=|\nabla U_j|^2\,\sigma\,dt\,d\sigma$ and $\mu_{12}:=|\nabla(U_1{+}U_2)|^2\,\sigma\,dt\,d\sigma$. For any Carleson box $B$, by Cauchy–Schwarz,
\[
\int_{B} |\nabla(U_1+U_2)|^2\,\sigma\,dt\,d\sigma
\ \le\ \Big(\sqrt{\int_B |\nabla U_1|^2\,\sigma}\ +\ \sqrt{\int_B |\nabla U_2|^2\,\sigma}\Big)^{\!2}.
\]
Taking supremum over Carleson boxes $B$ and dividing by $|I_B|$ yields
\[
 \sqrt{C_{\mathrm{box}}(U_1{+}U_2)}\ \le\ \sqrt{C_{\mathrm{box}}(U_1)}\ +\ \sqrt{C_{\mathrm{box}}(U_2)}.
\]
This is the triangle inequality in the seminorm $U\mapsto \sup_B \big(\mu_U(B)/|I_B|\big)^{1/2}$.
\end{proof}

\begin{proof}[Proof of Lemma~\ref{lem:xi-deriv-L1}]
Let $I\Subset\R$ and $\phi\in C_c^2(I)$. Let $V$ be the Poisson extension of $\phi$ on a fixed dilation $Q(\alpha I)$. Green's identity together with Cauchy–Riemann for $U_\xi=\Re\log\xi$ gives
\[
  \int_I \phi(t)\,\partial_\sigma\Re\log\xi\!\big(\tfrac12+\sigma+it\big)\,dt
  \,=\, \iint_{Q(\alpha I)} \nabla U_\xi\cdot\nabla V\,dt\,d\sigma.
\]
By Cauchy–Schwarz and the scale–invariant bound $\|\nabla V\|_{L^2(\sigma;Q(\alpha I))}\lesssim \|\phi\|_{H^1(I)}$, we get
\[
  \Big|\int_I \phi\,\partial_\sigma\Re\log\xi\Big|
  \,\le\ \Big(\iint_{Q(\alpha I)}|\nabla U_\xi|^2\,\sigma\Big)^{\!1/2}\,C_I\,\|\phi\|_{H^1(I)}.
\]
By Lemma~\ref{lem:carleson-xi} and Corollary~\ref{cor:xi-carleson-all-I}, $\iint_{Q(\alpha I)}|\nabla U_\xi|^2\,\sigma\le C_\xi^{\!*}\,|I|$, so the right–hand side is $\le C'_I\,\|\phi\|_{H^1(I)}$ with $C'_I$ depending only on $I$. This proves the claim.
\end{proof}
\begin{corollary}[Conservative numeric closure under Lemma~\ref{lem:carleson-sum}]\label{cor:conservative-closure}
With the constants \(c_0(\psi)=0.17620819\), \(C_\psi^{(H^1)}=0.2400\), \(C_H(\psi)\le 2/\pi\), \(K_0=0.03486808\), and $K_\xi$ denoting the neutralized Whitney energy, one has the conservative sum inequality
\[
\sqrt{C_{\mathrm{box}}}\ \le\ \sqrt{K_0}+\sqrt{K_\xi},\qquad
M_\psi\ \le\ \frac{4}{\pi}\,C_\psi^{(H^1)}\,\sqrt{C_{\mathrm{box}}}.
\]
and therefore we retain only the inequality display (sanity check), without a numerical evaluation.
These numbers provide quantitative diagnostics. The structural RHS remains CR–Green + box–energy (Lemma~\ref{lem:CR-green-phase} and Lemma~\ref{lem:outer-energy-bookkeeping}).

\medskip
\ifshownumerics
\noindent\textbf{Diagnostics (sanity check only).} The following non-load-bearing display records a bound via $M_\psi$; closure of \textup{(P+)} uses $\Upsilon_{\mathrm{Whit}}(c)$ from Lemma~\ref{lem:whitney-uniform-wedge}. Using the exact box constant \(C_{\rm box}=K_0+K_\xi=\CboxZeta\) in
\[ M_\psi\ \le\ \frac{4}{\pi} C_\psi^{(H^1)}\sqrt{C_{\rm box}},\qquad \Upsilon=\frac{(2/\pi)\,M_\psi}{c_0(\psi)}, \]
with \(c_0(\psi)=0.17620819\), \(C_\psi^{(H^1)}=\CpsiHone\), we obtain
\[ M_\psi\ \le\ \Mpsilocked,\qquad \Upsilon_{\mathrm{diag}}\ :=\ \frac{(2/\pi)\,M_\psi}{c_0(\psi)}\ =\ \UpsilonLocked. \]
All inputs are unconditional and independently enclosed.
\fi
\end{corollary}
% \fi
\medskip
\noindent\textbf{Proof of (N2) (non--cancellation at $\xi$--zeros).}
For $s=\sigma+it$ with $\sigma>\tfrac12$, define the diagonal operator $A(s)e_p=p^{-s}e_p$ on $\ell^2(\mathbb P)$. Then $\|A(s)\|=2^{-\sigma}<1$ and $\|A(s)\|_{\mathrm{HS}}^2=\sum_{p}p^{-2\sigma}<\infty$, so $A(s)$ is Hilbert--Schmidt. The 2--modified determinant for diagonal $A(s)$ is
\[
\det\nolimits_2\!\big(I-A(s)\big)\;=\;\prod_{p\in\mathbb P}(1-p^{-s})\,e^{p^{-s}},
\]
which converges absolutely and is nonzero because each factor is nonzero. Moreover, $I-A(s)$ is invertible with $\|(I-A(s))^{-1}\|\le (1-2^{-\sigma})^{-1}$ since $|1-p^{-s}|\ge 1-2^{-\sigma}>0$. Finally, the outer normalizer has the form $\mathcal O(s)=\exp H(s)$ with $H$ analytic on $\Omega$, hence $\mathcal O$ is zero--free on $\Omega$. Thus if $\rho\in\Omega$ with $\xi(\rho)=0$, then $\det_2(I-A(\rho))\neq0$ and $\mathcal O(\rho)\neq0$, i.e. no cancellation can occur at $\rho$. Local-uniform analyticity on $\Omega$ follows from HS$\to\dettwo$ continuity (Proposition~\ref{prop:hs-det2-continuity}).
\begin{lemma}[Diagonal HS determinant is analytic and nonzero]\label{lem:hs-diagonal}
For $s=\sigma+it$ with $\sigma>\tfrac12$, the diagonal operator $A(s)e_p=p^{-s}e_p$ satisfies
\[
\sup_{p}|p^{-s}|=2^{-\sigma}<1,\qquad \sum_{p}|p^{-s}|^2=\sum_{p}p^{-2\sigma}<\infty.
\]
Hence $A(s)\in\mathrm{HS}$, $I-A(s)$ is invertible, and
\[
\det\nolimits_2\big(I-A(s)\big)=\prod_{p}(1-p^{-s})\,e^{p^{-s}}
\]
is analytic and nonzero on $\{\Re s>\tfrac12\}$.
\end{lemma}
\begin{proof}
Immediate from the displayed bounds; invertibility follows since $|1-p^{-s}|\ge 1-2^{-\sigma}>0$, and the product defining $\det_2$ converges absolutely with nonzero factors.
\end{proof}
\paragraph{Normalization and finite port (eliminating $C_P$ and $C_\Gamma$).}
We record the implementation details that ensure the product certificate contains no prime budget and no Archimedean term.

\begin{lemma}[\(\zeta\)–normalized outer and compensator]\label{lem:zeta-normalization}
Define the outer $\mathcal O_\zeta$ on $\Omega$ with boundary modulus $\big|\dettwo(I-A)/\zeta\big|$ and set
\[ J_\zeta(s)\ :=\ \frac{\dettwo(I-A(s))}{\mathcal O_\zeta(s)\,\zeta(s)}\cdot B(s),\qquad B(s):=\frac{s-1}{s}. \]
On $\Re s=\tfrac12$ one has $|B|=1$. The phase–velocity identity of Theorem~\ref{thm:phase-velocity-quant} holds for $J_\zeta$ with the same Poisson/zero right-hand side. In particular, no separate Archimedean term enters the inequality used by the certificate.
\end{lemma}

\begin{proof}
Set $X:=\xi$ and $Z:=\zeta$, and let $G$ denote the archimedean factor linking them,
\[
  X(s)\;=\;\tfrac12 s(1{-}s)\,\pi^{-s/2}\,\Gamma(\tfrac s2)\,Z(s)\;=:\;G(s)\,Z(s).
\]
Define $\mathcal O_X$ (resp. $\mathcal O_Z$) to be the outer on $\Omega$ with boundary modulus $\big|\dettwo(I{-}A)/X\big|$ (resp. $\big|\dettwo(I{-}A)/Z\big|$). Then, by construction,
\[
  \Big|\frac{\dettwo(I{-}A)}{\mathcal O_X\,X}\Big|\equiv 1\equiv \Big|\frac{\dettwo(I{-}A)}{\mathcal O_Z\,Z}\Big|\quad \text{a.e. on }\{\Re s=\tfrac12\}.
\]
Consequently the phase–velocity identity (Theorem~\ref{thm:phase-velocity-quant}) applies to either unimodular ratio. Writing
\[
  \log \frac{\dettwo(I{-}A)}{\mathcal O_X\,X}
  \;=\; \log \frac{\dettwo(I{-}A)}{\mathcal O_Z\,Z}\; -\; \log\frac{\mathcal O_X}{\mathcal O_Z}\; -\; \log G,
\]
and differentiating in $\sigma$ on the boundary, the two outer terms contribute zero to the boundary phase derivative (by unimodularity and the outer/Poisson representation). The remaining difference is $-\partial_\sigma\Im\log G$.

On $\Re s=\tfrac12$ we have $|O_X/O_Z|=|Z/X|=|1/G|$, so by Lemma~\ref{lem:outer-phase-HT}
\[
  \partial_\sigma\Im\log\!\left(\frac{O_X}{O_Z}\right)(\tfrac12+it)\;=\;-\partial_\sigma\Im\log G(\tfrac12+it)
\]
in $\mathcal D'(\mathbb R)$. Compensating the simple zero at $s=1$ by the half–plane Blaschke factor
\[
  B(s)\;=\;\frac{s-1}{s}\qquad(|B|\equiv 1\text{ on }\Re s=\tfrac12)
\]
accounts for the inner contribution at $s=1$. Therefore, on the boundary,
\[
  \partial_\sigma\Im\log\!\Big(\frac{\dettwo(I{-}A)}{\mathcal O_Z\,Z}\cdot B\Big)
  \,=\, \partial_\sigma\Im\log\frac{\dettwo(I{-}A)}{\mathcal O_X\,X},
\]
and the quantitative phase–velocity identity holds in the same form for $J_\zeta=(\dettwo/(\mathcal O_\zeta\,\zeta))\,B$ as for $\mathcal J=\dettwo/(\mathcal O\,\xi)$. In particular, no Archimedean term enters the certificate.
\end{proof}

% (archived) A standalone prime-layer outer O_p is not used in the main chain; the ζ-normalized gauge and windowed identities suffice, and no C_P term enters the certificate.

\begin{corollary}[No $C_P$/$C_\Gamma$ in the certificate]
With $J_\zeta$ and $\widehat J$ as above, the active CR–Green route uses $c_0(\psi)$ and the CR–Green constant $C(\psi)$ together with the box–energy constant $C_{\rm box}^{(\zeta)}$. In particular, $C_P=0$ and $C_\Gamma=0$ on the RHS; $C_H(\psi)$ and $M_\psi$ are retained only as auxiliary/readability bounds.
\end{corollary}

\noindent\emph{Active route.} Throughout we use the $\zeta$-normalized boundary gauge with the Blaschke compensator; the product certificate uses $c_0(\psi)$ and the CR–Green constant $C(\psi)$ together with $C_{\rm box}^{(\zeta)}$ (no $C_P$, no $C_\Gamma$). From these inputs we lock a smallness $\Upsilon<\tfrac12$, and \textup{(P+)} follows by the quantitative wedge lemma (Lemma~\ref{lem:whitney-uniform-wedge}).

% (Removed alternative interior-pole lemma to keep a single contradiction path in the pinch.)

\begin{lemma}[Derivative envelope for the printed window]\label{lem:CH-derivative-explicit}
Let $\psi$ be the even $C^\infty$ flat--top window from the "Printed window" paragraph (equal to $1$ on $[-1,1]$, supported in $[-2,2]$, with monotone ramps on $[-2,-1]$ and $[1,2]$), and $\varphi_L(t):=L^{-1}\psi((t-T)/L)$. Then, for every $L>0$,
\[
  \big\|\big(\mathcal H[\varphi_L]\big)'\big\|_{L^\infty(\mathbb R)} \;\le\; \frac{C_H(\psi)}{L}
  \qquad\text{with}\qquad C_H(\psi)\;\le\;\frac{2}{\pi}\;<\;0.65.
\]
\begin{proof}
\textit{Step 1 (Scaling).} By the standard scale/translation identity (recorded in the manuscript),
\[
  \mathcal H[\varphi_L](t)=H_\psi\!\Big(\frac{t-T}{L}\Big),\qquad
  H_\psi(x):=\frac{1}{\pi}\,\mathrm{p.v.}\!\int_{\mathbb R}\frac{\psi(y)}{x-y}\,dy,
\]
we get
\[
  \big(\mathcal H[\varphi_L]\big)'(t)=\frac{1}{L}\,H_\psi'\!\Big(\frac{t-T}{L}\Big)
  \quad\Longrightarrow\quad
  \big\|\big(\mathcal H[\varphi_L]\big)'\big\|_\infty=\frac{1}{L}\,\|H_\psi'\|_\infty.
\]
Thus it suffices to bound $\|H_\psi'\|_\infty$.

\smallskip
\textit{Step 2 (Structure and signs).} Since $\psi'\equiv0$ on $(-1,1)$ and the ramps are monotone,
\[
  \psi'(y)\ge0\ \text{on }[-2,-1],\qquad \psi'(y)\le0\ \text{on }[1,2],\qquad
  \int_{-2}^{-1}\psi'(y)\,dy=1=\!-\!\int_{1}^{2}\psi'(y)\,dy.
\]
In distributions, $(H_\psi)'= \mathcal H[\psi']$, so for every $x\in\mathbb R$
\[
  H_\psi'(x)=\frac{1}{\pi}\,\mathrm{p.v.}\!\int_{-2}^{-1}\frac{\psi'(y)}{x-y}\,dy\;+\;
             \frac{1}{\pi}\,\mathrm{p.v.}\!\int_{1}^{2}\frac{\psi'(y)}{x-y}\,dy.
\]

\smallskip
\textit{Step 3 (Worst case occurs between the ramps).} Fix $x\in(-1,1)$.  On $y\in[-2,-1]$ the kernel $y\mapsto 1/(x-y)$ is positive and strictly increasing; on $y\in[1,2]$ the kernel is negative and strictly decreasing.  Since the ramp densities are monotone and have unit mass in absolute value, the rearrangement/endpoint principle (maximize a monotone–kernel integral by concentrating mass at an endpoint) gives the pointwise bound
\[
  \Big|\mathrm{p.v.}\!\int_{-2}^{-1}\frac{\psi'(y)}{x-y}\,dy\Big|
  \le \frac{1}{1+x},\qquad
  \Big|\mathrm{p.v.}\!\int_{1}^{2}\frac{\psi'(y)}{x-y}\,dy\Big|
  \le \frac{1}{1-x}.
\]
Therefore, for every $x\in(-1,1)$,
\[
  |H_\psi'(x)| \;\le\; \frac{1}{\pi}\Big(\frac{1}{1+x}+\frac{1}{1-x}\Big)
  \;\le\; \frac{2}{\pi}\,\frac{1}{1-x^2}
  \;\le\; \frac{2}{\pi},
\]
with the maximum at $x=0$.
\smallskip
\textit{Step 4 (Outside the plateau).} For $x\notin[-1,1]$ the two ramp contributions have opposite signs but larger denominators, hence smaller magnitude. More precisely, for $x>1$, the left–ramp integral is a principal value on $[-2,-1]$ against a $C^\infty$ density that vanishes at the endpoints; the standard $C^1$–vanishing at $y=-2,-1$ eliminates the endpoint singularity and keeps the PV finite and strictly smaller than its in–plateau counterpart (a short integration–by–parts argument on the left interval makes this explicit). By evenness, the same holds for $x<-1$.  Consequently,
\[
  \sup_{x\in\mathbb R}|H_\psi'(x)|=\sup_{x\in(-1,1)}|H_\psi'(x)|\;\le\;\frac{2}{\pi}.
\]
Putting Steps 1–4 together,
\[
  \big\|\big(\mathcal H[\varphi_L]\big)'\big\|_\infty
  \;=\;\frac{1}{L}\,\|H_\psi'\|_\infty
  \;\le\;\frac{1}{L}\cdot\frac{2}{\pi}.
\]
Hence we can take $C_H(\psi)\le 2/\pi < 0.65$.
\end{proof}

\end{lemma}

% removed stray fi
% =========================================================

\paragraph*{Certificate \textemdash{} weighted \(p\)-adaptive model at \(\sigma_0=0.6\).}
Fix \(\sigma_0=0.6\), take \(Q=29\) and \(p_{\min}=\mathrm{nextprime}(Q)=31\).\\
Use the \(p\)-adaptive weighted off-diagonal enclosure (for all \(p\neq q\), uniformly in \(\sigma\in[\sigma_0,1]\)):
\[
\|H_{pq}(\sigma)\|_2 \;\le\; \frac{C_{\mathrm{win}}}{4}\, p^{-(\sigma+\tfrac12)}\, q^{-(\sigma+\tfrac12)},
\qquad C_{\mathrm{win}}=0.25.
\]

\noindent\emph{Prime sums (small block \(p\le Q\)).} With \(\sigma_0=0.6\),
\[
S_{\sigma_0}(Q)\;=\;\sum_{p\le Q} p^{-\sigma_0}\;=\;2.9593220929,\qquad
S_{\sigma_0+\tfrac12}(Q)\;=\;\sum_{p\le Q} p^{-(\sigma_0+\tfrac12)}\;=\;1.3239981250.
\]

% alt-route Bridges/KYP removed from main body
% removed optional Bridges A--C discussion and references
\noindent\emph{In-block Gershgorin lower bounds (uniform on \([\sigma_0,1]\)).}
Define
\[
L(p)\;:=\;(1-\sigma_0)\,(\log p)\,p^{-\sigma_0},\qquad 
\mu_p^{\mathrm L}\;\ge\;1-\frac{L(p)}{6}.
\]
At \(p_{\min}=31\) this gives
\[
L(31)=0.1750014502,\qquad 
\mu_{\min}^{\mathrm{far}}\;:=\;1-\frac{L(31)}{6}\;=\;0.9708330916.
\]
Over the small block \(p\le Q\) the worst case is at \(p=5\):
\[
L(5)=0.2451050257,\qquad 
\mu_{\min}^{\mathrm{small}}\;:=\;1-\frac{L(5)}{6}\;=\;0.9591491624.
\]
\noindent\emph{Off-diagonal budgets (all rigorous).}
Let \(\sigma^\star:=\sigma_0+\tfrac12=1.1\).\\
With the integer-tail majorant \(\displaystyle \sum_{n\ge p_{\min}-1} n^{-\sigma^\star}\le
\frac{(p_{\min}-1)^{1-\sigma^\star}}{\sigma^\star-1}\)
we obtain:
\[
\Delta_{\mathrm{FS}}
=\frac{C_{\mathrm{win}}}{4}\,p_{\min}^{-\sigma^\star}\,S_{\sigma^\star}(Q)
=0.0018935184,
\]
\[
\Delta_{\mathrm{FF}}
=\frac{C_{\mathrm{win}}}{4}\,p_{\min}^{-\sigma^\star}\!
\sum_{n\ge p_{\min}-1}\! n^{-\sigma^\star}
\;\le\;\frac{C_{\mathrm{win}}}{4}\,p_{\min}^{-\sigma^\star}\,
\frac{(p_{\min}-1)^{1-\sigma^\star}}{\sigma^\star-1}
=0.0101781777,
\]
\[
\Delta_{\mathrm{SS}}
=\frac{C_{\mathrm{win}}}{4}\,2^{-\sigma^\star}
\!\sum_{\substack{p\le Q\\ p\neq 2}}\! p^{-\sigma^\star}
=0.0250018328,
\]
\[
\Delta_{\mathrm{SF}}
=\frac{C_{\mathrm{win}}}{4}\,2^{-\sigma^\star}\!
\sum_{n\ge p_{\min}-1}\! n^{-\sigma^\star}
\;\le\;\frac{C_{\mathrm{win}}}{4}\,2^{-\sigma^\star}\,
\frac{(p_{\min}-1)^{1-\sigma^\star}}{\sigma^\star-1}
=0.2075080249.
\]
\noindent\emph{Certified finite-block spectral gap.}
Combining the in-block lower bounds with the off-diagonal budgets yields
\[
\delta_{\mathrm{cert}}(\sigma_0)\;\ge\;
\min\Big\{
\underbrace{\mu_{\min}^{\mathrm{small}}-(\Delta_{\mathrm{SS}}+\Delta_{\mathrm{SF}})}_{\text{small-block rows}}\,,\;
\underbrace{\mu_{\min}^{\mathrm{far}}-(\Delta_{\mathrm{FS}}+\Delta_{\mathrm{FF}})}_{\text{far-block rows}}\
\Big\}
=0.7266393047\;>\;0.
\]
Hence the normalized finite block is uniformly positive definite on \([\sigma_0,1]\).
\begin{corollary}[Boundary-uniform smoothed control]\label{cor:det2-boundary}
Let $I\Subset\R$, $\varepsilon_0\in(0,\tfrac12]$, and $\varphi\in C_c^2(I)$. Then, uniformly for $\sigma\in(\tfrac12,\tfrac12+\varepsilon_0]$,
\[
  \Big|\int_{\R} \varphi(t)\,\partial_\sigma\,\Re\log\dettwo\big(I-A(\sigma+it)\big)\,dt\Big|\ \le\ C_*\,\|\varphi''\|_{L^1(I)}.
\]
In particular, the bound remains valid in the boundary limit $\sigma\downarrow \tfrac12$ in the sense of distributions.
\end{corollary}
\subsection*{Smoothed Cauchy and outer limit (A2)}
% Lead-in: We build outers from boundary data u_ε and pass to a locally-uniform outer limit to normalize the boundary modulus.
\begin{proposition}[Outer normalization: existence, boundary a.e. modulus, and limit]\label{prop:outer-central}
There exist outer functions \(\mathcal O_\varepsilon\) on \(\{\Re s>\tfrac12+\varepsilon\}\) with a.e. boundary modulus \(|\mathcal O_\varepsilon(\tfrac12+\varepsilon+it)|=\exp u_\varepsilon(t)|\), and \(\mathcal O_\varepsilon\to\mathcal O\) locally uniformly on \(\Omega\) as \(\varepsilon\downarrow 0\), where \(\mathcal O\) has boundary modulus \(\exp u(t)\). (Standard Poisson–outer representation; see, e.g., \cite{Garnett,RosenblumRovnyak}.) Consequently the outer-normalized ratio \(\mathcal J=\dettwo(I-A)/(\mathcal O\,\xi)\) has a.e. boundary values on \(\Re s=\tfrac12\) with \(|\mathcal J(\tfrac12+it)|=1\).
\end{proposition}
\begin{proof}
For each $\varepsilon\in(0,\tfrac12]$, set $u_\varepsilon(t):=\log\Big|\dettwo\!\big(I\! -\!A(\tfrac12\!+\!\varepsilon\!+\!it)\big)\Big|\! -\!\log\big|\xi(\tfrac12\!+\!\varepsilon\!+\!it)\big|$. For each compact $I\Subset\R$ and each $\varphi\in C_c^2(I)$ there exists $C(\varphi)<\infty$ such that, uniformly for $\varepsilon,\delta\in(0,\varepsilon_0]$,
\[\Big|\int_{\R} \varphi(t)\,\big(u_\varepsilon(t)-u_\delta(t)\big)\,dt\Big|\ \le\ C(\varphi)\,|\varepsilon-\delta|.\]
Consequently, the outer normalizations $\mathcal O_\varepsilon$ converge locally uniformly to an outer limit $\mathcal O$ on $\Omega$.
\end{proof}
\begin{proof}
Fix $I\Subset\R$ and $\varphi\in C_c^2(I)$. For $0<\delta<\varepsilon\le\varepsilon_0$,
\[\int \varphi\,\big(u_\varepsilon-u_\delta\big)\,dt = \int_\delta^{\varepsilon}\!\int \varphi(t)\,\partial_\sigma\,\Re\Big(\log\det_2(I-A)-\log\xi\Big)\big(\tfrac12+\sigma+it\big)\,dt\,d\sigma.\]
By Lemma~\ref{lem:det2-unsmoothed}, $\big|\int \varphi\,\partial_\sigma\Re\log\det_2\big|\le C_*\,\|\varphi''\|_{L^1(I)}$. For $\partial_\sigma\Re\log\xi=\Re(\xi'/\xi)$, test against $\varphi$ via the Poisson extension on a fixed dilation $Q(\alpha I)$ and use Lemma~\ref{lem:carleson-xi}:
\[\Big|\int \varphi\,\Re(\xi'/\xi)\Big|\ \lesssim\ \Big(\iint_{Q(\alpha I)} |\nabla U_\xi|^2\,\sigma\Big)^{1/2}\,\|\varphi\|_{H^1(I)}\ \lesssim\ |I|^{1/2}\,\|\varphi\|_{H^1(I)}.\]
Therefore $\big|\int \varphi\,(u_\varepsilon-u_\delta)\big|\le C(\varphi)\,|\varepsilon-\delta|$, proving the Lipschitz bound. Local-uniform convergence of outers follows from the Poisson representation and dominated convergence on $\{\Re s\ge\tfrac12+\eta\}$.
\end{proof}
\subsection*{Carleson energy and boundary BMO (unconditional)}
We record a direct Carleson–energy route to boundary BMO for the limit $u(t)=\lim_{\varepsilon\downarrow 0}u_\varepsilon(t)$.

\begin{lemma}[Arithmetic Carleson energy]\label{lem:carleson-arith}
Let
\[
 U_{\det_2}(\sigma,t)\ :=\ \sum_{p}\sum_{k\ge 2}\frac{(\log p)\,p^{-k/2}}{k\log p}\,e^{-k\log p\,\sigma}\,\cos\big(k\log p\,t\big),\qquad \sigma>0.
\]
Then for every interval $I\subset\R$ with Carleson box $Q(I):=I\times(0,|I|]$
\[
 \iint_{Q(I)} |\nabla U_{\det_2}|^2\,\sigma\,dt\,d\sigma\ \le\ \frac{|I|}{4}\,\sum_{p}\sum_{k\ge 2}\frac{p^{-k}}{k^2}
 \ =:\ K_0\,|I|,\qquad K_0:=\frac{1}{4}\sum_{p}\sum_{k\ge 2}\frac{p^{-k}}{k^2}<\infty.
\]
\end{lemma}
\begin{proof}
For a single mode $b\,e^{-\omega\sigma}\cos(\omega t)$ one has $|\nabla|^2=b^2\omega^2e^{-2\omega\sigma}$, hence
\[
 \int_0^{|I|}\!\int_I |\nabla|^2\,\sigma\,dt\,d\sigma\ \le\ |I|\cdot\sup_{\omega>0}\int_0^{|I|}\sigma\,\omega^2e^{-2\omega\sigma}d\sigma\cdot b^2\ \le\ \tfrac14\,|I|\,b^2.
\]
With $b=(\log p)\,p^{-k/2}/(k\log p)$ and $\omega=k\log p$, summing over $(p,k)$ gives the claim and the finiteness of $K_0$.
\end{proof}
\paragraph{Whitney scale and short–interval zeros.}
Throughout we use the Whitney schedule clipped at $L_\star$:
\[
  L\ =\ L(T)\ :=\ \frac{c}{\log\langle T\rangle}\ \le\ \frac{1}{\log\langle T\rangle},\qquad \langle T\rangle:=\sqrt{1+T^2},\
\]
for a fixed absolute $c\in(0,1]$; all boxes are $Q(\alpha I)$ with a uniform $\alpha\in[1,2]$.
We work on Whitney boxes $Q(I)$ with
\[
  L=L(T):=\min\Big\{\frac{c}{\log\langle T\rangle},\ L_\star\Big\},\qquad \langle T\rangle:=\sqrt{1+T^2},\quad c>0\ \text{fixed}.
\]
There exist absolute $A_0,A_1>0$ such that for $T\ge2$ and $0<H\le1$,
\[
  N(T;H)\ :=\ \#\{\rho=\beta+i\gamma:\ \gamma\in[T,T+H]\}\ \le\ A_0\ +\ A_1\,H\log\langle T\rangle.
\]
\begin{lemma}[Annular Poisson–balayage $L^2$ bound]\label{lem:annular-balayage}
Let $I=[T-L,T+L]$, $Q_\alpha(I)=I\times(0,\alpha L]$, and fix $k\ge1$. For
$\mathcal A_k:=\{\rho=\beta+i\gamma:\ 2^kL<|T-\gamma|\le 2^{k+1}L\}$ set
\[
  V_k(\sigma,t):=\sum_{\rho\in\mathcal A_k}\frac{\sigma}{(t-\gamma)^2+\sigma^2}.
\]
Then
\[
  \iint_{Q_\alpha(I)} V_k(\sigma,t)^2\,\sigma\,dt\,d\sigma\ \ll_\alpha\ |I|\,4^{-k}\,\nu_k,
\]
where $\nu_k:=\#\mathcal A_k$, and the implicit constant depends only on $\alpha$.
\end{lemma}
\begin{proof}
Write $K_\sigma(x):=\sigma/(x^2+\sigma^2)$ and $V_k=\sum_{\rho\in\mathcal A_k}K_\sigma(\cdot-\gamma)$. For any finite index set $\mathcal J$,
\[
  V_k^2\;\le\; \sum_{j\in\mathcal J} K_\sigma(\cdot-\gamma_j)^2\ +\ 2\!\!\sum_{i<j} K_\sigma(\cdot-\gamma_i)K_\sigma(\cdot-\gamma_j).
\]
Integrate over $t\in I$ first. For the diagonal terms, using $|t-\gamma|\ge 2^kL-L\ge 2^{k-1}L$ for $t\in I$ and $k\ge 1$,
\[
 \int_I K_\sigma(t-\gamma)^2\,dt\ =\ \sigma^2\!\int_I \frac{dt}{\big((t-\gamma)^2+\sigma^2\big)^2}\ \le\ \frac{L}{(2^{k-1}L)^2}\,\sigma\ \le\ \frac{\sigma}{4^{k-1}L}.
\]
Multiplying by the area weight $\sigma$ and integrating $\sigma\in(0,\alpha L]$ gives
\[
 \int_0^{\alpha L}\!\!\left(\int_I K_\sigma(t-\gamma)^2\,dt\right)\sigma\,d\sigma\ \le\ \frac{1}{4^{k-1}L}\int_0^{\alpha L}\!\sigma^2 d\sigma\ =\ \frac{\alpha^3 L^2}{3\cdot 4^{k-1}}\ \le\ \frac{C_{\mathrm{diag}}(\alpha)}{4^{k}}\,|I|,
\]
with $C_{\mathrm{diag}}(\alpha):=\tfrac{4\alpha^3}{3}\cdot\tfrac{L}{|I|}\asymp_\alpha 1$. Summing over $\nu_k$ choices of $\gamma$ contributes a factor $\nu_k$.

For the off-diagonal terms, for $i\ne j$ one has on $I$ that $K_\sigma(t-\gamma_j)\le \sigma/(2^{k-1}L)^2$. Hence
\[
 \int_I K_\sigma(t-\gamma_i)K_\sigma(t-\gamma_j)\,dt\ \le\ \frac{\sigma}{(2^{k-1}L)^2}\int_\R K_\sigma(t-\gamma_i)\,dt\ =\ \frac{\pi\sigma}{(2^{k-1}L)^2},
\]
and integrating $\sigma\in(0,\alpha L]$ with the extra factor $\sigma$ yields $\le C'_{\mathrm{off}}(\alpha)\,L\cdot 4^{-k}$. Summing in $i,j$ via the Schur test with $f_j(t):=K_\sigma(t-\gamma_j)\mathbf 1_I(t)$ gives
\[
 \int_I V_k(\sigma,t)^2\,dt\ \le\ C''(\alpha)\,\nu_k\,\frac{\sigma}{(2^kL)^2}.
\]
Integrating $\sigma\in(0,\alpha L]$ with weight $\sigma$ gives $\le C_{\mathrm{off}}(\alpha)\,|I|\cdot 4^{-k}\,\nu_k$. Combining diagonal and off–diagonal parts, absorbing harmless constants into $C_\alpha$, we obtain the stated bound with an explicit $C_\alpha=O(\alpha^3)$.
\end{proof}

\begin{lemma}[Analytic ($\xi$) Carleson energy on Whitney boxes]\label{lem:carleson-xi}
\emph{Reference.} The local zero count used below follows from the Riemann–von Mangoldt formula; see, e.g., Titchmarsh (Thm.~9.3) or Ivi\'c (Ch.~8). A Vinogradov–Korobov zero-density refinement yields the stated strip bounds with explicit exponents (unconditional).
There exist absolute constants $c\in(0,1]$ and $C_\xi<\infty$ such that for every interval $I=[T-L,\,T+L]$ with Whitney scale $L:=c/\log\langle T\rangle$, the Poisson extension
\[
 U_{\xi}(\sigma,t):=\Re\log\xi\big(\tfrac12+\sigma+it\big),\qquad (\sigma>0),
\]
\paragraph{Whitney scale and neutralization.}
Throughout this lemma we take the base interval $I=[T-L,T+L]$ with
\[
  L=L(T):=\frac{c}{\log\langle T\rangle},\qquad \langle T\rangle:=\sqrt{1+T^2},\quad c>0\ \text{fixed}.
\]
obeys the Carleson bound
\[ \iint_{Q(I)} |\nabla U_{\xi}(\sigma,t)|^2\,\sigma\,dt\,d\sigma\ \le\ C_\xi\,|I|. \]
\end{lemma}

\begin{proof}
All inputs are unconditional. Fix $I=[T-L,T+L]$ with $L=c/\log\langle T\rangle$ and aperture $\alpha\in[1,2]$. Neutralize near zeros by a local half-plane Blaschke product $B_I$ removing zeros of $\xi$ inside a fixed dilate $Q(\alpha'I)$ ($\alpha'>\alpha$). This yields a harmonic field $\widetilde U_\xi$ on $Q(\alpha I)$ and
\[
  \iint_{Q(\alpha I)} |\nabla U_\xi|^2\,\sigma\,dt\,d\sigma\ \asymp\ \iint_{Q(\alpha I)} |\nabla \widetilde U_\xi|^2\,\sigma\,dt\,d\sigma\ +\ O_\alpha(|I|),
\]
so it suffices to bound the neutralized energy.

Write $\partial_\sigma U_\xi=\Re\,(\xi'/\xi)=\Re\sum_\rho (s-\rho)^{-1}+A$, where $A$ is smooth on compact strips. Since $U_\xi$ is harmonic, $|\nabla U_\xi|^2\asymp |\partial_\sigma U_\xi|^2$ on $\R^2_+$; thus we bound the $L^2(\sigma\,dt\,d\sigma)$ norm of $\sum_\rho (s-\rho)^{-1}$ over $Q(\alpha I)$. Decompose the (neutralized) zeros into Whitney annuli $\mathcal A_k:=\{\rho:2^kL<|\gamma-T|\le 2^{k+1}L\}$, $k\ge1$. For $V_k(\sigma,t):=\sum_{\rho\in\mathcal A_k} K_\sigma(t-\gamma)$ with $K_\sigma(x):=\sigma/(x^2+\sigma^2)$, Lemma~\ref{lem:annular-balayage} gives
\[
  \iint_{Q_\alpha(I)} V_k(\sigma,t)^2\,\sigma\,dt\,d\sigma\ \le\ C_\alpha\,|I|\,4^{-k}\,\nu_k,
\]
where $\nu_k:=\#\mathcal A_k$ and $C_\alpha$ depends only on $\alpha$. Summing Cauchy–Schwarz bounds over annuli yields
\[
  \iint_{Q(\alpha I)} \Big|\sum_{\rho}(s-\rho)^{-1}\Big|^2\,\sigma\,dt\,d\sigma\ \le\ C_\alpha\,|I|\sum_{k\ge1}4^{-k}\,\nu_k.
\]
To bound $\nu_k$, use a zero-density estimate of Vinogradov–Korobov type (e.g., Ivi\'c, Thm.~13.30; Titchmarsh, Ch.~IX): for each fixed $\sigma\in[\tfrac34,1)$,
\[
  N(\sigma,T)\ \le\ C_{\mathrm{VK}}\,T^{1-\kappa(\sigma)}(\log T)^{B_{\mathrm{VK}}},\qquad \kappa(\sigma)=\tfrac{3(\sigma-1/2)}{2-\sigma}.
\]
Translating to the Whitney geometry gives, for some $a_1(\alpha),a_2(\alpha)$ depending only on $(C_{\mathrm{VK}},B_{\mathrm{VK}},\alpha)$,
\[
  \nu_k\ \le\ a_1(\alpha)\,2^k L\,\log\langle T\rangle\ +\ a_2(\alpha)\,\log\langle T\rangle.
\]
Therefore,
\[
  \sum_{k\ge1}4^{-k}\,\nu_k\ \le\ a_1(\alpha)\,L\,\log\langle T\rangle\sum_{k\ge1}2^{-k}\ +\ a_2(\alpha)\,\log\langle T\rangle\sum_{k\ge1}4^{-k}\ \ll\ L\,\log\langle T\rangle\ +\ 1.
\]
On Whitney scale $L=c/\log\langle T\rangle$ this is $\ll 1$. Adding the neutralized near-field $O(|I|)$ and the smooth $A$ contribution, we obtain
\[
  \iint_{Q(\alpha I)} |\nabla U_\xi|^2\,\sigma\,dt\,d\sigma\ \le\ C_\xi\,|I|,
\]
with $C_\xi$ depending only on $(\alpha,c,C_{\mathrm{VK}},B_{\mathrm{VK}})$. This proves the lemma.
\end{proof}

\begin{proposition}[Whitney Carleson finiteness for $U_\xi$]\label{prop:Kxi-finite}
For each fixed Whitney aperture $\alpha\in[1,2]$ there exists a finite constant
$K_\xi=K_\xi(\alpha)<\infty$ such that
\[
  \iint_{Q(\alpha I)} |\nabla U_\xi|^2\,\sigma\,dt\,d\sigma \;\le\; K_\xi\,|I|
\]
for every Whitney base interval $I$. Consequently $C_{\rm box}^{(\zeta)}=K_0+K_\xi<\infty$, and
\[
  c \;\le\; \Big(\tfrac{c_0(\psi)}{2\,C(\psi)\,\sqrt{K_0+K_\xi}}\Big)^2
\]
ensures $\Upsilon_{\mathrm{Whit}}(c)<\tfrac12$ and closes \textup{(P+)}.
\end{proposition}

\paragraph{Boxed audit: unconditional enclosure of $C_{\rm box}^{(\zeta)}$.}
Fix $I=[T-L,T+L]$ with $L=c/\log\langle T\rangle$ and $Q(I)=I\times(0,L]$. Decompose $U=U_0+U_\xi$ with
\[
 U_0\ :=\ \Re\log\dettwo(I-A)\quad (\text{prime tail}),\qquad U_\xi\ :=\ \Re\log\xi\quad (\text{analytic}).
\]
\emph{Prime tail.} Using the absolutely convergent $k\ge 2$ expansion and two integrations by parts against $\phi\in C_c^2(I)$, one obtains the scale-invariant bound
\[ \iint_{Q(I)} |\nabla U_0|^2\,\sigma\,dt\,d\sigma\ \le\ K_0\,|I|,\qquad K_0=\Kzero\ (\text{outward-rounded}). \]
\emph{Zeros (neutralized).} Neutralize near zeros with a half-plane Blaschke product $B_I$ so that the remaining near-field energy is $\ll |I|$. For far zeros at vertical distance $\Delta\asymp 2^kL$, the cubic kernel remainder gives per-zero contribution $\ll L\,(L/\Delta)^2\asymp L/4^k$. Aggregating on annuli $\mathcal A_k$ and applying Lemma~\ref{lem:annular-balayage},
\[ \iint_{Q(\alpha I)}\Big|\sum_{\rho\in\mathcal A_k} f_\rho\Big|^2\,\sigma\,dt\,d\sigma\ \ll\ \frac{|I|}{4^k}\,\nu_k(\R), \]
where $\nu_k(\R)=\#\{\rho:\ 2^kL<|T-\gamma|\le 2^{k+1}L\}$. By the unconditional zero-density bounds of Vinogradov–Korobov (with explicit constants), for each fixed Whitney scale one has a uniform count
\[ \nu_k(\R)\ \ll\ 2^kL\log\langle T\rangle\ +\ \log\langle T\rangle, \]
with the implied constant independent of $T$ and $k$.
Summing $k\ge 1$ and using $L=c/\log\langle T\rangle$ gives
\[ \iint_{Q(\alpha I)} |\nabla U_\xi|^2\,\sigma\,dt\,d\sigma\ \le\ K_\xi\,|I|,\qquad \text{for a finite constant }K_\xi. \]
\medskip
\ifshownumerics
\noindent\fbox{\begin{minipage}{0.98\textwidth}
\textbf{Boxed $K_\xi$ audit (parametric; diagnostic).} With $C_\alpha$ from Lemma~\ref{lem:annular-balayage},
\[
  K_\xi \ \le\ C_\alpha\!\left(\frac{1}{2\pi}\sum_{j\ge1} j^{-2} \ +\ 2\sum_{j\ge1} j^{-3}\right)
  \ =\ C_\alpha\!\left(\frac{\pi}{12} \ +\ 2\,\zeta(3)\right).
\]
\end{minipage}}
\fi
Combining,
\[
\boxed{\ C_{\rm box}^{(\zeta)}\ :=\ \sup_{I}\ \frac{1}{|I|}\iint_{Q(\alpha I)} |\nabla U|^2\,\sigma\,dt\,d\sigma\ \le\ K_0+K_\xi\ =\ \CboxZeta\ .\ }
\]
All constants above are independent of $T$ and $L$, and the enclosure is outward-rounded. This is the \emph{only} Carleson input used in the active certificate.
\begin{proof}
Write
\[
 \partial_\sigma U_{\xi}(\sigma,t)\ =\ \Re\frac{\xi'}{\xi}\!\left(\tfrac12+\sigma+it\right)
 \ =\ \Re\sum_{\rho}\frac{1}{\tfrac12+\sigma+it-\rho}\ +\ A(\sigma,t),
\]
where the sum runs over nontrivial zeros $\rho=\beta+i\gamma$ of $\zeta$, and $A(\sigma,t)$ collects the archimedean part and the trivial factors (these are smooth in $(\sigma,t)$ on compact strips). Since $U_{\xi}$ is harmonic, $|\nabla U_{\xi}|^2\asymp |\partial_\sigma U_{\xi}|^2$ on $\R^2_+$; it suffices to estimate the latter.

Fix $I=[T-L,T+L]$ and decompose the zero set into near and far parts relative to $Q(I)=I\times(0,L]$:
\[
 \mathcal Z_{\mathrm{near}}:=\{\rho:\ |\gamma-T|\le 2L\},\qquad \mathcal Z_{\mathrm{far}}:=\{\rho:\ |\gamma-T|>2L\}.
\]
\subsubsection*{Neutralized near field}
Let $B_I$ be the half-plane Blaschke product over zeros with $|\gamma-T|\le 3L$ and define the neutralized potential $\widetilde U_\xi:=\Re\log\big(\xi\,B_I\big)$ and its $\sigma$-derivative $\widetilde f:=\partial_\sigma\widetilde U_\xi$. Then $\sum_{\rho\in \mathcal Z_{\mathrm{near}}}\nabla f_\rho$ is canceled inside $Q(I)$ up to a boundary error controlled by the Poisson energy of $\psi$ (independent of $T,L$). Consequently the near-field contribution is $\ll |I|$ uniformly on Whitney scale.

\noindent\emph{Remark (bound used in the certificate).} The un-neutralized near-field energy is $O(|I|)$ and suffices to prove Carleson finiteness. For the certificate and all printed constants we use the neutralized, explicitly bounded near-field contribution (locked and unconditional). The coarse un-neutralized $O(1)$ bound is not used for numeric closure.

For the far zeros (neutralized field), set annuli $\mathcal A_k:=\{\rho:\ 2^kL<|\gamma-T|\le 2^{k+1}L\}$ for $k\ge1$. For a single zero at vertical distance $\Delta:=|\gamma-T|$ one has the kernel estimate
\[
 \int_0^{L}\!\int_{T-L}^{T+L} \frac{\sigma}{\sigma^2+(t-\gamma)^2}\,dt\,d\sigma\ \ll\ L\,\Big(\frac{L}{\Delta}\Big)^{\!2}.
\]
For the far annuli $\mathcal A_k$, apply Lemma~\ref{lem:annular-balayage} to the annular Poisson sums $V_k$ to control cross terms linearly in the annular mass:
\[
  \iint_{Q(\alpha I)}\Big|\sum_{\rho\in\mathcal A_k} f_{\rho}\Big|^2\,\sigma\,dt\,d\sigma\ \ll\ \frac{|I|}{4^k}\,\nu_k(\R),
\]
where $\nu_k(\R)=\#\{\rho:\ 2^kL<|T-\gamma|\le 2^{k+1}L\}$. By the unconditional zero-density bounds of Vinogradov–Korobov (with explicit constants), for each fixed Whitney scale one has a uniform count
\[ \nu_k(\R)\ \ll\ 2^kL\log\langle T\rangle\ +\ \log\langle T\rangle, \]
with the implied constant independent of $T$ and $k$.
Summing $k\ge1$ yields a total far contribution
\[ \ll\ |I|\sum_{k\ge1}\frac{1}{4^k}\big(2^kL\log\langle T\rangle+\log\langle T\rangle\big)\ \ll\ |I|\,(L\log\langle T\rangle+1), \]
which is $\ll |I|$ on the Whitney scale $L=c/\log\langle T\rangle$.

Adding the direct near-field $O(|I|)$ bound, the far-field $O(|I|)$ sum, and the smooth Archimedean term gives
\[
 \iint_{Q(\alpha I)} |\nabla U_\xi|^2\,\sigma\,dt\,d\sigma\ \ll\ |I|.
\]
This proves the claimed Carleson bound on Whitney boxes without neutralization in the energy step.
\end{proof}
\begin{remark}[VK zero-density constants and explicit $C_\xi$]
Let $N(\sigma,T)$ denote the number of zeros with $\Re\rho\ge \sigma$ and $0<\Im\rho\le T$. The Vinogradov–Korobov zero-density estimates give, for some absolute constants $C_0,\kappa>0$, that
\[
  N(\sigma,T)\ \le\ C_0\,T\,\log T\ +\ C_0\,T^{1-\kappa(\sigma-1/2)}\qquad (\tfrac12\le \sigma<1,\ T\ge T_1),
\]
with an effective threshold $T_1$. On Whitney scale $L=c/\log\langle T\rangle$, these bounds imply the annular counts used above with explicit $A,B$ of size $\ll 1$ for each fixed $c,\alpha$. Consequently, one can take
\[
  C_\xi\ \le\ C(\alpha,c)\,\big(C_0+1\big)
\]
in Lemma~\ref{lem:carleson-xi}, where $C(\alpha,c)$ is an explicit polynomial in $\alpha$ and $c$ arising from the annular $L^2$ aggregation (cf. Lemma~\ref{lem:annular-balayage}). We do not need the sharp exponents; any effective VK pair $(C_0,\kappa)$ suffices for a finite $C_\xi$ on Whitney boxes.
\end{remark}

\section*{Formal verification (Lean/Mathlib artifact)}
\noindent Repository: \url{https://github.com/jonwashburn/rh}\\
Tag (artifact): \texttt{v1.0.1-annals}\\
Zenodo DOI: \href{https://doi.org/10.5281/zenodo.17055989}{10.5281/zenodo.17055989}.

\paragraph{Environment and reproducibility.}
The Lean toolchain is pinned in \texttt{lean-toolchain}, and dependency resolution is captured in \texttt{lake-manifest.json}. Continuous Integration builds on GitHub Actions. Only the \texttt{rh/} tree participates in the Lean build.

\paragraph{Local verification.}
\begin{verbatim}
lake update && lake build
bash scripts/verify.sh
\end{verbatim}
The script checks for the absence of \texttt{sorry/admit/axiom} under \texttt{rh/} and confirms the presence of the key theorems.

\paragraph{Key theorems and locations.}
\begin{itemize}
\item \texttt{rh/Proof/Main.lean}: \texttt{theorem RH}
\item \texttt{rh/RS/SchurGlobalization.lean}: \texttt{no\_offcritical\_zeros\_from\_schur}, \texttt{ZetaNoZerosOnRe1FromSchur}
\item \texttt{rh/academic\_framework/EulerProductMathlib.lean}: \texttt{zeta\_nonzero\_re\_eq\_one}
\end{itemize}

\paragraph{Logical chain.}
Schur globalization yields $\zeta(s)\neq 0$ on $\Omega=\{\Re s>\tfrac12\}$. Since $\xi=G\cdot\zeta$ with $G$ nonvanishing on $\Omega$, one has $\xi\neq 0$ on $\Omega$. Using the functional equation symmetry for $\xi$ (zeros map under $s\mapsto 1-s$), instantiating \texttt{RH} with $\Xi:=\xi$ places every zero on $\Re s=\tfrac12$.
% Active version of the cutoff pairing lemma (unarchived for references)
\begin{lemma}[Cutoff pairing on boxes]\label{lem:cutoff-pairing}
Fix parameters $\alpha'>\alpha>1$. Let $\chi_{L,t_0}\in C_c^\infty(\R^2_+)$ satisfy $\chi\equiv1$ on $Q(\alpha I)$, $\operatorname{supp}\chi\subset Q(\alpha'I)$, $\|\nabla\chi\|_\infty\lesssim L^{-1}$ and $\|\nabla^2\chi\|_\infty\lesssim L^{-2}$. Let $V_{\psi,L,t_0}$ be the Poisson extension of $\psi_{L,t_0}$ and $\widetilde U$ the neutralized field. Then
\[
 \int_{\R} u(t)\,\psi_{L,t_0}(t)\,dt
 \ =\ \iint_{Q(\alpha'I)} \nabla \widetilde U\cdot \nabla\big(\chi_{L,t_0}\, V_{\psi,L,t_0}\big)\,dt\,d\sigma\ +\ \mathcal R_{\mathrm{side}}\ +\ \mathcal R_{\mathrm{top}},
\]
with
\[
 |\mathcal R_{\mathrm{side}}|+|\mathcal R_{\mathrm{top}}|
 \ \lesssim\ \Big(\iint_{Q(\alpha'I)} |\nabla \widetilde U|^2\,\sigma\Big)^{1/2}
               \cdot \Big(\iint_{Q(\alpha'I)} \big(|\nabla\chi|^2\,|V_{\psi,L,t_0}|^2+|\nabla V_{\psi,L,t_0}|^2\big)\,\sigma\Big)^{1/2}.
\]
\end{lemma}
% Archived duplicate block (removed in submission branch)
% archived block removed
\begin{lemma}[CR–Green pairing for boundary phase]\label{lem:CR-green-phase}
Let $J$ be analytic on $\Omega$ with a.e. boundary modulus $|J(\tfrac12+it)|=1$, and write $\log J=U+iW$ on $\Omega$, so $U$ is harmonic with $U(\tfrac12+it)=0$ a.e. Fix a Whitney interval $I=[t_0-L,t_0+L]$ and let $V_{\psi,L,t_0}$ be the Poisson extension of $\psi_{L,t_0}$. Then, with a cutoff $\chi_{L,t_0}$ as in Lemma~\ref{lem:cutoff-pairing},
\[
  \int_{\R} \psi_{L,t_0}(t)\,\big(-W'(t)\big)\,dt\ =\ \iint_{Q(\alpha'I)} \nabla U\cdot \nabla\big(\chi_{L,t_0}\,V_{\psi,L,t_0}\big)\,dt\,d\sigma\ +\ \mathcal R_{\mathrm{side}}\ +\ \mathcal R_{\mathrm{top}},
\]
and the remainders satisfy
\[
  |\mathcal R_{\mathrm{side}}|+|\mathcal R_{\mathrm{top}}|\ \lesssim\ \Big(\iint_{Q(\alpha'I)} |\nabla U|^2\,\sigma\Big)^{1/2}\ \cdot\ \Big(\iint_{Q(\alpha'I)} (|\nabla\chi|^2\,|V|^2+|\nabla V|^2)\,\sigma\Big)^{1/2}.
\]
In particular, by Cauchy–Schwarz and the scale–invariant Dirichlet bound for $V_{\psi,L,t_0}$, there is a constant $C(\psi)$ such that
\[
  \int_{\R} \psi_{L,t_0}(t)\,\big(-w'(t)\big)\,dt\ \le\ C(\psi)\,\Big(\iint_{Q(\alpha'I)} |\nabla U|^2\,\sigma\Big)^{1/2}.
\]
Moreover, replacing $U$ by $U-\Re\log\mathcal O$ for any outer $\mathcal O$ with boundary modulus $e^{u}$ leaves the left-hand side unchanged and affects only the right-hand side through $\nabla\Re\log\mathcal O$ (Lemma~\ref{lem:outer-cancel}).
\end{lemma}
\begin{proof}[Boundary identity justification]
On the bottom edge $\{\sigma=0\}$ the outward normal is $\partial_n=-\partial_\sigma$. By Cauchy–Riemann for $\log J=U+iW$ on the boundary line $\{\Re s=\tfrac12\}$ one has $\partial_n U=-\partial_\sigma U=\partial_t W$. Hence
\[
-\int_{\partial Q\cap\{\sigma=0\}} \chi\,V\,\partial_n U\,dt\ =\ -\int_{\R} \psi_{L,t_0}(t)\,\partial_t W(t)\,dt\ =\ \int_{\R} \psi_{L,t_0}(t)\,\big(-w'(t)\big)\,dt,
\]
which yields the displayed identity after including the interior term and remainders.
\end{proof}
\begin{lemma}[Outer cancellation in the CR--Green pairing]\label{lem:outer-cancel}
With the notation of Lemma~\ref{lem:CR-green-phase}, replace $U$ by $U-\Re\log\mathcal O$, where $\mathcal O$ is any outer on $\Omega$ with a.e.\ boundary modulus $e^{u}$ and boundary argument derivative $\frac{d}{dt}\Arg\mathcal O=\Hilb[u']$ (Lemma~\ref{lem:outer-phase-HT}). Then the left-hand side of the identity in Lemma~\ref{lem:CR-green-phase} is unchanged, and the right-hand side depends only on $\nabla\!\big(U-\Re\log\mathcal O\big)$.
\end{lemma}
\begin{proof}
On the bottom edge, replacing $U$ by $U-\Re\log\mathcal O$ changes the boundary term by
$\int_{\mathbb R}\psi_{L,t_0}(t)\,\partial_t\Arg\mathcal O(\tfrac12+it)\,dt
=\int_{\mathbb R}\psi_{L,t_0}(t)\,\Hilb[u'](t)\,dt$
(Lemma~\ref{lem:outer-phase-HT}), which cancels against the outer contribution already subsumed in $-w'$. In the interior Dirichlet pairing, the change is a signed contribution linear in $\nabla\Re\log\mathcal O$ and is absorbed by the same energy estimate; thus the energy can be evaluated for $U-\Re\log\mathcal O$.
\end{proof}
\begin{corollary}[Explicit remainder control]
With notation as in Lemma~\ref{lem:CR-green-phase}, there exists $C_{\mathrm{rem}}=C_{\mathrm{rem}}(\alpha,\psi)$ such that
\[
  |\mathcal R_{\mathrm{side}}|+|\mathcal R_{\mathrm{top}}|
 \lesssim\ C_{\mathrm{rem}}\,\Big(\iint_{Q(\alpha'I)} |\nabla U|^2\,\sigma\Big)^{1/2}.
\]
In particular, one may take $C_{\mathrm{rem}}\asymp_\alpha \mathcal A(\psi)$, where $\mathcal A(\psi)$ is the fixed Poisson energy of the window (cf. Corollary~\ref{cor:CH-Mpsi-final}).
\end{corollary}
% end archived block removal
\begin{proof}
From Lemma~\ref{lem:CR-green-phase},
\[
  |\mathcal R_{\mathrm{side}}|+|\mathcal R_{\mathrm{top}}| \lesssim\ \Big(\iint_{Q(\alpha'I)} |\nabla U|^2\,\sigma\Big)^{1/2}\,\cdot\,\Big(\iint_{Q(\alpha'I)} (|\nabla\chi|^2\,|V|^2+|\nabla V|^2)\,\sigma\Big)^{1/2}.
\]
The cutoff satisfies $\|\nabla\chi\|_\infty\lesssim L^{-1}$ and is supported in a fixed dilate $Q(\alpha' I)$ with bounded overlap, while $V$ is the Poisson extension of the fixed window $\psi$; hence the second factor is $\asymp_\alpha \mathcal A(\psi)$, independent of $(T,L)$. Absorbing constants depending only on $(\alpha,\psi)$ yields the claim.
\end{proof}

% --- Outer cancellation and which energy is bounded ---
\begin{lemma}[Outer cancellation and energy bookkeeping on boxes]\label{lem:outer-energy-bookkeeping}
Let
\[
u_0(t):=\log\Big|\det\nolimits_2\!\big(I-A(\tfrac12+it)\big)\Big|,\qquad
u_\xi(t):=\log\big|\xi(\tfrac12+it)\big|,
\]
and let $O$ be the outer on $\Omega$ with boundary modulus
\(
|O(\tfrac12+it)|=\exp\!\big(u_0(t)-u_\xi(t)\big).
\)
 
\[
J(s):=\frac{\det\nolimits_2(I-A(s))}{O(s)\,\xi(s)},\qquad
\log J=U+iW,\qquad U_0:=\Re\log\det\nolimits_2(I-A),\quad U_\xi:=\Re\log\xi.
\]
Then for every Whitney interval $I=[t_0-L,t_0+L]$ and the standard test field $V_{\psi,L,t_0}$,
\begin{equation}\label{eq:CRG-outer-cancel}
\int_{\R}\psi_{L,t_0}(t)\,(-W'(t))\,dt
=\iint_{Q(\alpha' I)} \nabla\!\big(U_0-U_\xi-\Re\log O\big)\cdot\nabla\!\big(\chi_{L,t_0}V_{\psi,L,t_0}\big)\,dt\,d\sigma
+\mathcal R_{\mathrm{side}}+\mathcal R_{\mathrm{top}}
\end{equation}
and hence, by Cauchy--Schwarz and the scale‑invariant Dirichlet bound for $V_{\psi,L,t_0}$,
\begin{equation}\label{eq:energy-U-used}
\int_{\R}\psi_{L,t_0}\,(-W')\ \le\ C(\psi)\,\Big(C_{\rm box}\big(U_0-U_\xi-\Re\log O\big)\,|I|\Big)^{1/2}
\end{equation}
Moreover $\Re\log O$ is the Poisson extension of the boundary function $u:=u_0-u_\xi$, so
\begin{equation}\label{eq:Poisson-splitting}
U_0-U_\xi-\Re\log O
:=\underbrace{(U_0-\Poisson[u_0])}_{\equiv 0}\ -\ \big(U_\xi-\Poisson[u_\xi]\big)
\end{equation}
and consequently the Carleson box energy that actually enters \eqref{eq:energy-U-used} satisfies
\begin{equation}\label{eq:sharp-Kxi}
C_{\rm box}\big(U_0-U_\xi-\Re\log O\big)\ \le\ K_\xi
\end{equation}
In particular, the coarse bound
\begin{equation}\label{eq:coarse-K0Kxi}
C_{\rm box}\big(U_0-U_\xi-\Re\log O\big)\ \le\ K_0+K_\xi\ =\ \CboxZeta
\end{equation}
also holds, by the triangle inequality for $C_{\rm box}$ and linearity of the Poisson extension.
\end{lemma}

\begin{proof}
The identity \eqref{eq:CRG-outer-cancel} is Lemma~\ref{lem:CR-green-phase} with $U$ replaced by $U-\Re\log O$, together with the outer cancellation Lemma~\ref{lem:outer-cancel}; subtracting $\Re\log O$ leaves the left side (phase) unchanged. The estimate \eqref{eq:energy-U-used} follows as in Lemma~\ref{lem:CR-green-phase} from Cauchy--Schwarz and the scale‑invariant Dirichlet bound, with $C(\psi)=C_{\mathrm{rem}}(\alpha,\psi)\,\mathcal A(\psi)$ independent of $L,t_0$.

By Lemma~\ref{lem:outer-phase-HT}, $\Re\log O=\Poisson[u]$ with $u=u_0-u_\xi$, and since $U_0$ is harmonic with boundary trace $u_0$ we have $U_0=\Poisson[u_0]$, giving \eqref{eq:Poisson-splitting}. The remainder $U_\xi-\Poisson[u_\xi]$ is the (neutralized) Green potential of zeros; its Whitney–box energy is bounded by $K_\xi$ (see Lemma~\ref{lem:carleson-xi} and the annular $L^2$ aggregation), which yields \eqref{eq:sharp-Kxi}. Finally, \eqref{eq:coarse-K0Kxi} follows from the subadditivity
\(
\sqrt{C_{\rm box}(U_1+U_2)}\le \sqrt{C_{\rm box}(U_1)}+\sqrt{C_{\rm box}(U_2)}
\)
(Lemma~\ref{lem:carleson-sum}) together with $C_{\rm box}(U_0)\le K_0$ and $C_{\rm box}(U_\xi)\le K_\xi$.
\end{proof}

\noindent\emph{Consequences.}
In the CR–Green certificate the field you pair is exactly
\(
U_0-U_\xi-\Re\log O,
\)
and its box energy is controlled by $K_\xi$ (sharp) and certainly by $K_0+K_\xi=\CboxZeta$ (coarse).
The aperture dependence is confined to $C(\psi)$, not to the box constant.
% --- end snippet ---

% (Removed global reduction: certificate constants are taken as Whitney-only suprema.)
% --- Atom-safe admissible test class and uniform CR–Green estimate ---

\begin{definition}[Admissible, atom-safe test class]\label{def:admissible-class}
Fix a Whitney interval \(I=[t_0-L,t_0+L]\) (with the standing aperture schedule)
and a smooth cutoff \(\chi_{L,t_0}\) supported in \(Q(\alpha'I)\), equal to \(1\) on \(Q(\alpha I)\), with
\(\|\nabla\chi_{L,t_0}\|_\infty\lesssim L^{-1}\), \(\|\nabla^2\chi_{L,t_0}\|_\infty\lesssim L^{-2}\).
Let \(V_\varphi:=P_\sigma*\varphi\) denote the Poisson extension of \(\varphi\).


We say that a collection \(\mathcal A=\mathcal A(I)\subset C_c^\infty(I)\) is \emph{admissible}
if each \(\varphi\in\mathcal A\) is nonnegative, \(\int_{\R}\varphi=1\), and there is a constant \(A_\ast<\infty\),
independent of \(L,t_0\) and of \(\varphi\in\mathcal A\), such that the (scale-invariant) Poisson test energy obeys
\begin{equation}\label{eq:Poisson-energy-bound}
  \iint_{Q(\alpha'I)} \Big(|\nabla V_\varphi|^2 + |\nabla\chi_{L,t_0}|^2\,|V_\varphi|^2\Big)\,\sigma\,dt\,d\sigma
  \ \le\ A_\ast
\end{equation}
We call \(\mathcal A\) \emph{atom-safe} on \(I\) if, whenever \(I\) contains critical-line atoms \(\{\gamma_j\}\) for \(-w'\),
there exists \(\varphi\in\mathcal A\) with \(\varphi(\gamma_j)=0\) for all such \(\gamma_j\).
\end{definition}


\begin{lemma}[Uniform CR--Green bound for the class \(\mathcal A\)]\label{lem:uniform-CRG-A}
Let \(J\) be analytic on \(\Omega\) with a.e.\ boundary modulus \(|J(\tfrac12+it)|=1\) and write \(\log J=U+iW\) with boundary phase \(w=W|_{\sigma=0}\).
Assume the Carleson box-energy bound for \(U\) on Whitney boxes:
\[
  \iint_{Q(\alpha I)} |\nabla U|^2\,\sigma\,dt\,d\sigma \ \le\ C_{\rm box}^{(\zeta)}\,|I|\ =\ 2L\,C_{\rm box}^{(\zeta)}.
\]
If \(\mathcal A=\mathcal A(I)\) is admissible in the sense of \eqref{eq:Poisson-energy-bound},
then there exists a constant \(C_{\rm rem}=C_{\rm rem}(\alpha)\) such that, uniformly in \(I\),
\begin{equation}\label{eq:supA-bound}
  \sup_{\varphi\in\mathcal A}\ \int_{\R} \varphi(t)\,(-w'(t))\,dt
  \ \le\ C_{\rm rem}\,\sqrt{A_\ast}\,\big(C_{\rm box}^{(\zeta)}\big)^{1/2}\,L^{1/2}
  \ \ :=:\ C_{\mathcal A}\,C_{\rm box}^{(\zeta)}{}^{1/2}\,L^{1/2}
\end{equation}
\end{lemma}


\begin{proof}
For each \(\varphi\in\mathcal A\), apply the CR--Green pairing on \(Q(\alpha'I)\) to \(U\) and \(\chi_{L,t_0}V_\varphi\):
\[
  \int_{\R}\varphi(t)\,(-w'(t))\,dt
  \ =\ \iint_{Q(\alpha'I)} \nabla U\cdot\nabla(\chi_{L,t_0}V_\varphi)\,dt\,d\sigma\ +\ \mathcal R_{\mathrm{side}}+\mathcal R_{\mathrm{top}},
\]
with remainders bounded by \(C_{\rm rem}(\alpha)\) times the product of the Dirichlet norms
(of \(\nabla U\) on \(Q(\alpha'I)\) and of the test field, cf.\ \eqref{eq:Poisson-energy-bound}).
By Cauchy--Schwarz and the Carleson bound for \(U\),
\[
  \int_{\R}\varphi(-w') \ \le\ C_{\rm rem}(\alpha)\,
  \Big(\iint_{Q(\alpha'I)} |\nabla U|^2\,\sigma\Big)^{\!1/2}
  \Big(\iint_{Q(\alpha'I)} (|\nabla V_\varphi|^2+|\nabla\chi|^2|V_\varphi|^2)\,\sigma\Big)^{\!1/2}.
\]
Insert the hypotheses to obtain
\(
\int \varphi(-w') \le C_{\rm rem}(\alpha)\,\sqrt{2L\,C_{\rm box}^{(\zeta)}}\ \sqrt{A_\ast},
\)
which is \eqref{eq:supA-bound} upon setting \(C_{\mathcal A}:=C_{\rm rem}(\alpha)\sqrt{2A_\ast}\) (and absorbing absolute factors).
\end{proof}


\begin{corollary}[Atom neutralization and clean Whitney scaling]\label{cor:atom-safe}
With the notation above, the phase--velocity identity yields, for every \(\varphi\in C_c^\infty(I)\),
\[
  \int_{\R}\varphi(t)\,(-w'(t))\,dt
  \ =\ \pi\!\int_{\R}\varphi\,d\mu\ +\ \pi\sum_{\gamma\in I} m_\gamma\,\varphi(\gamma),
\]
where \(\mu\) is the Poisson balayage measure (absolutely continuous) and the sum ranges over critical-line atoms.
If \(I\) contains atoms, pick \(\varphi\in\mathcal A(I)\) with \(\varphi(\gamma)=0\) at each such atom; then the atomic term vanishes and
\[
  \int_{\R}\varphi\,(-w')\ =\ \pi\!\int \varphi\,d\mu\ \le\ C_{\mathcal A}\,C_{\rm box}^{(\zeta)}{}^{1/2}\,L^{1/2}.
\]
Thus the \(\,L^{-1}\) plateau blow-up from atoms is removed, and the Whitney\-uniform \(L^{1/2}\) bound \eqref{eq:supA-bound}
holds verbatim in the atomic case as well.
\end{corollary}


\begin{remark}[Local-to-global wedge]\label{rem:wedge-application}
The local-to-global wedge lemma only requires that on each Whitney interval \(I\) there exists a nonnegative mass\-1 bump
\(\varphi_I\) with \(\int \varphi_I(-w')\le \pi\,\Upsilon\) for some \(\Upsilon<\tfrac12\).
By Lemma~\ref{lem:uniform-CRG-A} and the Carleson bound for \(U\),
choose \(c>0\) in the Whitney schedule so that
\(C_{\mathcal A}\,C_{\rm box}^{(\zeta)}{}^{1/2}\,L^{1/2}\le \pi\Upsilon\) with \(\Upsilon<\tfrac12\).
When \(I\) contains atoms, take \(\varphi_I\in\mathcal A(I)\) vanishing at those atoms (Def.~\ref{def:admissible-class});
otherwise any \(\varphi_I\in\mathcal A(I)\) works.
The wedge then follows exactly as in the manuscript.
\end{remark}
\begin{corollary}[Unconditional local window constants]\label{cor:CH-Mpsi-final}
Define, for $I=[t_0-L,t_0+L]$ and $u$ the boundary trace of $U$, the mean-oscillation constant
\[
  M_\psi\ :=\ \sup_{L>0,\ t_0\in\R}\ \frac{1}{L}\,\Big|\int_{\R} (u(t)-u_I)\,\psi_{L,t_0}(t)\,dt\Big|,\qquad u_I:=\frac{1}{|I|}\int_I u,\quad \psi_{L,t_0}(t):=\psi\big((t-t_0)/L\big),
\]
and the Hilbert constant
\[
  C_H(\psi)\ :=\ \sup_{L>0,\ t_0\in\R}\ \frac{1}{L}\,\Big|\int_{\R} \mathcal H[u'](t)\,\psi_{L,t_0}(t)\,dt\Big|.
\]
Then there are constants $C_1(\psi),C_2(\psi)<\infty$ depending only on $\psi$ and the dilation parameter $\alpha$ such that
\[
  M_\psi\ \le\ C_1(\psi)\,\sqrt{C_{\rm box}^{(\mathrm{Whitney})}}\,\mathcal A(\psi),\qquad
  C_H(\psi)\ \le\ C_2(\psi)\,\sqrt{C_{\rm box}^{(\mathrm{Whitney})}}\,\mathcal A(\psi),
\]
where the fixed Poisson energy of the window is
\[
  \mathcal A(\psi)^2\ :=\ \iint_{\R^2_+}|\nabla(P_\sigma*\psi)|^2\,\sigma\,dt\,d\sigma\ <\ \infty.
\]
In particular, both constants are finite and determined by local box energies.
\end{corollary}
\begin{lemma}[Poisson–BMO bound at fixed height]\label{lem:poisson-bmo-strip}
Let $u\in \mathrm{BMO}(\mathbb R)$ and $U(\sigma,t):=(P_\sigma*u)(t)$ be its Poisson extension on $\Omega$. Then for every fixed $\sigma_0>0$,
\[
\sup_{t\in\mathbb R}|U(\sigma,t)|\ \le\ C_{\mathrm{BMO}}\,\|u\|_{\mathrm{BMO}}\qquad(\sigma\ge \sigma_0),
\]
with a finite constant $C_{\mathrm{BMO}}$ depending only on $\sigma_0$ and the fixed cone/box geometry. Consequently, if $\mathcal O$ is the outer with boundary modulus $e^u$, then for $\sigma\ge \sigma_0$ one has $e^{-C_{\mathrm{BMO}}\|u\|_{\mathrm{BMO}}}\le |\mathcal O(\sigma+it)|\le e^{C_{\mathrm{BMO}}\|u\|_{\mathrm{BMO}}}$.
\end{lemma}
\subsection*{Hilbert pairing via affine subtraction (uniform in $T,L$)}
% Archived duplicate block (not load-bearing in the active route)
% archived block removed
\begin{lemma}[Uniform Hilbert pairing bound (local box pairing)]\label{lem:hilbert-H1BMO}
Let $\psi\in C_c^\infty([-1,1])$ be even with $\int_\R\psi=1$ and define the mass--1 windows $\varphi_I(t)=L^{-1}\psi\big((t-T)/L\big)$. Then there exists $C_H(\psi)<\infty$ (independent of $T,L$) such that for $u$ from the smoothed Cauchy theorem,
\[
  \Big|\int_\R \mathcal H[u'](t)\,\varphi_I(t)\,dt\Big|\ \le\ C_H(\psi)\quad\text{for all intervals }I.
\]
\end{lemma}
% \fi
\begin{proof}
In distributions, $\langle \mathcal H[u'],\varphi_I\rangle=\langle u,(\mathcal H[\varphi_I])'\rangle$. Since $\psi$ is even, $(\mathcal H[\varphi_I])'$ annihilates affine functions; subtract the calibrant $\ell_I$ and write $v:=u-\ell_I$. Let $V$ be the Dirichlet test field for $(\mathcal H[\varphi_I])'$ supported in $Q(\alpha'I)$ with $\|\nabla V\|_{L^2(\sigma)}\asymp L^{1/2}\,\mathcal A(\psi)$ (scale invariance). The local box pairing (Lemma~\ref{lem:cutoff-pairing}) gives
\[
  |\langle v,(\mathcal H[\varphi_I])'\rangle|\ \le\ \Big(\iint_{Q(\alpha'I)} |\nabla \widetilde U|^2\,\sigma\Big)^{1/2}\,\cdot\,\Big(\iint_{Q(\alpha'I)} |\nabla V|^2\,\sigma\Big)^{1/2}.
\]
Using the neutralized area bound $\iint_{Q(\alpha'I)} |\nabla \widetilde U|^2\,\sigma\lesssim |I|\asymp L$ (Lemma~\ref{lem:carleson-xi}) and the fixed test energy for $V$, we obtain
\[
  |\langle v,(\mathcal H[\varphi_I])'\rangle|\ \lesssim\ (L)^{1/2}\,(L^{1/2}\,\mathcal A(\psi))\ =\ C(\psi)\,\mathcal A(\psi),
\]
uniformly in $(T,L)$. This proves the uniform bound with $C_H(\psi)\asymp \mathcal A(\psi)$.
\end{proof}
\begin{lemma}[Hilbert-transform pairing]\label{lem:hilbert}
There exists a window–dependent constant \(C_H(\psi)>0\) such that for every interval \(I\),
\[ \Big|\int_{\R} \mathcal H[u'](t)\,\varphi_I(t)\,dt\Big|\ \le\ C_H(\psi).\]
\end{lemma}
\begin{proof}
By Lemma~\ref{lem:hilbert-H1BMO}, for mass–1 windows and even \(\psi\), the pairing \(\langle \mathcal H[u'],\varphi_I\rangle\) is uniformly bounded in \((T,L)\). In distributions, \(\langle \mathcal H[u'],\varphi_I\rangle=\langle u,(\mathcal H[\varphi_I])'\rangle\); evenness implies \((\mathcal H[\varphi_I])'\) annihilates affine functions. Subtract the affine calibrant on \(I\) and write \(v=u-\ell_I\). The bound follows from the local box pairing in the Carleson energy lemma (Lemma~\ref{lem:carleson-xi}) applied to the test field associated with \((\mathcal H[\varphi_I])'\).
\end{proof}
% --- PSC route moved to archived appendix; placeholder removed from main chain ---
We adopt the \(\zeta\)-normalized boundary route with the half-plane Blaschke compensator \(B(s)=(s-1)/s\) to cancel the pole at \(s=1\). On \(\Re s=\tfrac12\), \(|B|=1\), so the compensator contributes no boundary phase and the Archimedean term vanishes. We print a concrete even mass--1 window \(\psi\), derive \(c_0(\psi)\), \(C_H(\psi)\), and use the product certificate
\[
  \frac{(2/\pi)\,M_\psi}{c_0(\psi)}\ <\ \frac{\pi}{2}.
\]
\paragraph{Printed window.}
Let \(\beta(x):=\exp\!\big(-1/(x(1-x))\big)\) for \(x\in(0,1)\) and \(\beta=0\) otherwise. Define the smooth step
\[
  S(x):=\frac{\int_0^{\min\{\max\{x,0\},1\}} \beta(u)\,du}{\int_0^{1} \beta(u)\,du}\qquad (x\in\R),
\]
so that \(S\in C^\infty(\R)\), \(S\equiv 0\) on \(({-}\infty,0]\), \(S\equiv1\) on \([1,\infty)\), and \(S'\ge 0\) supported on \((0,1)\). Set the even flat-top window \(\psi:\R\to[0,1]\) by
\[
  \psi(t):=\begin{cases}
    0,& |t|\ge 2,\\
    S(t+2),& -2<t<-1,\\
    1,& |t|\le 1,\\
    S(2-t),& 1<t<2.
  \end{cases}
\]
Then \(\psi\in C_c^\infty(\R)\), \(\psi\equiv 1\) on \([-1,1]\), and \(\operatorname{supp}\psi\subset[-2,2]\). For windows we take \(\varphi_L(t):=L^{-1}\psi(t/L)\).

\paragraph{Poisson lower bound.}
\begin{lemma}[Poisson plateau lower bound]\label{lem:poisson-plateau}
For the printed even window \(\psi\) with \(\psi\equiv 1\) on \([-1,1]\),
\[ c_0(\psi)\ :=\ \inf_{0<b\le 1,\ |x|\le 1} (\Poisson_b*\psi)(x)\ \ge\ \frac{1}{2\pi}\,\arctan 2. \]
\end{lemma}
As in the plateau computation already recorded, for \(0<b\le 1\) and \(|x|\le 1\) one has
\[
 (\Poisson_b*\psi)(x)\ \ge\ (\Poisson_b*\mathbf 1_{[-1,1]})(x)
  = \frac{1}{2\pi}\Big(\arctan\tfrac{1-x}{b}+\arctan\tfrac{1+x}{b}\Big),
\]
whence
\[
 c_0(\psi)\ :=\ \inf_{0<b\le 1,\ |x|\le 1} (\Poisson_b*\psi)(x)\ \ge\ 0.1762081912\,.
\]
\begin{proof}
For the normalized Poisson kernel \(P_b(y)=\dfrac{1}{\pi}\dfrac{b}{b^2+y^2}\), for \(|x|\le 1\)
\[
 (P_b*\mathbf 1_{[-1,1]})(x)=\frac{1}{\pi}\int_{-1}^{1}\frac{b}{b^2+(x-y)^2}\,dy=\frac{1}{2\pi}\Big(\arctan\frac{1-x}{b}+\arctan\frac{1+x}{b}\Big).
\]
Set \(S(x,b):=\arctan\big((1-x)/b\big)+\arctan\big((1+x)/b\big)\). Symmetry gives \(S(-x,b)=S(x,b)\). For \(x\in[0,1]\),
\[
 \partial_x S(x,b)=\frac{1}{b}\Big(\frac{1}{1+\big(\tfrac{1+x}{b}\big)^2}-\frac{1}{1+\big(\tfrac{1-x}{b}\big)^2}\Big)\le 0,
\]
so \(S\) decreases in \(x\) and is minimized at \(x=1\). Also \(\partial_b S(x,b)\le 0\) for \(b>0\), so the minimum in \(b\in(0,1]\) is at \(b=1\). Thus the infimum occurs at \((x,b)=(1,1)\) giving \(\frac{1}{2\pi}\arctan 2=0.1762081912\ldots\). Since \(\psi\ge \mathbf 1_{[-1,1]}\), this yields the bound for \(\psi\).
\end{proof}
\paragraph{No Archimedean term in the \(\zeta\)-normalized route.}
Writing \(J_\zeta:=\dettwo(I-A)/\zeta\) and \(J_{\mathrm{comp}}:=J_\zeta\,B\), one has \(|B|=1\) on the boundary and no Gamma factor in \(J_\zeta\). Hence the boundary phase contribution from Archimedean factors is identically zero in the phase–velocity identity, i.e. \(C_\Gamma\equiv 0\) for this normalization.

% (bridge AAB archived)
We carry out the boundary phase test in the $\zeta$–normalized gauge with the Blaschke compensator at $s=1$; on $\Re s=\tfrac12$ one has $|B|=1$, so the Archimedean boundary contribution vanishes. Any residual interior effect is absorbed into the $\zeta$–side box constant $C_{\mathrm{box}}^{(\zeta)}$. In the a.e. wedge route no additive wedge constants are used.

\paragraph{Hilbert term (structural bound).}
For the mass--1 window and even \(\psi\), the local box pairing bound of Lemma~\ref{lem:hilbert-H1BMO} applies and is uniform in \((T,L)\). We write the certificate in terms of the abstract window-dependent constant \(C_H(\psi)\) from Lemma~\ref{lem:hilbert-H1BMO}. An explicit envelope for the printed window is recorded below, but is not required for the symbolic certificate.
\begin{lemma}[Explicit envelope for the printed window]\label{lem:CH-explicit}
For the flat-top \(\psi\) above with symmetric monotone ramps of width \(\varepsilon\in(0,1)\) on each side of \(\pm1\), one has the variation bound
\[
  \sup_{t\in\R}\,|\mathcal H[\varphi_L](t)|\ \le\ \frac{\mathrm{TV}(\psi)}{\pi}\,\log\frac{1+\varepsilon}{1-\varepsilon},\qquad \mathrm{TV}(\psi)=2.
\]
In particular, with \(\varepsilon=\tfrac15\) one obtains the certified envelope
\[
  \sup_{t\in\R}\,|\mathcal H[\varphi_L](t)|\ \le\ \frac{2}{\pi}\,\log\tfrac{3}{2}\ \approx\ 0.258\ <\ 0.26.
\]
Consequently, we may take \(C_H(\psi)\le 0.26\) for the printed window. This bound is uniform in \(L\).
\end{lemma}
\begin{lemma}[Derivative envelope: $C_H(\psi)\le 2/\pi$]\label{lem:CH-derivative-2pi}
For the printed flat–top window \(\psi\) (even, plateau on $[-1,1]$), with \(\varphi_L(t)=L^{-1}\psi((t-T)/L)\) one has
\[ \sup_{t\in\R}\,|\mathcal H[\varphi_L](t)|\ \le\ \frac{2}{\pi}\,\log\frac{1+\varepsilon}{1-\varepsilon}\quad\text{and}\quad \big\|\big(\mathcal H[\varphi_L]\big)'\big\|_{L^\infty(\R)}\ \le\ \frac{2}{\pi}\,\frac{1}{L}. \]
In particular, $C_H(\psi)\le 2/\pi$.
\end{lemma}
\begin{proof}
By scaling, \(\mathcal H[\varphi_L](t)=\mathcal H\psi((t-T)/L)\) and \(\big(\mathcal H[\varphi_L]\big)'(t)=\tfrac{1}{L}\,(\mathcal H\psi)'((t-T)/L)\). Since \(\psi'\equiv 0\) on \((-1,1)\) and the ramps are monotone on \([-1-\varepsilon,-1]\) and \([1,1+\varepsilon]\) with total variation \(2\), the variation/IBP argument of Lemma~\ref{lem:CH-explicit} yields the stated envelope and its derivative bound. Taking the supremum in \(t\) gives the \(2/\pi\) constant uniformly in \(L\).
\end{proof}
\begin{proof}[Derivation (variation/IBP estimate)]
Write \(\psi=\mathbf 1_{[-1,1]}+\eta\) with \(\eta\) supported on the disjoint transition layers \([1,1+\varepsilon]\) and \([-1-\varepsilon,-1]\), monotone on each layer, and total variation \(\mathrm{TV}(\psi)=2\). Using the identity \(\mathcal H[\psi](x)=\tfrac{1}{\pi}\,\mathrm{p.v.}\int \tfrac{\psi(y)}{x-y}\,dy=\tfrac{1}{\pi}\int \psi'(y)\,\log|x-y|\,dy\) (integration by parts; boundary cancellations by monotonicity/symmetry) and that \(\psi'\) is a finite signed measure of total variation \(\mathrm{TV}(\psi)\), one gets
\[
  |\mathcal H\psi(x)|\ \le\ \frac{\mathrm{TV}(\psi)}{\pi}\,\sup_{y\in[-1-\varepsilon,\,1+\varepsilon]}\big|\log|x-y|\big|\ -\ \inf_{y\in[-1-\varepsilon,\,1+\varepsilon]}\big|\log|x-y|\big|.
\]
The worst case is at \(x=0\), yielding \(|\mathcal H\psi(0)|\le \tfrac{\mathrm{TV}(\psi)}{\pi}\log\tfrac{1+\varepsilon}{1-\varepsilon}\). Scaling gives \(\mathcal H[\varphi_L](t)=\mathcal H\psi\big((t-T)/L\big)\), so the same bound holds uniformly in \(L\). Taking \(\varepsilon=\tfrac15\) gives the stated numeric envelope.
\end{proof}
\paragraph{Window mean-oscillation constant \(M_\psi\): definition and bound.}
For an interval \(I=[T{-}L,T{+}L]\) and the boundary modulus \(u(t):=\log\big|\dettwo(I{-}A(\tfrac12{+}it))\big|{-}\log\big|\xi(\tfrac12{+}it)\big|\), define the mean-oscillation calibrant \(\ell_I\) as the affine function matching \(u\) at the endpoints of \(I\), and set
\[
  M_\psi\ :=\ \sup_{T\in\R,\ L>0}\ \frac{1}{|I|}\int_I \big|u(t)-\ell_I(t)\big|\,dt.
\]
By the smoothed Cauchy theorem and the local pairing in a local pairing bound, one obtains a window-dependent constant bounding the mean oscillation uniformly over $(T,L)$. For the printed flat-top window, Lemma~\ref{lem:Mpsi-correct} yields an explicit H$^1$--BMO/box-energy bound for $M_\psi$; in our calibration (see Numeric instantiation below), this gives a strict numerical bound well below the certificate threshold.
\begin{lemma}[Window mean--oscillation via H$^1$--BMO and box energy]\label{lem:Mpsi-correct}
Let $U$ be the Poisson extension of the boundary function $u$, and let $\mu := |\nabla U|^2\,\sigma\,dt\,d\sigma$.
Fix the even $C^\infty$ window $\psi$ (support $\subset[-2,2]$, plateau on $[-1,1]$), and let $m_\psi:=\int_{\R}\psi(x)\,dx$ denote its mass. Set
\[
\phi(t):=\psi(t)-\tfrac{m_\psi}{2}\,\mathbf 1_{[-1,1]}(t),\qquad 
\phi_{L,t_0}(t):=\phi\!\Big(\frac{t-t_0}{L}\Big).
\]
Define $M_\psi:=\sup_{L>0,t_0\in\R}\frac1L\big|\int_\R u(t)\,\phi_{L,t_0}(t)\,dt\big|$ and
\[
C_{\rm box}^{(\mathrm{Whitney})}:=\sup_{I\,:\,|I|\asymp c/\log\langle T\rangle}\frac{\mu(Q(\alpha I))}{|I|},\qquad
C_\psi^{(H^1)}:=\frac12\int_{\R} S\phi(x)\,dx,
\]
where $S$ is the Lusin area function for the Poisson semigroup with cone aperture $\alpha$.
Then
\[
M_\psi\ \le\ \frac{4}{\pi}\,C_{\mathrm{CE}}(\alpha)\,C_\psi^{(H^1)}\,\sqrt{C_{\rm box}^{(\mathrm{Whitney})}}.
\]
\end{lemma}
\begin{proof}
By H$^1$--BMO duality, for every $I=[t_0-L,t_0+L]$,
\[ \Big|\int u\,\phi_{L,t_0}\Big|\ \le\ \|u\|_{\rm BMO}\,\|\phi_{L,t_0}\|_{H^1}. \]
Carleson embedding (aperture $\alpha$) gives
\[ \|u\|_{\rm BMO}\ \le\ \tfrac{2}{\pi}\,C_{\mathrm{CE}}(\alpha)\,\big(C_{\rm box}^{(\mathrm{Whitney})}\big)^{1/2}. \]
Since $S$ is scale-invariant in $L^1$ (up to $|I|$),
\[ \|\phi_{L,t_0}\|_{H^1}\ =\ \int S(\phi_{L,t_0})(x)\,dx\ =\ 2L\,C_\psi^{(H^1)}. \]
Divide by $L$ to conclude.
\end{proof}
\paragraph{Carleson box linkage.}
With $U=U_{\det_2}+U_{\xi}$ on the boundary in the $\zeta$–normalized route, the box constant used in the certificate is
\[
  C_{\mathrm{box}}^{(\zeta)}\ :=\ K_0\ +\ K_\xi.
\]
No separate $\Gamma$–area term enters the certificate path.

% shownumerics gated section (disabled)
\ifshownumerics
\paragraph{Numeric instantiation (diagnostic; gated).}
All concrete values (audited constants for $K_0$, $K_\xi$, the $\zeta$–side box constant $C_{\mathrm{box}}^{(\zeta)}$, the evaluation of $C_\psi^{(H^1)}$, and the locked $M_\psi$) are collected for reproducibility; the proof of (P+) uses only the CR–Green right-hand side with the box constant.
\begin{itemize}
  \item \textbf{Window:} fixed $C^\infty$ even $\psi$ with $\psi\equiv 1$ on $[-1,1]$ and $\mathrm{supp}\,\psi\subseteq[-2,2]$, and $\varphi_L(t)=L^{-1}\psi(t/L)$.
  \item \textbf{Poisson lower bound.} Using the closed form for the plateau and monotonicity, $c_0(\psi)\ge 0.1762081912$.
  \item \textbf{Archimedean term.} In the $\zeta$-normalized route with the Blaschke compensator at $s=1$, $C_\Gamma=0$.
  \item \textbf{Hilbert term.} We retain $C_H(\psi)$ symbolically; an explicit envelope can be inserted.
  \item \textbf{Inequality form.} With $M_\psi= (4/\pi)\,C_\psi^{(H^1)}\,\sqrt{C_{\mathrm{box}}^{(\zeta)}}$, the display $\frac{(2/\pi)\,M_\psi}{c_0(\psi)}<\frac{\pi}{2}$ is diagnostic.
\end{itemize}
 
\fi
\subsection*{Explicit proofs and constants for key lemmas (archimedean, prime-tail, Hilbert)}
We record complete proofs with explicit constants, making finiteness and dependence on the window $\psi$ transparent.
% Duplicate prime-tail subsection removed (see earlier \S{subsec:prime-tail})
\begin{equation}\label{eq:P1}
 \sum_{p>x} p^{-\alpha}\ \le\ \frac{1.25506\,\alpha}{(\alpha-1)\,\log x}\,x^{\,1-\alpha}
\end{equation}
This follows by partial summation together with $\pi(t)\le 1.25506\,t/\log t$ for $t\ge 17$. A uniform variant over $\alpha\in[\alpha_0,2]$ (with $\alpha_0:=2\sigma_0>1$) is
\begin{equation}\label{eq:P1uniform}
 \sum_{p>x} p^{-\alpha}\ \le\ \frac{1.25506\,\alpha_0}{(\alpha_0-1)\,\log x}\,x^{\,1-\alpha_0}\qquad(x\ge 17)
\end{equation}
Two convenient alternatives:
\begin{align}
 \sum_{p>x}p^{-\alpha}&\ \le\ \frac{\alpha}{(\alpha-1)(\log x-1)}\,x^{1-\alpha}\qquad(x\ge 599)\label{eq:P1dusart}\\
 \sum_{p>x}p^{-\alpha}&\ \le\ \sum_{n>\lfloor x\rfloor}n^{-\alpha}\ \le\ \frac{x^{1-\alpha}}{\alpha-1}\qquad(x>1).\label{eq:P1triv}
\end{align}
\begin{proof}[Proof of \eqref{eq:P1}--\eqref{eq:P1triv}]
Fix $\alpha>1$ and $x\ge 17$. For $u>1$ write $f(u):=u^{-\alpha}$. By Stieltjes integration with $d\pi(u)$ and one integration by parts,
\[
\sum_{p\le y} p^{-\alpha}
=\int_{2^-}^{y} u^{-\alpha}\,d\pi(u)
= y^{-\alpha}\pi(y)+\alpha\!\int_{2}^{y} \pi(u)\,u^{-\alpha-1}\,du.
\]
Letting $y\to\infty$ and using $\alpha>1$ (so $y^{-\alpha}\pi(y)\to 0$) gives the exact tail identity
\begin{equation}\label{eq:P1-exact}
\sum_{p>x} p^{-\alpha}
=\alpha\!\int_{x}^{\infty}\!\pi(u)\,u^{-\alpha-1}\,du\;-\;x^{-\alpha}\pi(x)
\ \le\ \alpha\!\int_{x}^{\infty}\!\pi(u)\,u^{-\alpha-1}\,du
\end{equation}
For $u\ge x\ge 17$ we have the explicit bound $\pi(u)\le 1.25506\,\dfrac{u}{\log u}$. Inserting this into \eqref{eq:P1-exact} and using $1/\log u\le 1/\log x$ for $u\ge x$ yields
\[
\sum_{p>x} p^{-\alpha}
\ \le\ \frac{1.25506\,\alpha}{\log x}\!\int_{x}^{\infty}\!u^{-\alpha}\,du
\ =\ \frac{1.25506\,\alpha}{(\alpha-1)\,\log x}\,x^{\,1-\alpha},
\]
which is \eqref{eq:P1}. For the uniform version, if $\alpha\in[\alpha_0,2]$ with $\alpha_0>1$, then the map $\alpha\mapsto \alpha/(\alpha-1)$ is decreasing and $x^{1-\alpha}\le x^{1-\alpha_0}$, so \eqref{eq:P1uniform} follows immediately from \eqref{eq:P1}.

For \eqref{eq:P1dusart}, assume $x\ge 599$ and use the sharper pointwise bound $\pi(u)\le \dfrac{u}{\log u-1}$ for $u\ge x$. Then
\[
\sum_{p>x} p^{-\alpha}
\ \le\ \alpha\!\int_{x}^{\infty}\!\frac{u^{-\alpha}}{\log u-1}\,du
\ \le\ \frac{\alpha}{\log x-1}\!\int_{x}^{\infty}\!u^{-\alpha}\,du
\ =\ \frac{\alpha}{(\alpha-1)(\log x-1)}\,x^{1-\alpha}.
\]

Finally, \eqref{eq:P1triv} is the integer-majorant: $\sum_{p>x}p^{-\alpha}\le \sum_{n>\lfloor x\rfloor}n^{-\alpha}=\dfrac{x^{1-\alpha}}{\alpha-1}$ for $x>1$.
\end{proof}

\begin{lemma}[Monotonicity of the tail majorant]\label{lem:P1-monotone}
For fixed $\alpha>1$, the function $g(P):=\dfrac{P^{\,1-\alpha}}{\log P}$ is strictly decreasing on $P>1$.
\end{lemma}
\begin{proof}
Writing $\log g(P)=(1-\alpha)\log P-\log\log P$ gives
$(\log g)'=\dfrac{1-\alpha}{P}-\dfrac{1}{P\log P}<0$ for $P>1$.
\end{proof}

\begin{corollary}[Minimal tail parameter for a target $\eta$]\label{cor:P1-minP}
Given $\alpha>1$, $x_0\ge 17$ and target $\eta>0$, define $P_\eta$ to be the smallest integer $P\ge x_0$ such that
\[
\frac{1.25506\,\alpha}{(\alpha-1)\,\log P}\,P^{1-\alpha}\ \le\ \eta.
\]
By Lemma~\ref{lem:P1-monotone} this $P_\eta$ exists and is unique; moreover, the inequality then holds for every $P\ge P_\eta$. (The same definition with $\log P$ replaced by $\log P-1$ gives the $x_0\ge 599$ Dusart variant.)
\end{corollary}
\paragraph{Use in $(\star)$ and covering.}
To enforce a tail $\sum_{p>P}p^{-\alpha}\le \eta$ it suffices, by \eqref{eq:P1}, to take $P\ge17$ solving
\[
 \frac{1.25506\,\alpha}{(\alpha-1)\,\log P}\,P^{\,1-\alpha}\ \le\ \eta.
\]
The practical choice $P=\max\{17,\ ((1.25506\,\alpha)/((\alpha-1)\eta))^{1/(\alpha-1)}\}$ already meets the inequality up to the mild $\log P$ factor; one may increase $P$ monotonically until the left side is $\le\eta$.
\subsection*{Finite-block spectral gap certificate on $[\sigma_0,1]$}
Let $\sigma_0\in(\tfrac12,1]$ and $\mathcal I=\{(p,n):\ p\le P\text{ prime},\ 1\le n\le N_p\}$. Let $H(\sigma)\in\C^{|\mathcal I|\times|\mathcal I|}$ be the Hermitian block matrix of the truncated finite block at abscissa $\sigma$, partitioned as $H=[H_{pq}]_{p,q\le P}$ with $H_{pq}(\sigma)\in\C^{N_p\times N_q}$. Write $D_p(\sigma):=H_{pp}(\sigma)$ and $E(\sigma):=H(\sigma)-\mathrm{diag}(D_p(\sigma))$.
\begin{lemma}[Block Gershgorin lower bound]\label{lem:block-gersh}
For every $\sigma\in[\sigma_0,1]$,
\[
  \lambda_{\min}\big(H(\sigma)\big)\ \ge\ \min_{p\le P}\Big(\lambda_{\min}\big(D_p(\sigma)\big)\ -\ \sum_{q\ne p}\|H_{pq}(\sigma)\|_2\Big).
\]
\end{lemma}
\begin{lemma}[Schur--Weyl bound]\label{lem:schur-weyl-gap}
For every $\sigma\in[\sigma_0,1]$,
\[
  \lambda_{\min}\big(H(\sigma)\big)\ \ge\ \delta(\sigma_0),\qquad \delta(\sigma_0):=\max\Big\{0,\ \min_p\Big(\mu_p^L-\sum_{q\ne p}U_{pq}\Big),\ \min_p \mu_p^L\ -\ \max_q\frac{1}{\sqrt{\mu_q^L}}\sum_{p\ne q}\sqrt{\mu_p^L}\,U_{pq}\Big\}.\]
\end{lemma}
\subsection*{Determinant--zeta link (L1; corrected domain)}

\begin{remark}[Using prime-tail bounds]
If $\|H_{pq}(\sigma)\|_2\le C(\sigma_0)(pq)^{-\sigma_0}$ for $p\ne q$, then $\sum_{q\ne p}U_{pq}\le C(\sigma_0)\,p^{-\sigma_0}\sum_{q\le P} q^{-\sigma_0}$, and the sum is bounded explicitly by the Rosser--Schoenfeld tail with $\alpha=2\sigma_0>1$. Thus $\delta(\sigma_0)>0$ can be certified by choosing $P,\{N_p\}$ so that the off-diagonal budget is dominated by $\min_p\mu_p^L$.
\end{remark}

\subsection*{Truncation tail control and global assembly (P4)}
Write the head/tail split by primes as $\mathcal P_{\le P}=\{p\le P\}$ and $\mathcal P_{>P}=\{p>P\}$. In the normalised basis at $\sigma_0$ set
\[ X:=\bigl[\widetilde H_{pq}\bigr]_{p,q\le P},\quad Y:=\bigl[\widetilde H_{pq}\bigr]_{p\le P<q},\quad Z:=\bigl[\widetilde H_{pq}\bigr]_{p,q>P}. \]
Let $A_p^2:=\sum_{i\le N_p} w_i^2$ denote the block weight squares (unweighted: $A_p^2=N_p$; weighted example $w_n=3^{-(n+1)}$ gives $A_p^2\le\tfrac18$). Define
\[ S_2(\le P):=\sum_{p\le P} A_p^2 p^{-2\sigma_0},\qquad S_2(>P):=\sum_{p>P} A_p^2 p^{-2\sigma_0}. \]
Then
\[ \|Y\|\le C_{\rm win}\sqrt{S_2(\le P)S_2(>P)},\qquad \lambda_{\min}(Z)\ge \mu_{\mathrm{diag}}-C_{\rm win}S_2(>P), \]
where $\mu_{\mathrm{diag}}:=\inf_{p>P}\mu_p^{\mathrm L}$. Consequently,
\[ \lambda_{\min}(\mathbb A)\ge \min\Big\{\,\delta_P - \dfrac{C_{\rm win}^2 S_2(\le P)S_2(>P)}{\mu_{\mathrm{diag}}-C_{\rm win}S_2(>P)}\,,\ \mu_{\mathrm{diag}}-C_{\rm win}S_2(>P)\Big\}, \]
with $\delta_P$ the head finite-block gap from above. Using the integer tail $\sum_{n>P}n^{-2\sigma_0}\le (P-1)^{1-2\sigma_0}/(2\sigma_0-1)$ yields a closed-form tail bound for $S_2(>P)$.
\paragraph{Small-prime disentangling (P3).}
Excising $\{p\le Q\}$ improves the head budget by at least $\min_{p>Q}\sum_{q\le Q}\|\widetilde H_{pq}\|$, which in the unweighted case is $\ge N_{\max} P^{-\sigma_0} S_{\sigma_0}(Q)$ and in the weighted case $\ge \tfrac14 P^{-\sigma_0} S_{\sigma_0}(Q)$, with $S_{\sigma_0}(Q)=\sum_{p\le Q}p^{-\sigma_0}$.

\subsection*{No-hidden-knobs audit (P6)}
All constants in $(\star)$, (4), and the gap $B$ are fixed by explicit inequalities: prime tails via integer or Rosser--Schoenfeld bounds, weights $w_n=3^{-(n+1)}$ with $\sum w=1/2$, off-diagonal $U_{pq}\le (\sum w^{(p)})(\sum w^{(q)})(pq)^{-\sigma_0}\le \tfrac14 (pq)^{-\sigma_0}$, and in-block $\mu_p^{\rm L}$ by interval Gershgorin/LDL$^\top$. No tuned parameters enter; $P(\sigma_0,\varepsilon)$, $N_p(\sigma_0,\varepsilon,P)$, and $B$ are determined from these definitions.

\paragraph{Explicit prime-side difference (unconditional bandlimit estimate; archived, not used in the proof route).}
Let $\mathcal P(t):=\Im\big((\zeta'/\zeta)-(\dettwo'/\dettwo)\big)(\tfrac12+it)=\sum_{p}(\log p)\,p^{-1/2}\sin(t\log p)$. Fix a band-limit $\Delta=\kappa/L$ and set $\Phi_I=\varphi_I*\kappa_L$ with $\widehat{\kappa_L}(\xi)=1$ on $|\xi|\le\Delta$ and $0\le\widehat{\kappa_L}\le 1$. By Plancherel and Cauchy–Schwarz,
\[
 \left|\int_\R \!\mathcal P(t)\,\Phi_I(t)\,dt\right|
 \le \Bigg(\sum_{\log p\le \kappa/L}\frac{(\log p)^2}{p}\,|\widehat{\Phi_I}(\log p)|^2\Bigg)^{\!1/2}
 \cdot\Bigg(\sum_{\log p\le \kappa/L}1\Bigg)^{\!1/2}.
\]
Since $|\widehat{\Phi_I}(\xi)|\le L\,|\widehat{\psi}(L\xi)|\,\|\widehat{\kappa_L}\|_\infty\le L\,\|\psi\|_{L^1}$ and, unconditionally, $\sum_{p\le x}(\log p)^2/p\ll (\log x)^2$ by partial summation and Chebyshev's bound $\theta(x)\ll x$ (Titchmarsh), we obtain
\[
 \left|\int \!\mathcal P\,\Phi_I\right|\ \le\ \sqrt{2}\,\|\psi\|_{L^1}\,\frac{\kappa}{L}\,L\ =\ \sqrt{2}\,\|\psi\|_{L^1}\,\kappa.
\]
Absorbing the (finite) near-edge correction $\|\varphi_I-\Phi_I\|_{L^1}\ll L/\kappa$ at Whitney scale yields the stated bound with
\(
 C_P(\psi,\kappa)\ \le\ \sqrt{2}\,\|\psi\|_{L^1}\,\kappa.
\)
\begin{theorem}[Limit \(N\to\infty\) on rectangles: \(2J\) Herglotz, \(\Theta\) Schur]\label{thm:limit-rect}
Let \(R\Subset\Omega\) with \(\xi\neq 0\) on a neighborhood of \(\overline R\). Then \(2\mathcal J_N\to 2\mathcal J\) locally uniformly on \(R\), and \(\Re(2\mathcal J)\ge 0\) on \(R\). Consequently, \(\Theta=(2\mathcal J-1)/(2\mathcal J+1)\) is Schur on \(R\).
\end{theorem}
\begin{proof}
By the \(HS\to\dettwo\) convergence proposition, $\dettwo(I-A_N)\to \dettwo(I-A)$ locally uniformly on $R$. Since $\xi$ is bounded away from zero on $R$, division is continuous, hence $\mathcal J_N\to \mathcal J$ locally uniformly on $R$. Each $2\mathcal J_N$ is Herglotz on $R$, and Herglotz functions are closed under local-uniform limits; therefore $\Re(2\mathcal J)\ge 0$ on $R$. The Cayley transform yields that $\Theta$ is Schur on $R$.

For completeness: local-uniform convergence of holomorphic functions implies pointwise convergence, hence $\Re(2\mathcal J)(z)=\lim_N \Re(2\mathcal J_N)(z)\ge 0$ for every $z\in R$, since each $\Re(2\mathcal J_N)\ge 0$ on $R$. Continuity of the Cayley map on compacta avoiding $\{-1\}$ preserves the contractive bound, so $|\Theta(z)|=\lim_N |\Theta_N(z)|\le 1$ for $z\in R$.
\end{proof}
\begin{remark}[Boundary uniqueness and (H+) on $R$]\label{rem:boundary-uniqueness}
If $\Re F\ge 0$ holds a.e. on $\partial R$ and $F$ is holomorphic on $R$, then the Herglotz–Poisson integral $H$ with boundary data $\Re F$ satisfies $\Re H\ge 0$ and shares the a.e. boundary values with $\Re F$. By boundary uniqueness for Smirnov/Hardy classes on rectangles, $\Re F\ge 0$ in $R$; hence (H+) holds. We use this in tandem with the $N\to\infty$ passage above.
\end{remark}
\begin{corollary}[Unconditional Schur on \(\Omega\setminus Z(\xi)\)]\label{cor:Schur-off-zeros}
For every compact \(K\Subset \Omega\setminus Z(\xi)\), there exists a rectangle \(R\Subset\Omega\) with \(K\subset R\) and \(\xi\neq 0\) on \(\overline R\). Hence, by Theorem~\ref{thm:limit-rect}, \(\Theta\) is Schur on $R$, and therefore on $K$. Exhausting \(\Omega\setminus Z(\xi)\) by such \(K\) shows that \(\Theta\) is Schur on \(\Omega\setminus Z(\xi)\).
\end{corollary}
\begin{lemma}[Removable singularity under Schur bound]\label{lem:removable-schur}
Let $D\subset\Omega$ be a disc centered at $\rho$ and let $\Theta$ be holomorphic on $D\setminus\{\rho\}$ with $|\Theta|<1$ there. Then $\Theta$ extends holomorphically to $D$. In particular, the Cayley inverse $(1+\Theta)/(1-\Theta)$ extends holomorphically to $D$ with nonnegative real part.
\end{lemma}
\begin{proof}
Since $\Theta$ is bounded on the punctured disc $D\setminus\{\rho\}$, Riemann's removable singularity theorem yields a holomorphic extension of $\Theta$ to $D$. Where $|\Theta|<1$, the Cayley inverse is analytic with $\Re\tfrac{1+\Theta}{1-\Theta}\ge 0$; continuity extends this across $\rho$.
\end{proof}

% (Removed duplicate theorem statement; see Theorem~\ref{thm:globalize-main}.)


\begin{corollary}[Zero-free right half-plane]
Assuming removability across $Z(\xi)$ (Lemma~\ref{lem:removable-schur}) and the (N1)–(N2) pinch in Section~\ref{sec:globalization}, one has $\xi(s)\neq 0$ for all $s\in\Omega$.
\emph{Proof.}
On $\Omega\setminus Z(\xi)$, $2\mathcal J$ is Herglotz and $\Theta$ is Schur; removability extends across each $\rho\in Z(\xi)$. The pinch then rules out any off–critical zero, hence $Z(\xi)\cap\Omega=\varnothing$ and RH holds. \qed
\end{corollary}
\begin{corollary}[Conclusion (RH)]
By the functional equation $\xi(s)=\xi(1-s)$ and conjugation symmetry, zeros are symmetric with respect to the critical line. Since there are no zeros in $\Re s>\tfrac12$ and none in $\Re s<\tfrac12$ by symmetry, every nontrivial zero lies on $\Re s=\tfrac12$. This completes the proof.
\end{corollary}

\begin{corollary}[Poisson transport]\label{cor:poisson-herglotz}
From Theorem~\ref{thm:psc-certificate-stage2}, $2\mathcal J$ is Herglotz on $\Omega\setminus Z(\xi)$.
\end{corollary}

\begin{corollary}[Cayley]\label{cor:cayley-schur}
$\Theta=\frac{2\mathcal J-1}{2\mathcal J+1}$ is Schur on $\Omega\setminus Z(\xi)$ (see also \cite{RosenblumRovnyak,SarasonSubHardy}).
\end{corollary}
\begin{theorem}[Globalization across $Z(\xi)$]\label{thm:globalize-main}
Under (P+), $2\mathcal J$ is Herglotz and $\Theta$ is Schur on $\Omega\setminus Z(\xi)$. By removability at putative $\xi$--zeros and the (N1) pinch, this extends across $Z(\xi)$; thus $Z(\xi)\cap\Omega=\varnothing$ and RH holds. Consequently, $2\mathcal J$ is Herglotz and $\Theta$ is Schur on $\Omega$.
\end{theorem}
\begin{corollary}[No far-far budget from triangular padding]\label{cor:K-no-FF}
Let $K$ be strictly upper-triangular in the prime basis and independent of $s$. Then its contribution to the far-far Schur budget vanishes: $\Delta_{\mathrm{FF}}^{(K)}=0$.
\end{corollary}
\begin{proof}
In the prime order, $K$ has no entries on or below the diagonal. Hence there are no cycles confined to the far block induced by $K$, and no far$\to$far absolute-sum contribution. Thus the far-far row/column sums are unchanged.
\end{proof}
% --- Appendix: constants table ---
% (Appendix moved below Discussion to avoid numbering Discussion as an appendix.)
% \appendix
% \section*{Appendix: Constants and definitions used in certification}
\begin{table}[H]
\centering
\caption{Compact constants used in the covering and budgets (fixed example values shown).}
\begin{tabular}{l l}
\toprule
Arithmetic energy & $K_0=\tfrac14\sum_{p}\sum_{k\ge2} \dfrac{p^{-k}}{k^2}$ \\ 
Prime cut / minimal prime & $Q=29$, $\ p_{\min}=31$ \\ 
Tail bounds & $\sum_{p>x}p^{-\alpha} \le \dfrac{1.25506\,\alpha}{(\alpha-1)\,\log x}\,x^{\,1-\alpha}$ (for $x\ge 17$) \\ 
Row/col budgets & $\Delta_{SS},\Delta_{SF},\Delta_{FS},\Delta_{FF}$ as in Lemma~\ref{lem:block-gersh} and Lemma~\ref{lem:schur-weyl-gap} \\ 
In-block lower bounds & $\mu^{\mathrm{small}}=1-\Delta_{SS}$, $\ \mu^{\mathrm{far}}=1-\tfrac{L(p_{\min})}{6}$ \\ 
Link barrier & $L(\sigma)=(1-\sigma)(\log p_{\min})\,p_{\min}^{-\sigma}$ \\ 
Lipschitz constant & $K(\sigma)=S_{\sigma+1/2}(Q)+\tfrac14\,p_{\min}^{-\sigma}S_{\sigma}(Q)$ \\ 
Prime sums & $S_{\alpha}(Q)=\sum_{p\le Q} p^{-\alpha}$, $\ T_{\alpha}(p_{\min})=\sum_{p\ge p_{\min}} p^{-\alpha}$ \\ 
\bottomrule
\end{tabular}
\end{table}
\appendix
\section{Carleson embedding constant for fixed aperture}\label{app:CE-constant}
We record a one-time bound for the Carleson-BMO embedding constant with the cone aperture $\alpha$ used throughout. For the Poisson extension $U$ and the area measure $\mu=|\nabla U|^2\,\sigma\,dt\,d\sigma$, the conical square function with aperture $\alpha$ satisfies the Carleson embedding inequality
\[
  \|u\|_{\mathrm{BMO}}\ \le\ \frac{2}{\pi}\,C_{\mathrm{CE}}(\alpha)\,\Big(\sup_I \frac{\mu(Q(\alpha I))}{|I|}\Big)^{\!1/2}.
\]
% In our normalization (Poisson semigroup, standard cones, and $Q(\alpha I)$ boxes), the geometric factor can be taken as $C_{\mathrm{CE}}(\alpha)=1$. Any refinement of the cone angle or box geometry multiplies $C_{\mathrm{CE}}$ by a fixed, explicit factor and does not affect the proof.
\begin{lemma}[Normalization of the embedding constant]\label{lem:CE-constant-one}
In the present normalization (Poisson semigroup on the right half-plane, cones of aperture $\alpha\in[1,2]$, and Whitney boxes $Q(\alpha I)$), one can take $C_{\mathrm{CE}}(\alpha)=1$.
\end{lemma}
\section{VK$\to$annuli$\to C_\xi\to K_\xi$ numeric enclosure}\label{app:vk-annuli-kxi}
Fix $\alpha\in[1,2]$ and the Whitney parameter $c\in(0,1]$. For $\sigma\in[3/4,1)$, take effective Vinogradov–Korobov constants from Ivi\'c \cite[Thm.~13.30]{Ivic}. Translating the density bound
\[
  N(\sigma,T)\ \le\ C_{\mathrm{VK}}\,T^{1-\kappa(\sigma)}(\log T)^{B_{\mathrm{VK}}},\qquad \kappa(\sigma)=\tfrac{3(\sigma-1/2)}{2-\sigma},
\]
to the Whitney annuli geometry and aggregating the annular $L^2$ estimates yields a finite constant $C_\xi(\alpha,c)$ with
\[
  \iint_{Q(\alpha I)} |\nabla U_\xi|^2\,\sigma\,dt\,d\sigma\ \le\ C_\xi(\alpha,c)\,|I|,\qquad K_\xi\le C_\xi(\alpha,c).
\]
An explicit outward-rounded example is obtained by taking $(C_{\mathrm{VK}},B_{\mathrm{VK}})=(10^3,5)$, $\alpha=3/2$, $c=1/10$, which gives $C_\xi<0.160$.
\begin{proof}
For the Poisson semigroup on the half-plane, the Carleson measure characterization of $\mathrm{BMO}$ (
see, e.g., Garnett \cite[Thm.~VI.1.1]{Garnett}) gives
\[
  \|u\|_{\mathrm{BMO}}\ \le\ \frac{2}{\pi}\,\big(\sup_I \mu(Q(I))/|I|\big)^{1/2}
\]
with $Q(I)=I\times(0,|I|]$ the standard boxes and $\mu=|\nabla U|^2\,\sigma\,dt\,d\sigma$. Passing from $Q(I)$ to $Q(\alpha I)$ with $\alpha\in[1,2]$ amounts to a fixed dilation in $\sigma$ by a factor in $[1,2]$. Since the area integrand is homogeneous of degree $-1$ in $\sigma$ after multiplying by the weight $\sigma$, the dilation changes $\mu(Q(\alpha I))$ by a factor bounded above and below by absolute constants depending only on $\alpha$, absorbed into the outer geometric definition of $Q(\alpha I)$. Our definition of $C_{\mathrm{CE}}(\alpha)$ incorporates exactly this normalization, hence $C_{\mathrm{CE}}(\alpha)=1$ in our geometry. (Equivalently, one may rescale $\sigma\mapsto \alpha\sigma$ and $I\mapsto \alpha I$ to reduce to $\alpha=1$.)
\end{proof}
\section{Numerical evaluation of $C_\psi^{(H^1)}$ for the printed window}\label{app:Cpsi-compute}
We record a reproducible computation of the window constant
\[
  C_\psi^{(H^1)}\ :=\ \frac12\int_{\R} S\phi\,dx,\qquad \phi(x):=\psi(x)-\frac{m_\psi}{2}\,\mathbf 1_{[-1,1]}(x),\quad m_\psi:=\int_\R\psi.
\]
Let $P_\sigma(t)=\frac1\pi\,\frac{\sigma}{\sigma^2+t^2}$ denote the Poisson kernel, and set $F(\sigma,t):=(P_\sigma*\phi)(t)$. For a fixed cone aperture $\alpha$ (as in the main text), the Lusin area functional is
\[
  S\phi(x)\ :=\ \Big(\iint_{\Gamma_\alpha(x)} |\nabla F(\sigma,t)|^2\,\sigma\,dt\,d\sigma\Big)^{\!1/2},\qquad \Gamma_\alpha(x):=\{(\sigma,t):|t-x|<\alpha\sigma,\ \sigma>0\}.
\]
Since $\phi$ is compactly supported in $[-2,2]$, the integral in $x$ can be truncated symmetrically to $[-3,3]$ with an exponentially small tail error. Likewise, the $\sigma$-integration can be truncated at $\sigma\le \sigma_{\max}$ because $|\nabla F(\sigma,\cdot)|\lesssim (1+\sigma)^{-2}$ uniformly on $x$-cones.
\paragraph{Interval-arithmetic protocol.} Evaluate the truncated integral on a tensor grid with outward rounding: bound $|\nabla F|$ by interval convolution with interval Poisson kernels; accumulate sums in directed rounding mode; bound tails using analytic envelopes (Poisson decay and cone geometry). Report $C_\psi^{(H^1)}$ as $0.23973\pm 3\times 10^{-4}$ and lock $0.2400$.
\subsection*{Locked Constants (with cross-references)}
\noindent\emph{Policy note.} \textbf{The proof uses the conservative numeric certificate (Cor.~\ref{cor:conservative-closure}) for the quantitative closure.} The box-energy bookkeeping (Lemma~\ref{lem:outer-energy-bookkeeping}) is the structural justification (no $\xi$--only energy; removable singularities) and is not used to lock numbers.
\noindent For the printed window and outer normalization, we record once:
\[
 c_0(\psi)=0.17620819,\quad C_\Gamma=0\ 
\]
With the a.e. wedge, the closing condition is
\[ \pi\Upsilon\ <\ \tfrac{\pi}{2}. \]
Sum-form route: choose \(\kappa=10^{-3}\) so \(C_P=0.002\) and use the analytic envelope bound \(C_H(\psi)\le 0.26\) (Lemma~\ref{lem:CH-explicit}). Then
\[ \frac{C_\Gamma+C_P+C_H}{c_0}=\frac{0+0.002+0.26}{0.17620819}=1.4869<\frac{\pi}{2} \] (archival PSC corollary).
Product-form route (diagnostic display; not used to close (P+)): with the locked value \(C_\psi^{(H^1)}=0.2400\) and \(C_{\mathrm{box}}^{(\zeta)}=\CboxZeta\), we have
\[ M_\psi= \tfrac{4}{\pi}\,C_\psi^{(H^1)}\sqrt{C_{\mathrm{box}}^{(\zeta)}}\ =\ \Mpsilocked,\qquad \Upsilon_{\mathrm{diag}}=\frac{(2/\pi)\cdot \Mpsilocked}{c_0}=\UpsilonLocked.\]
\subsection*{PSC certificate (locked constants; canonical form)}
\noindent\textit{Locked evaluation used throughout (revised; product route via $\Upsilon$):}
\begin{align*}
 (c_0,\ C_H,\ C_\psi^{(H^1)},\ C_{\mathrm{box}})
 &\ =\ (0.17620819,\ 2/\pi,\ 0.2400,\ \CboxZeta),\\
 M_\psi\ &\ =\ \Mpsilocked,\\
 \Upsilon_{\mathrm{diag}}\ &\ =\ \frac{(2/\pi)\cdot \Mpsilocked}{0.17620819}\ =\ \UpsilonLocked. 
\end{align*}
See Appendices~\ref{app:CE-constant}--\ref{app:Cpsi-compute} for derivations and enclosures.
\paragraph{Reproducible numerics (self-contained).}
For the printed window and the \(\zeta\)–normalized route:
\begin{itemize}
\item \(c_0(\psi)\): Poisson plateau infimum (see Appendix~\ref{app:Cpsi-compute}) — exact value with digits
\[ c_0(\psi)=0.17620819. \]
\item \(K_0\): arithmetic tail \(\tfrac14\sum_{p}\sum_{k\ge2} p^{-k}/k^2\) with explicit tail enclosure — locked
\[ K_0=0.03486808. \]
\item \(K_\xi\): Neutralized Whitney–box \(\xi\) energy via annular $L^2$ + VK zero–density — locked (outward-rounded)
% Avoid tautology in symbolic mode; state definition/link only
\[ K_\xi \text{ is the neutralized Whitney energy (see Lemma~\ref{lem:carleson-xi}).} \]
\item \(C_{\mathrm{box}}^{(\zeta)}\): $=K_0+K_\xi$ — used in certificate only
\[ C_{\mathrm{box}}^{(\zeta)}=\CboxZeta. \]
\item \(C_\psi^{(H^1)}\): analytic enclosure $<0.245$ and quadrature $0.23973\pm3\times10^{-4}$; we lock
\[ C_\psi^{(H^1)}=0.2400. \]
\item \(M_\psi\): Fefferman–Stein/Carleson embedding
\[ M_\psi=\tfrac{4}{\pi}\,C_\psi^{(H^1)}\,\sqrt{C_{\mathrm{box}}^{(\zeta)}}\ =\ \Mpsilocked. \]
\item \(\Upsilon\): product certificate value (no prime budget)
\[ \Upsilon_{\mathrm{diag}}\ =\ \frac{(2/\pi)\cdot \Mpsilocked}{0.17620819}\ =\ \UpsilonLocked. \]
\end{itemize}
Each number is computed once and locked with outward rounding. The certificate wedge uses only \(c_0(\psi),\,C(\psi),\,C_{\rm box}^{(\zeta)}\) and the a.e. boundary passage.
\paragraph{Constants table (for quick reference).}
\begin{center}
\begin{tabular}{ll}
\toprule
Symbol & Value/definition \\
\midrule
$c_0(\psi)$ & $\czeroplateau$ (Poisson plateau; see Appendix~\ref{app:Cpsi-compute}) \\
$C_H(\psi)$ & $\CHone$ (Hilbert envelope; analytic envelope used) \\
$C_\psi^{(H^1)}$ & $\CpsiHone$ (locked from quadrature) \\
$K_0$ & $0.03486808$ (arithmetic tail; see Lemma~\ref{lem:carleson-arith}) \\
$K_\xi$ & $\Kxi$ (neutralized Whitney energy) \\
$C_{\mathrm{box}}^{(\zeta)}$ & $\CboxZeta=K_0+K_\xi$ \\
$M_\psi$ & $\Mpsilocked=(4/\pi)\,C_\psi^{(H^1)}\sqrt{C_{\mathrm{box}}^{(\zeta)}}$ \\
\(\Upsilon_{\mathrm{diag}}\) & $\UpsilonLocked=((2/\pi)\,M_\psi)/c_0$ \quad(\emph{diagnostic})\\
\bottomrule
\end{tabular}
\end{center}
\paragraph{Non-circularity (sequencing).}
We first enclose \(K_\xi\) unconditionally from annular $L^2$ and zero–counts, independent of \(M_\psi\). We then evaluate \(M_\psi\) via \((4/\pi)\,C_\psi^{(H^1)}\sqrt{C_{\mathrm{box}}^{(\zeta)}}\) using the locked \(C_{\mathrm{box}}^{(\zeta)}=K_0+K_\xi\). No step uses \(M_\psi\) to bound \(K_\xi\), so there is no feedback.
% ================================================================
%  Stage 2 Closure: PSC ⇒ (P+) and PSC from a locked certificate
% ================================================================

\subsection*{Definitions and standing normalizations}

Let $\Omega:=\{s\in\C:\ \Re s>\tfrac12\}$ and write $s=\tfrac12+it$ on the boundary.
Set
Let $\Poisson_b(x):=\frac{1}{\pi}\frac{b}{b^2+x^2}$ and let $\mathcal H$ denote the boundary Hilbert transform.

\paragraph{Poisson lower bound.}
Define
\[
 c_0(\psi)\ :=\ \inf_{0<b\le 1,\ |x|\le 1}\ (\Poisson_{b}*\psi)(x)\ \ge\ 0.1762081912\,.
\]
For the printed flat--top window this is locked as
\[
  c_0(\psi)=0.17620819.
\]
\subsection*{Product certificate $\Rightarrow$ boundary wedge and (P+)}
\noindent\textit{Route status.} We prove (P+) via the product certificate. PSC sum/density material is archived and not used in the main chain. \emph{Closure uses the quantitative wedge criterion with a Whitney–uniform smallness $\Upsilon_{\mathrm{Whit}}(c)<\tfrac12$ for some small absolute $c$ (no numeric lock), obtained from unconditional bounds on $c_0(\psi)$, $C_\psi^{(H^1)}$, and $C_{\rm box}^{(\zeta)}$.}

Fix an even $C^\infty$ window $\psi$ with $\psi\equiv 1$ on $[-1,1]$, $\operatorname{supp}\psi\subset[-2,2]$, and mass $\int_\R\psi=1$, and set
\[
  \varphi_{L,t_0}(t)\ :=\ \frac{1}{L}\,\psi\!\left(\frac{t-t_0}{L}\right),\qquad \int_{\R}\!\varphi_{L,t_0}=1,\quad \operatorname{supp}\varphi_{L,t_0}\subset I.
\]
On intervals avoiding critical-line ordinates, the a.e. wedge follows directly from the product certificate without additive constants.
\begin{theorem}[Boundary wedge from the product certificate (atom-safe)]\label{thm:psc-certificate-stage2}
For every Whitney interval $I=[t_0-L,t_0+L]$ one has the Poisson plateau lower bound
\[
  c_0(\psi)\,\mu\!\big(Q(I)\big)\ \le\ \int_{\R} (-w')(t)\,\varphi_{L,t_0}(t)\,dt.
\]
Moreover, for every $\phi\in\mathcal W_{\rm adm}(I;\varepsilon)$ from Definition~\ref{def:adm-bumps} (choose the mask to vanish at any critical-line atoms in $I$),
\[
  \int_{\R} \phi(t)\,(-w')(t)\,dt\ \le\ C_{\rm test}(\psi,\varepsilon,\alpha')\,\Big(\iint_{Q(\alpha'I)} |\nabla U|^2\,\sigma\Big)^{1/2}.
\]
By the all-interval Carleson bound, for each $I=[t_0-L,t_0+L]$,
\[
  \int_{\R} \phi(t)\,(-w')(t)\,dt\ \le\ C_{\rm test}(\psi,\varepsilon,\alpha')\,\sqrt{C_{\rm box}^{(\zeta)}}\,L^{1/2}.
\]
Consequently, by Lemma~\ref{lem:local-to-global-wedge} and the schedule clip, the quantitative phase cone holds on all Whitney intervals, hence \eqref{eq:Pplus}.
\end{theorem}
\begin{proof}
The Poisson plateau lower bound holds for $\varphi_{L,t_0}$ by Lemma~\ref{lem:poisson-plateau} and Theorem~\ref{thm:phase-velocity-quant}. The admissible-class upper bound is Proposition~\ref{prop:length-free}. The conclusion \textup{(P+)} follows from Lemma~\ref{lem:whitney-uniform-wedge} and Lemma~\ref{lem:mu-to-lebesgue}.
\end{proof}
% [archived duplicate removed]

\paragraph{Scaling remark (why the density-point contradiction does not follow).}
At a density point $t_*$ of $Q$, the left inequality in \eqref{eq:window-certificate} yields a lower bound $\gtrsim c_0(\psi)\,\mu(Q(I))$, while the CR–Green/Carleson bound gives an upper bound $\lesssim C(\psi)\,\sqrt{C_{\rm box}^{(\zeta)}}\,L^{1/2}$. For $L\downarrow 0$ one has $c_0\,L\le C\,L^{1/2}$, so there is no contradiction from single-interval scaling alone. This is why the proof uses the quantitative wedge criterion with $\Upsilon<\tfrac12$ to conclude (P+).

\begin{remark}
Let $N(\sigma,T)$ denote the number of zeros with $\Re\rho\ge \sigma$ and $0<\Im\rho\le T$. The Vinogradov–Korobov zero-density estimates give, for some absolute constants $C_0,\kappa>0$, that
\[
  N(\sigma,T)\ \le\ C_0\,T\,\log T\ +\ C_0\,T^{1-\kappa(\sigma-1/2)}\qquad (\tfrac12\le \sigma<1,\ T\ge T_1),
\]
with an effective threshold $T_1$. On Whitney scale $L=c/\log\langle T\rangle$, these bounds imply the annular counts used above with explicit $A,B$ of size $\ll 1$ for each fixed $c,\alpha$. Consequently, one can take
\[
  C_\xi\ \le\ C(\alpha,c)\,\big(C_0+1\big)
\]
in Lemma~\ref{lem:carleson-xi}, where $C(\alpha,c)$ is an explicit polynomial in $\alpha$ and $c$ arising from the annular $L^2$ aggregation (cf. Lemma~\ref{lem:annular-balayage}). We do not need the sharp exponents; any effective VK pair $(C_0,\kappa)$ suffices for a finite $C_\xi$ on Whitney boxes.
\end{remark}

% References
\begin{thebibliography}{99}
\bibitem{AmbrosioFuscoPallara} L. Ambrosio, N. Fusco, and D. Pallara, \emph{Functions of Bounded Variation and Free Discontinuity Problems}, Oxford Mathematical Monographs, Oxford University Press, Oxford, 2000. (BV compactness/Helly selection.)
\bibitem{Donoghue} W.~F. Donoghue, Jr., \emph{Monotone Matrix Functions and Analytic Continuation}, Springer, New York, 1974. (Pick/Herglotz functions and positivity.)
\bibitem{DurenHp} P.~L. Duren, \emph{Theory of $H^p$ Spaces}, Academic Press, New York, 1970; reprint, Dover Publications, Mineola, NY, 2000. (Hardy/Smirnov background.)
\bibitem{Dusart2010} P. Dusart, Estimates of some functions over primes without Riemann Hypothesis, arXiv:1002.0442, 2010. (Explicit prime-sum bounds; alternative to Rosser--Schoenfeld.)
\bibitem{FeffermanStein1972} C. Fefferman and E.~M. Stein, $H^p$ spaces of several variables, \emph{Acta Math.} 129 (1972), 137--193. (Fefferman--Stein theory; area/square functions and $H^1$--BMO.)
\bibitem{Garnett} J.~B. Garnett, \emph{Bounded Analytic Functions}, Graduate Texts in Mathematics, vol.~236, revised 1st ed., Springer, New York, 2007. (Thm. VI.1.1: Carleson embedding; Thm. II.4.2: boundary uniqueness; Ch. IV: H$^1$–BMO.)
\bibitem{Ivic} A. Ivi\'c, \emph{The Riemann Zeta-Function: Theory and Applications}, Dover Publications, Mineola, NY, 2003. (Thm. 13.30: VK zero-density, used parametrically.)
\bibitem{RosserSchoenfeld1962} J.~B. Rosser and L. Schoenfeld, Approximate formulas for some functions of prime numbers, \emph{Illinois J. Math.} 6 (1962), no.~1, 64--94. (Explicit bounds; e.g. $\pi(t)\le 1.25506\,t/\log t$ for $t\ge 17$.)
\bibitem{RosserSchoenfeld1975} J.~B. Rosser and L. Schoenfeld, Sharper bounds for the Chebyshev functions $\theta(x)$ and $\psi(x)$, \emph{Math. Comp.} 29 (1975), no.~129, 243--269. (Refined explicit prime bounds.)
\bibitem{RosenblumRovnyak} M. Rosenblum and J. Rovnyak, \emph{Hardy Classes and Operator Theory}, Dover Publications, Mineola, NY, 1997. (Ch. 2: outer/inner and boundary transforms.)
\bibitem{RudinRCA} W. Rudin, \emph{Real and Complex Analysis}, 3rd ed., McGraw--Hill, New York, 1987. (Removable singularities; Poisson integrals.)
\bibitem{SarasonSubHardy} D. Sarason, \emph{Sub-Hardy Hilbert Spaces in the Unit Disk}, John Wiley \& Sons, Inc., New York, 1994. (Schur/Cayley background.)
\bibitem{SimonTrace} B. Simon, \emph{Trace Ideals and Their Applications}, 2nd ed., Mathematical Surveys and Monographs, vol.~120, American Mathematical Society, Providence, RI, 2005. (Hilbert--Schmidt determinants and continuity.)
\bibitem{SteinHA} E.~M. Stein, \emph{Harmonic Analysis: Real-Variable Methods, Orthogonality, and Oscillatory Integrals}, Princeton University Press, Princeton, NJ, 1993. (Poisson/Hilbert transform on $\mathbb R$; square functions.)
\bibitem{Titchmarsh} E.~C. Titchmarsh, \emph{The Theory of the Riemann Zeta-Function}, 2nd ed., revised by D.~R. Heath-Brown, Oxford University Press, Oxford, 1986. (RvM, zero-density background in Ch. VIII--IX.)
\bibitem{IwaniecKowalski} H. Iwaniec and E. Kowalski, \emph{Analytic Number Theory}, Amer. Math. Soc. Colloquium Publications, vol.~53, Amer. Math. Soc., Providence, RI, 2004.
\bibitem{MontgomeryVaughan} H.~L. Montgomery and R.~C. Vaughan, \emph{Multiplicative Number Theory I. Classical Theory}, Cambridge Studies in Advanced Mathematics, vol.~97, Cambridge Univ. Press, Cambridge, 2007.
\bibitem{DavenportMNT} H. Davenport, \emph{Multiplicative Number Theory}, 3rd ed., revised by H.~L. Montgomery, Graduate Texts in Mathematics, vol.~74, Springer-Verlag, New York, 2000.
\bibitem{KoosisLI} P. Koosis, \emph{The Logarithmic Integral I}, Cambridge Studies in Advanced Mathematics, vol.~12, Cambridge Univ. Press, Cambridge, 1988.
\bibitem{Hoffman} K. Hoffman, \emph{Banach Spaces of Analytic Functions}, Dover Publications, Mineola, NY, 2007. (Reprint of the 1962 Prentice--Hall edition.)
\bibitem{CarlesonCorona} L. Carleson, Interpolation by bounded analytic functions and the corona problem, \emph{Ann. of Math.} (2) 76 (1962), 547--559.
\bibitem{SteinSingInt} E.~M. Stein, \emph{Singular Integrals and Differentiability Properties of Functions}, Princeton Mathematical Series, no.~30, Princeton Univ. Press, Princeton, NJ, 1970.
\bibitem{Grafakos} L. Grafakos, \emph{Classical Fourier Analysis}, 3rd ed., Graduate Texts in Mathematics, vol.~249, Springer, New York, 2014.
\bibitem{NISTDLMF} F.~W.~J. Olver, D.~W. Lozier, R.~F. Boisvert, and C.~W. Clark (eds.), \emph{NIST Digital Library of Mathematical Functions}, National Institute of Standards and Technology, Washington, DC, 2010. Available at \url{https://dlmf.nist.gov/}.
\bibitem{Edwards} H.~M. Edwards, \emph{Riemann's Zeta Function}, Academic Press, New York, 1974; reprint, Dover Publications, Mineola, NY, 2001.
\bibitem{AglerMcCarthy} J. Agler and J.~E. McCarthy, \emph{Pick Interpolation and Hilbert Function Spaces}, Graduate Studies in Mathematics, vol.~44, Amer. Math. Soc., Providence, RI, 2002.
\bibitem{Pick1916} G. Pick, \emph{Über die Beschränkungen analytischer Funktionen, welche durch vorgegebene Funktionswerte bewirkt werden}, Math. Ann. 77 (1916), 7--23.
\bibitem{GohbergKrein} I.~C. Gohberg and M.~G. Krein, \emph{Introduction to the Theory of Linear Nonselfadjoint Operators}, Translations of Mathematical Monographs, vol.~18, American Mathematical Society, Providence, RI, 1969.
\bibitem{deMouraLean} L. de Moura, S. Kong, J. Avigad, F. van Doorn, and J. von Raumer, The Lean Theorem Prover (system description), in: \emph{Automated Deduction -- CADE-25}, Lecture Notes in Computer Science, vol.~9195, Springer, Cham, 2015, 378--388.
\bibitem{mathlib} The mathlib Community, The Lean mathematical library, arXiv:1910.09336, 2020.
\end{thebibliography}

\end{document}
